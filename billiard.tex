\chapter{Gluing and billiards}

This lecture is nearly a copy of \cite[Chapter 2]{alexander-kapovitch-petrunin-2019};
here we define upper curvature bound in the sense of Alexandrov, 
prove Reshetnyak's gluing theorem,
and apply it to a problem in billiards.

\section{Geodesics}

The $\CAT$ comparison can be applied to any metric space,
but it is usually applied to geodesic spaces (or complete length spaces).
To simplify the presentation we will assume in addition that the space is \index{proper space}\emph{proper}.
The latter means that any closed ball is compact;
equivalently, the distance function from one (and therefore any) point is proper. 

\begin{thm}{Proposition}\label{ex:CAT-geodesic}
Let $\spc{U}$ be a complete geodesic $\CAT(0)$ space.
Then any two points in $\spc{U}$ are joined by a unique geodesic.
\end{thm}

\parit{Proof.} 
Suppose there are two geodesics between $x$ and $y$.
Then we can choose two points $p\ne q$ on these geodesics such that $\dist{x}{p}{}=\dist{x}{q}{}$ and therefore $\dist{y}{p}{}=\dist{y}{q}{}$.

Observe that the model triangles $[\tilde p\tilde x\tilde y]=\modtrig(pxy)$ and $[\tilde q\tilde x\tilde y]\z=\modtrig(qxy)$ are degenerate and moreover $\tilde p=\tilde q$.
Applying $\CAT(0)$ comparison with $\tilde z=\tilde p=\tilde q$,
we get that $\dist{p}{q}{}=0$, a contradiction.
\qeds

\begin{wrapfigure}{r}{25mm}
\vskip-0mm
\centering
\includegraphics{mppics/pic-750}
\end{wrapfigure}

\begin{thm}{Exercise}\label{ex:noncreasing-CAT}
Given $\hinge p x y$ in a $\CAT(0)$ space $\spc{U}$, consider the function 
\[f\:(\dist{p}{\bar x}{},\dist{p}{\bar y}{})\mapsto \angk p{\bar x}{\bar y},\]
where $\bar x\in\left]p x\right]$ and $\bar y\in\left]p y\right]$.
Show that $f$ is nondecreasing in each argument.

Conclude that any hinge in a $\CAT(0)$ space has defined angle.
\end{thm}

\begin{thm}{Exercise}\label{ex:contractible}
Fix a point $p$ in a  be a complete geodesic $\CAT(0)$ space~$\spc{U}$.
Given a point $x\in \spc{U}$, denote by $\gamma_x\:[0,1]\to \spc{U}$ a (necessarily unique) geodesic path from $p$ to $x$.

Show that the family of maps $h_t\: \spc{U}\to \spc{U}$ defined by 
\[h_t(x)= \gamma_x(t)\]
is a homotopy; it is called \index{geodesic homotopy}\emph{geodesic homotopy}. 
Conclude that $\spc{U}$ is contractible.
\end{thm}

The geodesic homotopy introduced in the previous exercise should help to solve the next one.

\begin{thm}{Exercise}\label{ex:CAT-mnfld=>ext.geod}
Let $\spc{U}$ be a complete geodesic $\CAT(0)$ space.
Assume $\spc{U}$ is a topological manifold.
Show that any geodesic in $\spc{U}$ can be extended 
as a two-side infinite geodesic.
\end{thm}

\section{Thin triangles}

Let us recall the definition of thin triangles.

\begin{thm}{Definition}\label{def:k-thin}
A triangle $\trig{x}{y}{z}$ in the metric space $\spc{U}$ 
is called \index{thin triangle}\emph{thin} if the natural map $\modtrig{}({x}{y}{z})_{\EE^2}\to \trig{x}{y}{z}$ is distance nonincreasing.

{\sloppy 

Analogously, a triangle $\trig{x}{y}{z}$ 
is called \index{spherically thin}\emph{spherically thin} if
the natural map from the spherical model triangle $\modtrig{}({x}{y}{z})_{\SSS^2}$ to $\trig{x}{y}{z}$ is distance nonincreasing.

}
\end{thm}

\begin{thm}{Proposition}\label{prop:thin=cat}
A geodesic space is $\CAT(0)$ 
($\CAT(1)$) 
if and only if 
all its triangles are thin (respectively, all its triangles of perimeter $<2\cdot\pi$ are spherically thin).
\end{thm}

\parit{Proof; if part.} 
Apply  the triangle inequality and thinness of triangles $\trig pxy$ and $\trig qxy$, where $p$, $q$, $x$, and $y$ are as in the definition of the $\CAT(\kappa)$ comparison.

\parit{Only-if part.} 
Applying $\CAT(0)$ comparison to a quadruple $p,q,x,y$ with $q\in [xy]$ shows that any triangle satisfies \index{point-side comparison}\emph{point-side comparison}, that is, the distance from a vertex to a  point on the opposite side is no greater than the corresponding distance in the Euclidean model triangle.  

Now consider a triangle $\trig{x}{y}{z}$ and let $p\in [xy]$ and $q\in [xz]$.
Let $\tilde p$, $\tilde q$ be the corresponding points on the sides of the model triangle $\modtrig({x}{y}{z})_{\EE^2}$.
Applying \ref{ex:noncreasing-CAT}, we get that
\[\angk {x} {y} {z}_{\EE^2} \ge \angk {x} p q _{\EE^2}.\]
Therefore $ \dist{\tilde p}{\tilde q}{\EE^2}\ge \dist{p}{q}{}$.

The $\CAT(1)$ argument is the same.
\qeds

A curve $\gamma\:\II\to \spc{U}$ is called a \index{geodesic!local geodesic}\emph{local geodesic} if for any $t\in\II$ there is a neighborhood $U$ of $t$ in $\II$ such that the restriction $\gamma|_U$ is a  geodesic.

\begin{thm}{Proposition}\label{cor:loc-geod-are-min}
Suppose $\spc{U}$ is a proper geodesic $\CAT(0)$ space.  
Then any local geodesic in $\spc{U}$ is a geodesic.

Analogously, if $\spc{U}$ is a proper geodesic $\CAT(1)$ space, then any local geodesic in $\spc{U}$ which is shorter than $\pi$ is a geodesic.
\end{thm}


\begin{wrapfigure}{r}{25mm}
\vskip-0mm
\centering
\includegraphics{mppics/pic-800}
\end{wrapfigure}

\parit{Proof.}
Suppose $\gamma\:[0,\ell]\to\spc{U}$ is a local geodesic that is not a geodesic.
Choose $a$ to be the maximal value 
such that $\gamma$ is a geodesic on $[0,a]$.
Further, choose $b>a$ so that $\gamma$ is a geodesic on $[a,b]$.

Since the triangle $\trig{\gamma(0)}{\gamma(a)}{\gamma(b)}$ is thin (see the next section) and 
$\dist{\gamma(0)}{\gamma(b)}{}<b$ we have
\[\dist{\gamma(a-\eps)}{\gamma(a+\eps)}{}<2\cdot\eps\]
for all small~$\eps>0$.
That is, $\gamma$ is not length-minimizing on the interval $[a-\eps,a+\eps]$ for any $\eps>0$,
a contradiction.

The spherical case is done in the same way.
\qeds

\begin{thm}{Exercise}\label{ex:convex-distfun}
Let $\spc{U}$ be a complete geodesic space.
Show that $\spc{U}$ is $\CAT(0)$ if and only if the function $f=\tfrac12\cdot\distfun_p^2$ is 1-convex for any $p\in \spc{U}$.
\end{thm}

\begin{thm}{Exercise}\label{ex:convex-dist}
Suppose $\gamma_1,\gamma_2\:[0,1]\to \spc{U}$ are two geodesic paths in a complete geodesic $\CAT(0)$ space $\spc{U}$.
Show that
\[t\mapsto\dist{\gamma_1(t)}{\gamma_2(t)}{\spc{U}}\]
is a convex function.
\end{thm}

\begin{thm}{Exercise}\label{ex:convex-nbhd}
Let $A$ be a convex closed set in a proper geodesic $\CAT(0)$ space $\spc{U}$;
that is, if $x,y\in A$, then $[xy]\subset A$.
Show that for any $r>0$ the closed $r$-neighborhood of $A$ is convex;
that is, the set
\[A_r=\set{x\in \spc{U}}{\distfun_Ax\le r}\]
is convex.
\end{thm}

\begin{thm}{Exercise}\label{ex:closest-point}
Let  $\spc{U}$ be a proper geodesic $\CAT(0)$ space 
and $K\subset \spc{U}$ be a closed convex set.
Show that: 

\begin{subthm}{ex:closest-point:a}
For each point $p\in \spc{U}$ there is a unique point $p^*\in K$ that minimizes the distance $\dist{p}{p^*}{}$.
\end{subthm}

\begin{subthm}{ex:closest-point:b}
The closest-point projection $p\mapsto p^*$ defined by \ref{SHORT.ex:closest-point:a} is short. 
\end{subthm}

\end{thm}

Recall that a set $A$ in a metric space $\spc{U}$ is called \index{locally convex set}\emph{locally convex} if for any point $p\in A$ there is an open neighborhood $\spc{U}\ni p$ such that any geodesic in $\spc{U}$ with  ends in $A$ lies in~$A$. 

\begin{thm}{Exercise}\label{ex:locally-convex}
Let $\spc{U}$ be a proper geodesic $\CAT(0)$ space.
Show that any closed, connected, locally convex set in $\spc{U}$ is convex.
\end{thm}


\section{Inheritance lemma} \label{sec:thin-triangle}

\begin{thm}{Inheritance lemma}
\label{lem:inherit-angle} 
Assume that a triangle $\trig p x y$ 
in a metric space is \index{decomposed triangle}\emph{decomposed} 
into two triangles $\trig p x z$ and $\trig p y z$;
that is, $\trig p x z$ and $\trig p y z$ have a common side $[p z]$, and the sides $[x z]$ and $[z y]$ together form the side $[x y]$ of $\trig p x y$.

\begin{wrapfigure}{r}{25mm}
\vskip-4mm
\centering
\includegraphics{mppics/pic-810}
\end{wrapfigure}

If both triangles $\trig p x z$ and $\trig p y z$ are thin, 
then the triangle $\trig p x y$ is also thin.

Analogously, if $\trig p x y$ has perimeter $<2\cdot\pi$ and both triangles $\trig p x z$ and $\trig p y z$ are spherically thin, then triangle $\trig p x y$ is spherically thin.
\end{thm} 


\parit{Proof.}
Construct  the model triangles $\trig{\dot p}{\dot x}{\dot z}\z=\modtrig(p x z)_{\EE^2}$ 
and $\trig {\dot p} {\dot y} {\dot z}\z=\modtrig(p y z)_{\EE^2}$ so that $\dot x$ and $\dot y$ lie on opposite sides of $[\dot p\dot z]$.

\begin{wrapfigure}{r}{33mm}
\vskip-0mm
\centering
\includegraphics{mppics/pic-821}
\vskip0mm
\end{wrapfigure}

Let us show that 
\[\angk{z}{p}{x}+\angk{z}{p}{y}
\ge
\pi.
\eqlbl{eq:<+<>=pi}\]
If not, then for some point $\dot w\in[\dot p\dot z]$, we have \[\dist{\dot x}{\dot w}{}+\dist{\dot w}{\dot y}{}
<
\dist{\dot x}{\dot z}{}+\dist{\dot z}{\dot y}{}=\dist{x}{y}{}.\]
Let $w\in[p z]$ correspond to $\dot w$; that is, $\dist{z}{w}{}=\dist{\dot z}{\dot w}{}$. 
Since $\trig p x z$ and $\trig p y z$ are thin, we have 
\[\dist{x}{w}{}+\dist{w}{y}{}<\dist{x}{y}{},\]
contradicting the triangle inequality. 

Denote by $\dot D$ the union of two solid triangles $\trig {\dot p}{\dot x}{\dot z}$ and $\trig {\dot p} {\dot y} {\dot z}$.
Further, denote by $\tilde D$ the solid triangle $\trig{\tilde  p}{\tilde  x}{\tilde  y}=\modtrig(p x y)_{\EE^2}$.
By \ref{eq:<+<>=pi}, there is a short map $F\:\tilde D\to \dot D$ that sends 
\begin{align*}
\tilde p&\mapsto \dot p,
&
\tilde x&\mapsto \dot x,
&
\tilde z&\mapsto \dot z,
&
\tilde y&\mapsto \dot y.
\end{align*}

\begin{wrapfigure}{r}{50mm}
\vskip-4mm
\centering
\includegraphics{mppics/pic-830}
\vskip0mm
\end{wrapfigure}

Indeed, by Alexandrov's lemma (\ref{lem:alex}), 
there are nonoverlapping triangles 
\[\trig{\tilde p}{\tilde x}{\tilde z_x}\iso\trig {\dot p}{\dot x}{\dot z}\] 
and 
\[\trig{\tilde p}{\tilde y}{\tilde z_y}\iso\trig {\dot p}{\dot y}{\dot z}\]
inside the  triangle $\trig{\tilde p}{\tilde x}{\tilde y}$.

Connect  the points in each pair
$(\tilde z,\tilde z_x)$, 
$(\tilde z_x,\tilde z_y)$ 
and $(\tilde z_y,\tilde z)$ 
with arcs of circles centered at 
$\tilde y$, $\tilde p$, and $\tilde x$ respectively. 
Define $F$ as follows:
\begin{itemize}

\item Map  $\Conv\trig{\tilde p}{\tilde x}{\tilde z_x}$ isometrically onto  $\Conv\trig {\dot p}{\dot x}{\dot z}$;
similarly map $\Conv \trig{\tilde p}{\tilde y}{\tilde z_y}$ onto $\Conv \trig {\dot p}{\dot y}{\dot z}$.

\item If $x$ is in one of the three circular sectors, say at distance $r$ from its center, set $F(x)$ to be the point on the corresponding segment 
$[p z]$, 
$[x z]$ 
or $[y z]$ whose distance from the left-hand endpoint of the segment is~$r$.

\item Finally, if $x$ lies in the remaining curvilinear triangle $\tilde z \tilde z_x \tilde z_y$, 
set $F(x) = z$. 
\end{itemize}
By construction, $F$ satisfies the conditions.

By assumption, the natural maps $\trig {\dot p} {\dot x} {\dot z}\to\trig p x z$ and $\trig {\dot p} {\dot y} {\dot z}\to\trig p y z$ are short.  
By composition,  the natural map from $\trig{\tilde  p}{\tilde  x}{\tilde  y}$ to $\trig p y z$ is short, as claimed.

The spherical case is done along the same lines.
\qeds

\section{Reshetnyak's gluing}\label{sec:cba-gluing}

Suppose 
$\spc{U}^1$ and $\spc{U}^2$ are proper geodesic spaces 
with isometric closed convex sets $A^i\subset\spc{U}^i$ and let $\iota\:A^1\to A^2$ be an isometry.
Consider the space $\spc{W}$ of all equivalence classes in $\spc{U}^1\sqcup\spc{U}^2$ with the equivalence relation given by $a\sim\iota(a)$ for any $a\in A^1$. %??? repeating

It is straightforward to see that $\spc{W}$ is a proper geodesic space when equipped with the following metric
\begin{align*}
\dist{x}{y}{\spc{W}}&\df\dist{x}{y}{\spc{U}^i}
\\
&\quad\text{if}\quad x,y\in \spc{U}^i,\quad\text{and}
\\
\dist{x}{y}{\spc{W}}&\df\min\set{\dist{x}{a}{\spc{U}^1}+\dist{y}{\iota(a)}{\spc{U}^2}}{a\in A^1}
\\
&\quad\text{if}\quad x\in \spc{U}^1\quad\text{and}\quad y\in \spc{U}^2.
\end{align*}
Abusing notation, we denote by $x$ and $y$ the points in $\spc{U}^1\sqcup\spc{U}^2$ and their equivalence classes in $\spc{U}^1\sqcup\spc{U}^2/{{\sim}}$.

The  space $\spc{W}$ is called the \index{gluing}\emph{gluing} of $\spc{U}^1$ and  $\spc{U}^2$ along~$\iota$.
If one applies this construction to two copies of one space $\spc{U}$ with a set $A\subset \spc{U}$ and the identity map $\iota\:A\to A$, then the obtained space is called the \index{double}\emph{double} of $\spc{U}$ along~$A$.

We can (and will) identify $\spc{U}^i$ with its image in $\spc{W}$;
this way both subsets $A^i\subset \spc{U}^i$ will be identified and denoted further by~$A$.
Note that $A=\spc{U}^1\cap \spc{U}^2\subset \spc{W}$,
therefore $A$ is also a convex set in~$\spc{W}$.

{\sloppy

\begin{thm}{Reshetnyak gluing}\label{thm:gluing}
Suppose 
$\spc{U}^1$ and $\spc{U}^2$ are proper geodesic $\CAT(0)$ spaces 
with isometric 
closed 
 convex
sets $A^i\subset\spc{U}^i$, and $\iota\:A^1\z\to A^2$ is an isometry.
Then the gluing of $\spc{U}^1$ and  $\spc{U}^2$ along $\iota$ is a $\CAT(0)$ proper geodesic space.
\end{thm}

}

\parit{Proof.} 
By construction of the gluing space, the statement can be reformulated in the following way:


\begin{thm}{Reformulation of \ref{thm:gluing}}
Let $\spc{W}$ be a 
proper geodesic space with two closed 
convex sets $\spc{U}^1,\spc{U}^2\subset\spc{W}$ such that 
$\spc{U}^1\cup\spc{U}^2=\spc{W}$
and $\spc{U}^1$, $\spc{U}^2$ are $\CAT(0)$.
Then $\spc{W}$ is $\CAT(0)$.
\end{thm}

\begin{wrapfigure}{o}{45mm}
\vskip-2mm
\centering
\includegraphics{mppics/pic-840}
\end{wrapfigure}

It suffices to show that any triangle $\trig {x}{y}{z}$ 
in $\spc{W}$ is thin.
This is obviously true if all three points $x$, $y$, $z$ lie in one of~$\spc{U}^i$.
Thus, without loss of generality, we may assume that $x\z\in\spc{U}^1$ and $y,z\z\in\spc{U}^2$.


Choose points $a,b\in A\z=\spc{U}^1\z\cap\spc{U}^2$ 
that lie respectively on the sides $[xy], [xz]$.
Note that

\begin{itemize}
\item the triangle $\trig{x}{a}{b}$ lies in $\spc{U}^1$,
\item both triangles $\trig{y}{a}{b}$ and $\trig{y}{b}{z}$ lie in~$\spc{U}^2$.
\end{itemize}
In particular, each triangle $\trig{x}{a}{b}$,
$\trig{y}{a}{b}$, and $\trig{y}{b}{z}$ is thin.

Applying the inheritance lemma (\ref{lem:inherit-angle}) twice, 
we get that $\trig {x}{y}{b}$ 
and consequently $\trig {x}{y}{z}$ is thin.
\qeds

\begin{thm}{Exercise}\label{ex:reshetnyak-doubling}
Suppose $\spc{U}$ is a geodesic space and $A\subset \spc{U}$ is a closed subset.
Assume that the doubling of $\spc{U}$ in $A$ is $\CAT(0)$. 
Show that $A$ is a convex set of $\spc{U}$.
\end{thm}



\section{Puff pastry}\label{sec:puff-pastry}

In this section, we introduce 
the notion of 
Reshetnyak puff pastry.  This  construction will be used in the next section to prove the collision theorem (\ref{thm:collision}).

Let $\bm{A}=(A^1,\dots,A^N)$ be an array of convex closed sets in the Euclidean space~$\EE^m$.
Consider an array of $N+1$ copies of~$\EE^m$.
Assume that the space $\spc{R}$ is 
obtained by
gluing successive pairs of spaces along  $A^1,\dots,A^N$ respectively.


\begin{figure}[ht!]
\vskip-0mm
\centering
\includegraphics{mppics/pic-850}
\caption*{Puff pastry for $(A,B,A)$.}
\end{figure}

The resulting space $\spc{R}$  will be called 
the
\index{puff pastry}\emph{Reshetnyak puff pastry} for array~$\bm{A}$.
The copies of $\EE^m$ in the puff pastry $\spc{R}$
will be called {}\emph{levels};
they will be denoted by $\spc{R}^0,\dots,\spc{R}^N$.
The point in the $k$-th level $\spc{R}^k$
that corresponds to $x\in \EE^m$
will be denoted by~$x^k$.

Given $x\in \EE^m$, any point $x^k\in\spc{R}$ is called a {}\emph{lifting} of~$x$.
The map $x\mapsto x^k$ defines an isometry $\EE^m\to \spc{R}^k$;  
in particular, we can talk about liftings of subsets in~$\EE^m$.

Note that: 
\begin{itemize}
\item The intersection $A^1\cap\dots\cap A^N$ admits a unique lifting in~$\spc{R}$.
\item Moreover, $x^i=x^j$ for some $i<j$
if and only if 
\[x\in A^{i+1}\cap\dots\cap A^j.\]
\item The restriction $\spc{R}^k\to \EE^m$
of the natural projection $x^k\mapsto x$ is an isometry.
\end{itemize}




\begin{thm}{Observation}\label{obs:puff pastry is CAT}
Any Reshetnyak puff pastry is a proper geodesic $\CAT(0)$ space.
\end{thm}

\parit{Proof.} Apply Reshetnyak gluing theorem (\ref{thm:gluing}) recursively for the convex sets in the array.
\qeds

\begin{thm}{Proposition}\label{prop:A-check-A}
Assume $(A^1,\dots,A^N)$ and $(\check A^1,\dots,\check A^N)$ are two arrays of convex closed sets in $\EE^m$ 
such that $ A^k\subset \check A^k$ for each~$k$.
Let $\spc{R}$ and $\check{\spc{R}}$ be the corresponding Reshetnyak  puff pastries.
Then the map $\spc{R}\to\check{\spc{R}}$
defined by $x^k\mapsto\check x^k$ is short.

Moreover, if  
\[\dist{x^i}{y^j}{\spc{R}}=\dist{\check x^i}{\check y^j}{\check{\spc{R}}}\eqlbl{eq:dist=dist}\]
for some $x,y\in \EE^m$ and $i,j\in \{0,\dots,n\}$,
then the unique geodesic $[\check x^i \check y^j]_{\check{\spc{R}}}$ 
is the image of the unique geodesic $[x^i y^j]_{\spc{R}}$
under the map $x^i\mapsto \check x^i$.
\end{thm}

\parit{Proof.}
The first statement in the proposition 
follows from the construction of Reshetnyak  puff pastries.

By Observation~\ref{obs:puff pastry is CAT}, 
$\spc{R}$  and  $\check{\spc{R}}$ are proper geodesic $\CAT(0)$ spaces; 
hence $[x^i y^j]_{\spc{R}}$ 
and $[\check x^i \check y^j]_{\check{\spc{R}}}$ are unique.  
By \ref{eq:dist=dist}, since the map $\spc{R}\to\check{\spc{R}}$ is short, 
the image of $[x^i y^j]_{\spc{R}}$
is a geodesic of $\check{\spc{R}}$ joining $\check x^i$ to~$\check y^j$.
Hence the second statement follows.
\qeds

%??? Sasha suggests to add the following def:

%Say that a puff-pastry is "simple" (or whatever) if there exists a geodesic connecting a point from the first with a point in the last  leave and which does not cut $A_i\cap A_j$  for no $i,j$.

%The theorem about pastry you prove is that if the angles are large there are no high simple puff pastries. 

\begin{thm}{Definition}
Consider a Reshetnyak puff pastry $\spc{R}$ with the levels 
$\spc{R}^0,\dots,\spc{R}^N$.
We say that $\spc{R}$ is \index{end-to-end convex}\emph{end-to-end convex} 
if $\spc{R}^0\cup\spc{R}^N$, the union of its lower and upper levels, 
forms a convex set in~$\spc{R}$;
that is, if $x,y\in \spc{R}^0\cup\spc{R}^N$, then $[xy]_{\spc{R}}\subset  \spc{R}^0\cup\spc{R}^N$.
\end{thm}

Note that if $\spc{R}$ is the Reshetnyak puff pastry for an array of convex sets $\bm{A}=(A^{1},\dots, A^{N})$, 
then $\spc{R}$ is end-to-end convex
if and only if the union of the lower and the upper levels
$\spc{R}^0\cup\spc{R}^N$ is isometric to the double of $\EE^m$ along the nonempty intersection $A^1\cap\dots\cap A^N$.


\begin{thm}{Observation}\label{obs:end-to-end-convex}
Let $\check{\bm{A}}$ and $\bm{A}$ be arrays of convex bodies in~$\EE^m$.
Assume that array $\bm{A}$ is
obtained by inserting in $\check{\bm{A}}$ 
several copies of the bodies which were already listed in~$\check{\bm{A}}$.

For example, if $\check{\bm{A}}=(A,C,B,C,A)$, by placing $B$ in the second place and $A$ in the fourth place, we obtain $\bm{A}=(A,B,C,A,B,C,A)$.

Denote by $\check{\spc{R}}$ and $\spc{R}$ 
the Reshetnyak puff pastries for $\check{\bm{A}}$ and $\bm{A}$ respectively.

If $\check{\spc{R}}$ is end-to-end convex, then so is~$\spc{R}$.
\end{thm}

\parit{Proof.}
Without loss of generality, we may assume that $\bm{A}$ is 
obtained by inserting one element in $\check{\bm{A}}$,
say at the place number~$k$.

Note that $\check{\spc{R}}$ is isometric to the puff pastry 
for $\bm{A}$ with $A^k$ replaced by~$\EE^m$.
It remains to apply Proposition~\ref{prop:A-check-A}.
\qeds

{

\begin{wrapfigure}{o}{30mm}
\vskip-4mm
\centering
\includegraphics{mppics/pic-860}
\end{wrapfigure}

Let $X$ be a convex set in a Euclidean space.
By a \index{dihedral angle}\emph{dihedral angle}, we understand an intersection of two half-spaces;
the intersection of corresponding hyperplanes is called the {}\emph{edge} of the angle.
We say that a dihedral angle $D$ 
supports
 $X$ at a point $p\in X$ 
if $D$ contains $X$ and the edge of $D$ contains~$p$.

}

\begin{thm}{Lemma}\label{lem:end-to-end-convex}
Let $A$ and $B$ be two convex sets in~$\EE^m$.
Assume that any dihedral angle supporting $A\cap B$ has angle measure at least~$\alpha$.
Then the Reshetnyak puff pastry for the array
\[(\underbrace{A,B,A,\dots}_{\text{$\lceil\tfrac\pi\alpha\rceil+1$ times}}).\]
is end-to-end convex. 
\end{thm}


The proof of the lemma is based on a partial case,
which we formulate as a sublemma.

\begin{thm}{Sublemma}\label{sublem:end-to-end-convex}
Let $\ddot A$ and $\ddot B$ be two  
half-planes in $\EE^2$, where $\ddot A\cap \ddot B$ is an angle with measure~$\alpha$.
Then the Reshetnyak puff pastry for the array \[(\underbrace{\ddot A,\ddot B,\ddot A,\dots}_{\text{$\lceil\tfrac\pi\alpha\rceil+1$ times}})\]
is end-to-end convex. 
\end{thm}

\begin{wrapfigure}{r}{35mm}
\vskip0mm
\centering
\includegraphics{mppics/pic-870}
\end{wrapfigure}

\parit{Proof.}
Note that the puff pastry $\ddot{\spc{R}}$ is isometric to the cone over the space glued from the unit circles as shown on the diagram.

All the short arcs on the diagram have length $\alpha$;
the long arcs have length $\pi-\alpha$,
so making a circuit along any path will take~$2\cdot\pi$.

The end-to-end convexity of $\ddot{\spc{R}}$ is equivalent to the fact that any geodesic shorter than $\pi$ with the ends on the inner and the outer circles lies completely in the union of these two circles.

The latter holds if the zigzag line in the picture has length at least~$\pi$.
This line is formed by $\lceil\tfrac\pi\alpha\rceil$ arcs with length $\alpha$ each.
Hence the sublemma.
\qeds

{

\begin{wrapfigure}{r}{26mm}
\vskip-3mm
\centering
\includegraphics{mppics/pic-880}
\end{wrapfigure}

In the proof of \ref{lem:end-to-end-convex}, we will use the following exercise in convex geometry:

\begin{thm}{Exercise}\label{ex:supporting-planes}
Let $A$ and $B$ be two closed convex sets in $\EE^m$ and $A\cap B\ne\emptyset$.
Given two points $x,y\in \EE^m$  let $f(z)=\dist{x}{z}{}+\dist{y}{z}{}$.

Let $z_0\in A\cap B$ be  a point of minimum of $f|_{A\cap B}$.

Show that there are half-spaces $\dot A$ and $\dot B$ such that
$\dot A\supset A$ and $\dot B\supset B$
and $z_0$ is also a point of minimum of the restriction $f|_{\dot A\cap \dot B}$.

\end{thm}

}

\parit{Proof of \ref{lem:end-to-end-convex}.}
Fix arbitrary $x,y\in \EE^m$.
Choose a point $z\in A\cap B$
for which the sum 
\[\dist{x}{z}{}+\dist{y}{z}{}\] 
is minimal.
To show the end-to-end convexity of  $\spc{R}$,
it is sufficient to prove the following:

\begin{clm}{}\label{clm:z in xy}
The geodesic $[x^0y^N]_\spc{R}$ contains $z^0=z^N\in \spc{R}$.
\end{clm}

Without loss of generality, we may assume that $z\in\partial A\cap\partial B$.
Indeed, since the puff pastry for the 1-array $(B)$ is end-to-end convex,
Proposition~\ref{prop:A-check-A} together with \ref{obs:end-to-end-convex}
imply \ref{clm:z in xy} in  case $z$ lies in the interior of~$A$.
The same way we can treat the case when $z$ lies in the interior of~$B$.

Note that $\EE^{m}$ admits 
an
isometric splitting $\EE^{m-2}\times \EE^2$ 
such that 
\begin{align*}
\dot A&=\EE^{m-2}\times \ddot A
\\
\dot B&=\EE^{m-2}\times \ddot B
\end{align*}
where $\ddot A$ and $\ddot B$ are half-planes in~$\EE^2$.

Using Exercise \ref{ex:supporting-planes}, let us replace each $A$ by $\dot A$ and each $B$ by $\dot B$
in the array, to get the array
\[(\underbrace{\dot A,\dot B,\dot A,\dots}_{\text{$\lceil\tfrac\pi\alpha\rceil+1$ times}}).\]
The corresponding puff pastry $\dot{\spc{R}}$
splits as a product of $\EE^{m-2}$ and a puff pastry, 
call it $\ddot{\spc{R}}$,
glued from the copies of the plane $\EE^2$ for the array
\[(\underbrace{\ddot A,\ddot B,\ddot A,\dots}_{\text{$\lceil\tfrac\pi\alpha\rceil+1$ times}}).\]

Note that the dihedral angle $\dot A\cap \dot B$ is at least~$\alpha$.
Therefore the angle measure of  $\ddot A\cap \ddot B$ is also at least $\alpha$.
According to Sublemma~\ref{sublem:end-to-end-convex} and Observation~\ref{obs:end-to-end-convex}, $\ddot{\spc{R}}$ is end-to-end convex.

Since $\dot{\spc{R}}\iso\EE^{m-2}\times\ddot{\spc{R}}$, 
the puff pastry $\dot{\spc{R}}$ is also end-to-end convex.

It follows that the geodesic $[\dot x^0\dot y^N]_{\dot{\spc{R}}}$ contains $\dot z^0=\dot z^N\in\dot{\spc{R}}$.
By Proposition~\ref{prop:A-check-A}, 
the image of $[\dot x^0\dot y^N]_{\dot{\spc{R}}}$ 
under the map $\dot x^k\mapsto x^k$
is the geodesic $[x^0 y^N]_{\spc{R}}$.
Hence \ref{clm:z in xy} and the lemma follow.
\qeds

\section{Wide corners}
\label{sec:wide-corners}

We say that a closed convex set $A\subset \EE^m$ has  \index{$\eps$-wide corners}\emph{$\eps$-wide corners}\label{page:wide corners} for given $\eps >0$ 
if together with each point $p$, 
the set $A$ contains a small right circular cone
with the tip at $p$ and aperture $\eps$;
that is, $\eps$ is the maximum angle between two generating lines of the cone.

For example, 
a plane polygon 
has $\eps$-wide corners
if all its interior angles are at least~$\eps$.

We will consider finite collections of closed convex sets 
$A^1,\dots,A^n\z\subset\EE^m$ 
such that for any subset $F\subset\{1,\dots,n\}$,
the intersection
$\bigcap_{i\in F}A^i$ 
has $\eps$-wide corners.
In this case, we may say briefly \index{$\eps$-wide corners}\textit{all intersections of $A^i$ have $\eps$-wide corners}.


\begin{thm}{Exercise}\label{ex:compact-walls}
Assume $A^1,\dots,A^n\subset\EE^m$ are compact, convex sets with a common interior point.
Show that all intersections of $A^i$ have $\eps$-wide corners for some~$\eps>0$.
\end{thm}

\begin{thm}{Exercise}\label{ex:centrally-simmetric-walls}
Assume $A^1,\dots,A^n\subset\EE^m$ are
convex sets with nonempty interiors that have a common center of symmetry.
Show that all intersections of $A^i$ have $\eps$-wide corners for some~$\eps>0$.
\end{thm}

The proof of the following proposition is based on \ref{lem:end-to-end-convex};
this lemma is essentially the case $n=2$ in the proposition.

\begin{thm}{Proposition}\label{prop:end-to-end-convex}
Given $\eps>0$ and a positive integer $n$, 
there is an array of integers $\bm{j}_\eps(n)=(j_1,\dots,j_N)$ 
such that: 

\begin{subthm}{} For each $k$ we have $1\le j_k\le n$,
and each number $1,\dots,n$ appears in $\bm{j}_\eps$ at least once.
\end{subthm}

\begin{subthm}{}
If $A^1,\dots,A^n$ is a collection of closed convex sets in $\EE^m$ with a common point 
and all their intersections have $\eps$-wide corners,  
then the puff pastry for the array
$(A^{j_1},\dots,A^{j_N})$ is end-to-end convex.
\end{subthm}

Moreover, we can assume that $N\le (\lceil\tfrac\pi\eps\rceil+1)^n$.
\end{thm}

\parit{Proof.}
The array $\bm{j}_\eps(n)=(j_1,\dots,j_N)$  is constructed recursively.
For $n=1$, we can take $\bm{j}_\eps(1)=(1)$.

Assume that $\bm{j}_\eps(n)$ is constructed.
Let us replace each occurrence of $n$ in $\bm{j}_\eps(n)$ by the alternating string 
\[\underbrace{n,n+1,n,\dots}_{\text{$\lceil\tfrac\pi\eps\rceil+1$ times}}.\]
Denote the obtained array by $\bm{j}_\eps(n+1)$.

By Lemma \ref{lem:end-to-end-convex},
the end-to-end convexity of the puff pastry for $\bm{j}_\eps(n\z+1)$
follows from the end-to-end convexity of the puff pastry for the array
where each string
\[\underbrace{A^n,A^{n+1},A^n,\dots}_{\text{$\lceil\tfrac\pi\eps\rceil+1$ times}}\]
is replaced by  $Q=A^n\cap A^{n+1}$.
End-to-end convexity of the latter follows by the assumption on $\bm{j}_\eps(n)$, 
since all the intersections of $A^1,\dots,A^{n-1},Q$
have $\eps$-wide corners.

The upper bound on $N$ follows directly from the construction.
\qeds



\section{Billiards}

Let $A^1,A^2,\dots A^n$ be a finite collection of closed convex sets in~$\EE^m$.
Assume that for each $i$
the boundary $\partial A^i$ is a smooth hypersurface.

Consider the billiard table formed by the closure of the complement 
$$T=\overline{\EE^m\setminus \bigcup_{i} A^i}.$$
The sets $A^i$ will be called {}\emph{walls} of the table
and the billiards described above will be called {}\emph{billiards with convex walls}.

A billiard {}\emph{trajectory} 
on the table is a unit-speed broken line $\gamma$ 
that  follows  the
standard law of billiards 
at the breakpoints on $\partial A^i$ 
--- in particular, the angle of reflection is equal to the angle of incidence.
The breakpoints of the trajectory will be called {}\emph{collisions}.
We assume the trajectory meets only one wall at a time.

Recall that the definition of sets with $\eps$-wide corners is given in \ref{sec:wide-corners}.

\begin{thm}{Collision theorem}\label{thm:collision}
Assume $T\subset\EE^m$ is a billiard table with $n$ convex walls.
Assume that the walls of $T$ have a common interior point  and all their intersections have $\eps$-wide corners.
Then the number of collisions of any trajectory in  $T$  is bounded
by a number $N$ which depends only on $n$ and~$\eps$.
\end{thm}

As we will see from the proof,
the value $N$ can be found explicitly;
$N=(\lceil\tfrac\pi\eps\rceil+1)^{n^2}$
will do.

\begin{thm}{Corollary}\label{cor:balls}
Consider $n$ homogeneous hard balls
moving freely and colliding
elastically in~$\RR^3$. 
Every ball moves
along a straight line with constant speed until two balls collide, and then
the new velocities of the two balls are determined by the
laws of classical mechanics. 
We assume that only two balls can collide at the same time.

Then the total number of collisions cannot exceed some number $N$ that  depends on the radii and masses of the balls.
If the balls are identical, then $N$ depends only on~$n$.
\end{thm}

\begin{thm}{Exercise}\label{cor:balls:dim=1}
Show that in the case of identical balls in the one-dimensional space (in $\RR$) 
the total number of collisions cannot exceed $N=\tfrac{n\cdot(n-1)}2$.
\end{thm}



The proof below admits a straightforward generalization to all dimensions.

\parit{Proof of \ref{cor:balls} modulo \ref{thm:collision}.}
Denote by $a_i=(x_i,y_i,z_i) \in \RR^3$ the center of the $i$-th ball.
Consider the corresponding point in $\RR^{3\cdot N}$
\begin{align*}
\bm{a}&=(a_1, a_2 , \dots , a_n ) =
\\
&=(x_1, y_1 , z_1 , x_2 , y_2 , z_2 , \dots , x_n , y_n , z_n).
\end{align*}

The $i$-th and $j$-th balls intersect if 
$$|a_i - a_j | \le R_i+R_j,$$
where $R_i$ denotes the radius of the $i$-th ball.
These inequalities define $\tfrac{n\cdot(n-1)}{2}$ cylinders 
\[C_{i,j}=\set{(a_1, a_2 , \dots , a_n )\in\RR^{3\cdot n}} {|a_i - a_j |\le R_i+R_j}.\] 
The closure of the complement
\[T=\overline{\RR^{3\cdot n}\setminus \bigcup_{i< j} C_{i,j}}\] 
is the configuration space of our system. 
Its points correspond
to valid positions of the system of balls.

The evolution of the system
of balls is described by the motion of
the point $\bm{a}\in\RR^{3\cdot n}$.
It moves along a straight line at a
constant speed until it hits one of the cylinders $C_{i,j}$; 
this event corresponds
to a collision in the system of balls.

Consider the norm of $\bm{a}=(a_1,\dots,a_n)\in \RR^{3\cdot n}$ defined by
\[\lVert \bm{a}\rVert
=
\sqrt{M_1\cdot|a_1|^2+\dots+M_n\cdot |a_n|^2},\]
where $|a_i|=\sqrt{x_i^2+y_i^2+z_i^2}$ 
and $M_i$ denotes the mass of the $i$-th ball.
In the metric defined by $\lVert {*}\rVert$,
the collisions follow the
standard law of billiards. 

By construction, the number of collisions of hard balls that we need to estimate 
is the same as the number of collisions of the corresponding billiard trajectory on the table with $C_{i,j}$ as the walls.

Note that each cylinder $C_{i,j}$ is a convex set;
it has smooth boundary, 
and it is centrally symmetric around the origin.
By \ref{ex:centrally-simmetric-walls}, all the intersections of the walls have $\eps$-wide corners for some $\eps>0$ that depend on the radiuses $R_i$ and the masses~$M_i$.
It remains to apply the collision theorem (\ref{thm:collision}).
\qeds

Now we present the proof of the collision theorem (\ref{thm:collision})
based on the results developed in the previous section.

\parit{Proof of \ref{thm:collision}.}
Let us apply induction on~$n$.

\parit{Base: $n=1$.}
The number of collisions cannot exceed~1.  
Indeed, by the convexity of $A^1$,
if the trajectory is reflected once in $\partial A^1$, 
then it cannot return to~$A^1$. 

\parit{Step.}
Assume $\gamma$ is a trajectory that  meets the walls in the order $A^{i_1},\dots,A^{i_N}$ for a large integer~$N$.

Consider the array 
\[\bm{A}_\gamma=(A^{i_1},\dots,A^{i_N}).\]
The induction hypothesis implies:

\begin{clm}{}\label{clm:collision-induction hypothesis}
There is a positive integer $M$ such that any $M$ consecutive elements of $\bm{A}_\gamma$ contain each $A^i$ at least once.
\end{clm}

Let $\spc{R}_\gamma $ be  the  Reshetnyak puff pastry for~$\bm{A}_\gamma$.

Consider the lift of $\gamma$ to $\spc{R}_\gamma$,
defined by 
$\bar\gamma(t)=\gamma^k(t)\in \spc{R}_\gamma$ 
for any moment of time $t$ between the $k$-th and $(k+1)$-th collisions.  
Since $\gamma$ follows the standard law of billiards at breakpoints, the lift $\bar\gamma$ is locally a geodesic in~$\spc{R}_\gamma$.
By \ref{obs:puff pastry is CAT},
the puff pastry $\spc{R}_\gamma$ is a proper geodesic $\CAT(0)$ space.
Therefore $\bar\gamma$ is a geodesic.

Since $\gamma$ does not meet $A^1\cap\dots\cap A^n$,
the lift $\bar\gamma$ does not lie in  $\spc{R}_\gamma^0\cup \spc{R}_\gamma^N$.
In particular, $\spc{R}_\gamma$ is not end-to-end convex.

Let 
\[\bm{B}=(A^{j_1},\dots,A^{j_K})\] 
be the array provided by Proposition~\ref{prop:end-to-end-convex};
so $\bm{B}$ contains each $A^i$ at least once
and the puff pastry $\spc{R}_{\bm{B}}$ for $\bm{B}$ is end-to-end convex.
If $N$ is sufficiently large, namely $N\ge K\cdot M$, then
 \ref{clm:collision-induction hypothesis}
implies that $\bm{A}_\gamma$ can be obtained  
by inserting a finite number of $A^i$'s in~$\bm{B}$.

By \ref{obs:end-to-end-convex}, 
$\spc{R}_\gamma$ is end-to-end convex --- a contradiction.
\qeds

\section{Comments}


The gluing theorem (\ref{thm:gluing}) was proved by Yuri Reshetnyak~\cite{reshetnyak:glue}.
It can be extended to all geodesic $\CAT(0)$ spaces.
It also admits a natural generalization to 
geodesic $\CAT(\kappa)$
spaces;
see the book of Martin Bridson and  Andr\'e Haefliger \cite{bridson-haefliger} and our book \cite{alexander-kapovitch-petrunin-2025} for details.

The collision theorem (\ref{thm:collision}) was proved by Dmitri Burago, Serge Ferleger and Alexey Kononenko \cite{burago-ferleger-kononenko-1997}.
Its corollary (\ref{cor:balls}) answers a question posed by Yakov Sinai \cite{galperin}.
Puff pastry is used to bound topological entropy of the billiard flow 
and to approximate the shortest billiard path that  touches given lines in a given order; 
see the papers of Dmitri Burago with  Serge Ferleger,
and Alexey Kononenko~\cite{burago-ferleger-kononenko-1998}, and with
Dimitri Grigoriev and Anatol Slissenko~\cite{burago-grigoriev-slissenko}.
The lecture of Dmitri Burago \cite{burago-1998} gives a short survey on the subject.

Note that the interior points of the walls play a key role in the proof
despite that the trajectories never go inside the walls.
In a similar fashion, puff pastry was used by Stephanie Alexander and Richard Bishop \cite{alexander-bishop}
to find the upper curvature bound for warped products.

Joel Hass \cite{hass} constructed an example of a Riemannian metric on the 3-ball with negative curvature and concave boundary.
This example might decrease your appetite for generalizing the collision theorem --- while
locally such a 3-ball looks as good as the billiards table in the theorem, the number of collisions is obviously infinite.

It was shown by Dmitri Burago and Sergei Ivanov \cite{burago-ivanov} that the number of collisions that may occur between $n$ identical balls in $\RR^3$ grows at least exponentially in $n$; the two-dimensional remains open.
