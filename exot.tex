\chapter{Polyhedral spaces}\label{chap:poly}

This lecture is nearly a copy of \cite[Sections 3.4--3.8]{alexander-kapovitch-petrunin-2019};
here we describe a set of rules for gluing Euclidean cubes that produce a 
locally $\CAT(0)$ space and use these rules to construct exotic examples of aspherical manifolds.

\section{Products, cones, and suspension}
\label{sec:Products and cones}

Given two metric spaces $\spc{U}$ and $\spc{V}$, the \index{product space}\emph{product space} 
$\spc{U}\times\spc{V}$ is defined as the set of all pairs $(u,v)$ where $u\in\spc{U}$ and $v\in \spc{V}$ 
with the metric defined by Pythagorean theorem
\[\dist{(u_1,v_1)}{(u_2,v_2)}{\spc{U}\times\spc{V}}=\sqrt{\dist[2]{u_1}{u_2}{\spc{U}}+\dist[2]{v_1}{v_2}{\spc{V}}}.\]

\begin{thm}{Proposition}\label{ex:product-CAT}
Let $\spc{U}$ and $\spc{V}$ be $\CAT(0)$ spaces.
Then the product space $\spc{U}\times\spc{V}$ is $\CAT(0)$.
\end{thm}

\parit{Proof.}
Fix a quadruple in $\spc{U}\times \spc{V}$:
\begin{align*}
p&=(p_1,p_2),
&
q&=(q_1,q_2), 
&
x&=(x_1,x_2),
&
y&=(y_1,y_2).
\end{align*}
For the quadruple $p_1,q_1,x_1,y_1$ in $\spc{U}$,
construct two model triangles $\trig{\tilde p_1}{\tilde x_1}{\tilde y_1}=\modtrig(p_1x_1y_1)_{\EE^2}$ 
and $\trig{\tilde q_1}{\tilde x_1}{\tilde y_1}=\modtrig(q_1x_1y_1)_{\EE^2}$.  
Similarly, for the quadruple $p_2,q_2,x_2,y_2$ in $\spc{V}$
construct two model triangles $\trig{\tilde p_2}{\tilde x_2}{\tilde y_2}$ and $\trig{\tilde q_2}{\tilde x_2}{\tilde y_2}$.

Consider four points in $\EE^4=\EE^2\times\EE^2$ 
\begin{align*}
\tilde p&=(\tilde p_1,\tilde p_2),
&
\tilde q&=(\tilde q_1,\tilde q_2),
&
\tilde x&=(\tilde x_1,\tilde x_2),
&
\tilde y&=(\tilde y_1,\tilde y_2).
\end{align*}
Note that the triangles $\trig{\tilde p}{\tilde x}{\tilde y}$ and $\trig{\tilde q}{\tilde x}{\tilde y}$ in $\EE^4$ are isometric to the model triangles 
$\modtrig(pxy)_{\EE^2}$ and $\modtrig(qxy)_{\EE^2}$.

If $\tilde z=(\tilde z_1,\tilde z_2)\in [\tilde x\tilde y]$, then $\tilde z_1\in [\tilde x_1\tilde y_1]$ and $\tilde z_2\in [\tilde x_2\tilde y_2]$ and
\begin{align*}
\dist[2]{\tilde z}{\tilde p}{\EE^4}&=\dist[2]{\tilde z_1}{\tilde p_1}{\EE^2}+\dist[2]{\tilde z_2}{\tilde p_2}{\EE^2},
\\
\dist[2]{\tilde z}{\tilde q}{\EE^4}&=\dist[2]{\tilde z_1}{\tilde q_1}{\EE^2}+\dist[2]{\tilde z_2}{\tilde q_2}{\EE^2},
\\
\dist[2]{p}{q}{\spc{U}\times\spc{V}}&=\dist[2]{p_1}{q_1}{\spc{U}}+\dist[2]{p_2}{q_2}{\spc{V}}.
\end{align*}
Therefore $\CAT(0)$ comparison for the quadruples $p_1,q_1,x_1,y_1$ in $\spc{U}$
and 
$p_2,q_2,x_2,y_2$ in $\spc{V}$ implies 
$\CAT(0)$ comparison for the quadruples $p,q,x,y$ in $\spc{U}\times \spc{V}$.
\qeds

\begin{thm}{Exercise}\label{ex:product-CBB}
Assume $\spc{U}$ and $\spc{V}$ are $\CBB(0)$ spaces.
Show that the product space $\spc{U}\times\spc{V}$ is $\CBB(0)$.
\end{thm}

Recall that \label{page:cone}\index{cone}\emph{cone} $\spc{V}=\Cone\spc{U}$ over a metric space $\spc{U}$
is defined as the metric space whose underlying set consists of
equivalence classes in
$[0,\infty)\times \spc{U}$ with the equivalence relation ``$\sim$'' given by $(0,p)\sim (0,q)$ for any points $p,q\in\spc{U}$,
and whose metric is given by the cosine rule
\[
\dist{(p,s)}{(q,t)}{\spc{V}} 
=
\sqrt{s^2+t^2-2\cdot s\cdot t\cdot \cos\alpha},
\]
where $\alpha= \min\{\pi, \dist{p}{q}{\spc{U}}\}$.
Points in $\spc{V}$ might be called \emph{vectors},
they come with the norm, scalar product, and multiplication by nonnegative reals.
The space $\spc{U}$ can be identified with the subset $x\in \spc{V}$ such that $|x|=1$.

\begin{thm}{Proposition}\label{ex:cone+susp}
Let $\spc{U}$ be a metric space.
Then $\Cone\spc{U}$ is  $\CAT(0)$ if and only if $\spc{U}$ is $\CAT(1)$.
\end{thm}

\parit{Proof; if part.}
Given a point $x\in \Cone\spc{U}$, denote by $x'$ its projection to $\spc{U}$
and by $|x|$ the distance from $x$ to the tip of the cone;
if $x$ is the tip, then $|x|=0$ and we can take any point of $\spc{U}$ as~$x'$.

Let $p$, $q$, $x$, $y$
be a quadruple in $\Cone\spc{U}$.
Assume that the spherical model triangles $\trig{\tilde p'}{\tilde x'}{\tilde y'}_{\SSS^2}=\modtrig(p'x'y')_{\SSS^2}$ and $\trig{\tilde q'}{\tilde x'}{\tilde y'}_{\SSS^2}=\modtrig(q'x'y')_{\SSS^2}$ are defined.
Consider the following points in $\EE^3=\Cone\SSS^2$: 
\begin{align*}
\tilde p&=|p|\cdot\tilde p',
&
\tilde q&=|q|\cdot\tilde q',
&
\tilde x&=|x|\cdot\tilde x',
&
\tilde y&=|y|\cdot\tilde y'.
\end{align*}

Note that
$\trig{\tilde p}{\tilde x}{\tilde y}_{\EE^3}\iso\modtrig(pxy)_{\EE^2}$
and
$\trig{\tilde q}{\tilde x}{\tilde y}_{\EE^3}\iso\modtrig(qxy)_{\EE^2}$.
Further, note that if $\tilde z\in [\tilde x\tilde y]_{\EE^3}$, then
$\tilde z'=\tilde z/|\tilde z|$ lies on the geodesic $[\tilde x'\tilde y']_{\SSS^2}$.
Therefore the $\CAT(1)$ comparison for $\dist{p'}{q'}{}$ with $\tilde z'\in[\tilde x'\tilde y']_{\SSS^2}$ implies the 
$\CAT(0)$ comparison for $\dist{p}{q}{}$ with $\tilde z\in[\tilde x\tilde y]_{\EE^3}$.

If at least one of the model triangles $\modtrig(p'x'y')_{\SSS^2}$ and $\modtrig(q'x'y')_{\SSS^2}$ is undefined,
then the statement follows from the triangle inequalities 
\begin{align*}
|p'-x'|_{\spc{U}}+|q'-x'|_{\spc{U}}
&\ge |p'-q'|_{\spc{U}}
\\
|p'-y'|_{\spc{U}}+|q'-y'|_{\spc{U}}
&\ge |p'-q'|_{\spc{U}}
\end{align*}
This case is left as an exercise. %???

\parit{Only-if part.}
Suppose that $\tilde p'$, $\tilde q'$, $\tilde x'$, $\tilde y'$ are defined as above.
Assume all these points lie in a half-space of $\EE^3=\Cone\SSS^2$ with origin at its boundary. 
Then we can choose positive values $a$, $b$, $c$, and $d$ such that the points $a\cdot \tilde p'$, $b\cdot \tilde q'$, $c\cdot \tilde x'$, $d\cdot \tilde y'$ lie in one plane.
Consider the corresponding points $a\cdot  p'$, $b\cdot  q'$, $c\cdot  x'$, $d\cdot y'$ in $\Cone\spc{U}$.
Applying the $\CAT(0)$ comparison for these points leads to $\CAT(1)$ comparison for the quadruple $ p'$, $q'$, $ x'$, $ y'$ in $\spc{U}$.

It remains to consider the case when $\tilde p'$, $\tilde q'$, $\tilde x'$, $\tilde y'$ do not in a half-space.
Fix $\tilde z'\in [\tilde x' \tilde y']_{\SSS^2}$.
Observe that 
\begin{align*}\dist{\tilde p'}{\tilde x'}{\SSS^2}+\dist{\tilde q'}{\tilde x'}{\SSS^2}
 &
\le \dist{\tilde p'}{\tilde z'}{\SSS^2}+\dist{\tilde q'}{\tilde z'}{\SSS^2}
\intertext{or} 
\dist{\tilde p'}{\tilde y'}{\SSS^2}+\dist{\tilde q'}{\tilde y'}{\SSS^2}
&\le
\dist{\tilde p'}{\tilde z'}{\SSS^2}+\dist{\tilde q'}{\tilde z'}{\SSS^2}.\end{align*}
That is, in this case, the $\CAT(1)$ comparison follows from the triangle inequality.
\qeds

A similar argument gives the following,
but the proof requires the globalization theorem that will be proved much latter.

\begin{thm}{Proposition}\label{ex:cone+CBB}
Let $\spc{L}$ be a geodesic space.
Then $\Cone\spc{L}$ is  $\CBB(0)$ if and only if $\spc{L}$ is $\CBB(1)$.
\end{thm}

Suspension is a spherical analog of cone construction.

The \index{suspension}\emph{suspension} $\spc{V}=\Susp\spc{U}$ over a metric space $\spc{U}$
is defined as the metric space whose underlying set consists of equivalence classes in
$[0,\pi]\times \spc{U}$ with the equivalence relation ``$\sim$'' given by $(0,p)\z\sim (0,q)$ and $(\pi,p)\z\sim (\pi,q)$ for any points $p,q\in\spc{U}$,
and whose metric is given by the  spherical cosine rule
\[
\cos\dist{(p,s)}{(q,t)}{\Susp\spc{U}} 
=
\cos s\cdot\cos t-\sin s\cdot\sin t\cdot\cos\alpha,
\]
where $\alpha= \min\{\pi, \dist{p}{q}{\spc{U}}\}$.

The points in $\spc{V}$ formed by the equivalence classes of $0\times\spc{U}$ and $\pi\times\spc{U}$ are called  the {}\emph{north} and the \index{pole of suspension}\emph{south poles} of the suspension.

\begin{thm}{Exercise}\label{ex:product-cone}
Let $\spc{U}$ be a metric space.
Show that the spaces 
\[\RR\times \Cone\spc{U}\quad\text{and}\quad\Cone[\Susp\spc{U}]\]
are  isometric.
\end{thm}

The following statement is a direct analog of \ref{ex:cone+susp} and it can be proved along the same lines.

\begin{thm}{Proposition}\label{prop:susp}
Let $\spc{U}$ be a metric space.
Then a neighborhood $\spc{N}$ of the north in $\Susp\spc{U}$ is  $\CAT(1)$ if and only if $\spc{U}$ is $\CAT(1)$.
\end{thm}

\section{Polyhedral spaces}

\begin{thm}{Definition}\label{def:poly}
A geodesic space $\spc{P}$ is called  
a \index{polyhedral space}\emph{(spherical) polyhedral space} 
if it admits a finite triangulation $\tau$ 
such that every simplex in $\tau$ is isometric to a simplex in a Euclidean space (or respectively a unit sphere) of appropriate dimension.

By \index{triangulation of a polyhedral space}\emph{triangulation of a polyhedral space}, 
we will always understand a triangulation as above. 
\end{thm}

Note that according to the above definition,
all polyhedral spaces are compact.

The \index{dimension of a polyhedral space}\emph{dimension} of a polyhedral space $\spc{P}$
is defined as the maximal dimension of the simplices 
in one (and therefore any) triangulation of~$\spc{P}$.

\parbf{Links.}
Let $\spc{P}$ be a polyhedral space
and $\sigma$ be a simplex in a triangulation $\tau$ of~$\spc{P}$.

The simplices that  contain $\sigma$
form an abstract simplicial complex called the \index{link}\emph{link} of $\sigma$, 
denoted by $\Link_\sigma$.
If $m$ is  the dimension of $\sigma$,
then the set of vertices of $\Link_\sigma$
is formed by the $(m+1)$-simplices that  contain $\sigma$;
the set of its edges is formed by the $(m+2)$-simplices 
that contain $\sigma$; and so on.

The link $\Link_\sigma$
can be identified with the subcomplex of $\tau$ 
formed by all the simplices $\sigma'$ 
such that $\sigma\cap\sigma'=\emptyset$ 
but both $\sigma$ and $\sigma'$ are faces of a simplex of~$\tau$.

The points in $\Link_\sigma$ can be identified with the normal directions to $\sigma$ at a point in its interior.
The angle metric between directions makes  $\Link_\sigma$ into a spherical polyhedral space.
We will always consider the link with this metric.

\parbf{Tangent space and space of directions.}
Let $\spc{P}$ be a polyhedral space (Euclidean or spherical) and  $\tau$ be its triangulation.
If a point $p\in \spc{P}$ 
lies in the interior of a $k$-simplex $\sigma$ of $\tau$ 
then the tangent space $\T_p=\T_p\spc{P}$
is  naturally isometric to
\[\EE^k\times(\Cone\Link_\sigma).\]

If $\spc{P}$ is an $m$-dimensional polyhedral space,
then for any $p\in \spc{P}$
the space of directions $\Sigma_p$ is a spherical polyhedral space
of dimension at most $m-1$. 

In particular, 
for any point $p$ in $\sigma$,
the isometry class of $\Link_\sigma$ together with $k=\dim\sigma$
determines the isometry class of $\Sigma_p$, 
 and the other way around --- $\Sigma_p$ and $k$ determines the isometry class of $\Link_\sigma$.

A small neighborhood of $p$ is isometric to a neighborhood of the tip of $\Cone\Sigma_p$. 
In fact, if this property holds at any point of a compact length space $\spc{P}$,
then  $\spc{P}$ is a polyhedral space \cite{lebedeva-petrunin}.

\section{CAT test}

The following theorem provides a combinatorial description of polyhedral spaces with curvature bounded above.


\begin{thm}{Theorem}\label{thm:PL-CAT}
Let $\spc{P}$ be a polyhedral space and $\tau$ be its triangulation. 
Then $\spc{P}$ is locally $\CAT(0)$ if and only if the link of each simplex in $\tau$ has no closed local geodesic shorter than $2\cdot\pi$.

Analogously, let $\spc{P}$ be a spherical polyhedral space and $\tau$ be its triangulation. 
Then $\spc{P}$ is $\CAT(1)$ if and only if neither $\spc{P}$ nor  the  link of any simplex in $\tau$ has a closed local geodesic shorter than $2\cdot\pi$.
\end{thm}





\parit{Proof.}
The ``only if'' part follows from \ref{cor:loc-geod-are-min}, \ref{prop:susp}, and \ref{ex:cone+susp}.

To prove the ``if'' part,
we apply induction on $\dim\spc{P}$.
The base case $\dim\spc{P}=0$ is evident.
Let us start with the $\CAT(1)$ case.

\parit{Step.}
Assume that the theorem is proved in the case $\dim\spc{P}<m$. Suppose  $\dim\spc{P}=m$.


Fix a point $p\in\spc{P}$.
A neighborhood of $p$ 
is isometric to a neighborhood of the north pole in the suspension over the space of directions~$\Sigma_p$.

Note that $\Sigma_p$ is a spherical polyhedral space, 
and its  links are isometric to  links of~$\spc{P}$. 
By the  induction hypothesis, $\Sigma_p$ is $\CAT(1)$.
Thus, by the second part of Exercise~\ref{ex:cone+susp}, $\spc{P}$ is locally  $\CAT(1)$.


Applying the second part of Corollary~\ref{cor:closed-geod-cat},
we get the statement.

The $\CAT(0)$ case is done in exactly the same way except we need to use the first part of Exercise~\ref{ex:cone+susp} and  the first part of Corollary~\ref{cor:closed-geod-cat} on the last step.
\qeds

\begin{thm}{Exercise}\label{ex:unique-geod=CAT}
Let $\spc{P}$
be a polyhedral space such that any two points can be connected by a unique geodesic.
Show that $\spc{P}$ is $\CAT(0)$.
\end{thm}

\begin{thm}{Advanced exercise}\label{ex:S3}
Construct a Euclidean polyhedral metric on $\mathbb{S}^3$
such that the total angle around each edge in its triangulation is at least $2\cdot \pi$.
\end{thm}


Let us formulate an analogous test for spaces with lower curvature bound.

\begin{thm}{Theorem}\label{thm:poly-CBB} Let $\spc{P}$ be a polyhedral space and $\tau$ be a triangulation of $\spc{P}$.
Then $\spc{P}$ is $\CBB(0)$ if and only if the following conditions hold.

\begin{subthm}{} $\tau$ is \index{pure complex}\emph{pure}; 
that is, any simplex in $\tau$ is  a face of some simplex of dimension exactly $m$. 
\end{subthm}

\begin{subthm}{thm:poly-CBB:m-1}
The link of any simplex of dimension $m-1$ is formed by single point or two points.
\end{subthm}

\begin{subthm}{thm:poly-CBB:connected}
The link of any simplex of dimension $\le m-2$ is connected.
\end{subthm}

\begin{subthm}{thm:poly-CBB:2pi}
Any link of any simplex of dimension $m-2$
has diameter at most~$\pi$.
\end{subthm}
\end{thm}

The condition \ref{SHORT.thm:poly-CBB:connected}
can be reformulated in the following way:

\begin{itemize}
 \item[{\it \ref{SHORT.thm:poly-CBB:connected}$'$}] 
Any path $\gamma$ in $\spc{P}$ is a limit of paths
that cross only simplexes of dimension $m$ and $m-1$.
\end{itemize}

Further, modulo the other conditions,
the condition \ref{SHORT.thm:poly-CBB:2pi}
is equivalent to the following:


\begin{itemize}
 \item[{\it \ref{SHORT.thm:poly-CBB:2pi}$'$}] 
The link of any simplex of dimension $m-2$ is 
isometric to a circle of length $\le 2\cdot\pi$
or a closed real interval of length $\le \pi$.
\end{itemize}

\section{Flag complexes}


\begin{thm}{Definition}\label{def:flag}
A simplicial complex $\mathcal{S}$ 
is called \index{flag complex}\emph{flag} if whenever $\{v^0,\z\dots,v^k\}$
is a set of distinct vertices of $\mathcal{S}$
that are pairwise joined by edges, then the vertices $v^0,\dots,v^k$
span a $k$-simplex in~$\mathcal{S}$.

If the above condition is satisfied for $k=2$, 
then we say that $\mathcal{S}$ satisfies 
the \index{no-triangle condition}\emph{no-triangle condition}.
\end{thm}

Note that every flag complex is determined by its one-skeleton.
Moreover, for any graph, its \index{clique}\emph{cliques} (that is, complete subgraphs) define a flag complex.
For that reason, flag complexes are also called  \index{clique complex}\emph{clique complexes}.

\begin{thm}{Exercise}\label{ex:baricenric-flag}
Show that the barycentric subdivision of any simplicial complex is a flag complex.

Use the flag condition (see \ref{thm:flag} below)
to conclude that any finite simplicial complex is homeomorphic to a proper length $\CAT(1)$ space.

\end{thm}


\begin{thm}{Proposition}\label{prop:no-trig}
A simplicial complex $\mathcal{S}$ is flag if and only if 
$\mathcal{S}$ as well as all the links of all its simplices
satisfy the no-triangle condition.
\end{thm}

From the definition of flag complex, 
we get the following.

\begin{thm}{Observation}\label{obs:link-of-flag}
Any link of any simplex in a flag complex is flag.
\end{thm}


\parit{Proof of \ref{prop:no-trig}.}
By Observation~\ref{obs:link-of-flag}, the no-triangle condition holds 
for any flag complex and the  links of all its simplices.

Now assume that a complex $\spc{S}$ and all its links satisfy 
the no-triangle condition.
It follows that $\spc{S}$ includes a 2-simplex for each triangle.
Applying the same observation for each edge we get that $\spc{S}$ 
includes a 3-simplex for any complete graph with 4 vertices.
Repeating this observation 
for triangles, 
4-simplices,
5-simplices,
and so on, we get that $\spc{S}$ is flag.
\qeds


\parbf{All-right triangulation.}
A triangulation of a spherical polyhedral space 
is called an  \index{all-right triangulation}\emph{all-right triangulation} 
if each simplex of the triangulation is isometric 
to a spherical simplex all of whose angles are right.
Similarly, we say that a simplicial complex 
is equipped with an  \index{all-right spherical metric}\emph{all-right spherical metric}
if it is a length metric and each simplex is isometric 
to a spherical simplex all of whose angles are right.

Spherical polyhedral $\CAT(1)$ spaces glued from right-angled simplices
admit the following characterization 
discovered by Mikhael Gromov \cite[p.~122]{gromov-1987}.

\begin{thm}{Flag condition}\label{thm:flag}
Assume that a spherical polyhedral space $\spc{P}$
admits an all-right triangulation~$\tau$.
Then $\spc{P}$ is $\CAT(1)$
if and only if $\tau$ is flag.
\end{thm}

\parit{Proof; only-if part.} 
Assume there are three vertices $v^1$, $v^2$, and $v^3$ of $\tau$
that are pairwise joined by edges 
but do not span a triangle.
Note that in this case 
\[
\mangle\hinge{v^1}{v^2}{v^3}=
\mangle\hinge{v^2}{v^3}{v^1}=
\mangle\hinge{v^3}{v^1}{v^2}=
\pi.
\]
Equivalently,
\begin{clm}{}\label{clm:3pi/2}
The product
of the geodesics $[v^1v^2]$, $[v^2v^3]$, and $[v^3v^1]$
forms a locally geodesic loop in~$\spc{P}$ of length $\tfrac32\cdot\pi$.
\end{clm}

Now assume that $\spc{P}$ is $\CAT(1)$.
Then by \ref{prop:susp},
$\Link_\sigma\spc{P}$ is $\CAT(1)$ for every simplex $\sigma$ 
in~$\tau$. 

Each of these links is an all-right spherical complex and by \ref{cor:closed-geod-cat}, none of these links can contain a geodesic circle shorter than $2\cdot\pi$. 

Therefore Proposition~\ref{prop:no-trig} and \ref{clm:3pi/2} 
imply the ``only if'' part.

\parit{If part.} 
By \ref{obs:link-of-flag} and \ref{cor:closed-geod-cat},
it is sufficient to show that any closed local geodesic $\gamma$ 
in a flag complex $\spc{S}$ with all-right metric has length at least $2\cdot\pi$.

Recall that the  \index{star of vertex}\emph{closed star} of a vertex $v$ (briefly $\overline \Star_v$)
is formed by all the simplices containing~$v$. 
Similarly, $\Star_v$, the open star of $v$, is the union of all simplices containing $v$ with faces opposite $v$ removed.

Choose a vertex $v$ such that $\Star_v$ contains a point $\gamma(t_0)$ of $\gamma$.
Consider the maximal arc $\gamma_v$ of $\gamma$ 
that contains the point $\gamma(t_0)$
and runs in $\Star_v$.
Note that the distance $\dist{v}{\gamma_v(t)}{\spc{P}}$ behaves in exactly the same way 
as the distance from the north pole in~$\mathbb{S}^2$ to a geodesic in the northern hemisphere;
that is, there is a geodesic $\tilde\gamma_v$ in the northern hemisphere of $\mathbb{S}^2$ such that for any $t$ we have
\[\dist{v}{\gamma_v(t)}{\spc{P}}
=
\dist{n}{\tilde\gamma_v(t)}{\mathbb{S}^2},\]
where $n$ denotes the north pole of~$\mathbb{S}^2$.
In particular, 
\[\length\gamma_v=\pi;\]
that is, $\gamma$ spends time $\pi$ on every visit to $\Star_v$.

\begin{wrapfigure}{r}{45mm}
\vskip-2mm
\centering
\includegraphics{mppics/pic-1000}
\end{wrapfigure}

After leaving $\Star_v$,
the local geodesic $\gamma$ has to enter another simplex, 
say~$\sigma'$.
Since $\tau$ is flag, the simplex $\sigma'$
has a vertex $v'$ not joined to $v$ by an edge;
that is, 
\[\Star_v\cap\Star_{v'}=\emptyset\]

The same argument as above shows that $\gamma$ spends time $\pi$ on every visit to $\Star_{v'}$.
Therefore the total length of $\gamma$ is at least~$2\cdot\pi$.
\qeds

\begin{thm}{Exercise}\label{ex:flag>=pi/2}
Assume that a spherical polyhedral space $\spc{P}$
admits a triangulation $\tau$ such that all edge lengths of all simplices are at least~$\tfrac\pi2$.
Show that $\spc{P}$ is $\CAT(1)$
if $\tau$ is flag.
\end{thm}

\begin{thm}{Exercise}\label{ex:polyhedron-glue}
Let $P$ be a convex polyhedron in $\EE^3$ with $n$ faces $F_1,\dots, F_n$.
Suppose that each face of $P$ has only obtuse or right angles.
Let us take $2^n$ copies of $P$ indexed by an $n$-bit array.
Glue two copies of $P$ along $F_i$ if their arrays differ only in the $i$-th bit.
Show that the obtained space is a locally $\CAT(0)$ topological manifold.
\end{thm}

\parbf{The space of trees.}
The following construction is given by
Louis Billera,
Susan Holmes,
and  Karen Vogtmann \cite{billera-holmes-vogtmann}.

Let $\spc{T}_n$ be the set of all metric trees with 
$n$ end vertices
labeled by $a^1,\dots,a^n$.
To describe one tree in $\spc{T}_n$ we may fix a topological tree $t$ with end vertices $a^1,\dots$, $a^n$,
and all other vertices of degree 3,  
and prescribe the lengths of $2\cdot n-3$ edges.
If the length of an edge vanishes, we assume that this edge degenerates;
such a tree can be also described using a different topological tree~$t'$.
The subset of $\spc{T}_n$ corresponding to the given topological tree $t$ can be identified with the octant
\[\set{(x_1,\dots,x_{2\cdot n-3})\in\mathbb{R}^{2\cdot n-3}}{x_i\ge 0}.\]
Equip each such subset with the metric induced from $\mathbb{R}^{2\cdot n-3}$ and consider the length metric on $\spc{T}_n$ induced by these metrics.

\begin{thm}{Exercise}\label{ex:tree}
Show that $\spc{T}_n$ with the described metric is $\CAT(0)$.
\end{thm}

\section{Cubical complexes}

The definition of a cubical complex
mostly repeats the definition of a simplicial complex, 
with simplices replaced by cubes.

Formally, a \index{cubical complex}\emph{cubical complex} is defined as a subcomplex 
of the unit cube in the Euclidean space $\RR^N$ of large dimension;
that is, a collection of faces of the cube
such that together with each face it contains all its sub-faces.
Each cube face in this collection 
will be called a \index{cube}\emph{cube} of the cubical complex.

Note that according to this definition, 
any cubical complex is finite.

The union of all the cubes in a cubical complex $\spc{Q}$ will be called its \index{underlying space}\emph{underlying space}.
A homeomorphism from the underlying space of $\spc{Q}$ to a topological space $\spc{X}$ is called a \index{cubulation}\emph{cubulation of}~$\spc{X}$.

The underlying space of a cubical complex $\spc{Q}$ will be always considered with the length metric
induced from~$\RR^N$.
In particular, with this metric, 
each cube of $\spc{Q}$ is isometric to the unit cube of the corresponding dimension.

It is straightforward to construct a triangulation 
of the underlying space of $\spc{Q}$ 
such that each simplex is isometric to a Euclidean simplex.
In particular, the underlying space of $\spc{Q}$ is a Euclidean polyhedral space.

The link of a cube in a cubical complex is defined similarly to the link of a simplex in a simplicial complex.
It is a simplicial complex that admits a natural all-right triangulation --- each simplex corresponds to an adjusted cube.

\parbf{Cubical analog of a simplicial complex.}
Let $\spc{S}$ be a finite simplicial complex and $\{v_1,\dots,v_N\}$ be the set of its vertices.

Consider $\RR^N$ with the standard basis $\{e_1,\dots,e_N\}$.
Denote by $\square^N$ the standard unit cube in $\RR^N$;
that is, 
\[\square^N=\set{(x_1,\dots,x_N)\in \RR^N}{0\le x_i\le 1\ \text{for each}\ i}.\]

Given a $k$-dimensional simplex $\<v_{i_0},\dots,v_{i_k}\>$ in $\spc{S}$, 
mark the $(k\z+1)$-dimensional faces in $\square^N$ (there are  $2^{N-k}$ of them)
which are parallel to the coordinate $(k+1)$-plane 
spanned by $e_{i_0},\dots,e_{i_k}$.


Note that the set of all marked faces of $\square^{N}$
forms a cubical complex;
it will be called 
the \index{cubical analog}\emph{cubical analog} of $\spc{S}$
and will be denoted as $\square_\spc{S}$.

\begin{thm}{Proposition}\label{prob:cubical-analog}
Let $\spc{S}$ be a finite connected simplicial complex
and $\spc{Q}=\square_{\spc{S}}$ be its cubical analog.
Then the underlying space of $\spc{Q}$ is connected 
and the link of any vertex of $\spc{Q}$
is isometric to  ${\spc{S}}$
equipped with the spherical right-angled metric.

In particular, if $\spc{S}$ is a flag complex,
then $\spc{Q}$ is a locally $\CAT(0)$,
and therefore its universal cover $\tilde{\spc{Q}}$ is $\CAT(0)$.
\end{thm}

\parit{Proof.}
The first part of the proposition follows 
from the construction of $\square_{\spc{S}}$.

If ${\spc{S}}$ is flag, 
then by the flag condition (\ref{thm:flag}) 
the link of any cube in $\spc{Q}$ is $\CAT(1)$.
Therefore, by the cone construction (\ref{ex:cone+susp})
$\spc{Q}$
is locally $\CAT(0)$.
It remains to apply the globalization theorem 
(\ref{thm:hadamard-cartan}).
\qeds

From Proposition \ref{prob:cubical-analog}, 
it follows that the cubical analog
of any flag complex is aspherical.
The following exercise states that the  converse also holds; see \cite[5.4]{davis-2001}.

\begin{thm}{Exercise}\label{ex:flag-aspherical}
Show that a finite simplicial complex is flag 
if and only if its cubical analog is aspherical.
\end{thm}

\section{Construction}

By \ref{ex:contractible}, any complete length $\CAT(0)$ space is contractible.
Therefore, by  the globalization theorem (\ref{thm:hadamard-cartan}), all proper length, locally $\CAT(0)$ spaces 
are \index{aspherical}\emph{aspherical};
that is, they have contractible universal covers.
This observation will be used to construct examples of  aspherical spaces. 

Let $\spc X$ be a proper topological space.
Recall that $\spc X$ is called 
\index{simply connected space at infinity}\emph{simply connected at infinity} 
if for any compact set $K\subset\spc X$
there is a bigger compact set $K'\supset K$
such that  $\spc X\setminus K'$ is path-connected 
and any loop which lies in $\spc X\setminus K'$
is null-homotopic in  $\spc X\setminus K$.

Recall that path-connected spaces are not empty by definition.
Therefore compact spaces are not simply connected at infinity.

The following example was constructed by Michael Davis \cite{davis-1983}.

\begin{thm}{Proposition}\label{prop:aspherical}
For any $m\ge 4$, there is a closed aspherical $m$-dimensional manifold whose universal cover is not simply connected at infinity.

In particular, the universal cover of this manifold 
is not homeomorphic to the $m$-dimensional Euclidean space.
\end{thm}

The proof requires the following lemma.

\begin{thm}{Lemma}\label{lem:example-pi_infty}
Let $\spc{S}$ be a finite flag complex,
$\spc{Q}=\square_{\spc{S}}$ be its cubical analog
and $\tilde{\spc{Q}}$ be the universal cover of~$\spc{Q}$.

Assume  $\tilde{\spc{Q}}$ is simply connected at infinity.
Then $\spc{S}$ is simply connected.
\end{thm}

\parit{Proof.}
Assume $\spc{S}$ is not simply connected. Equip $\spc{S}$ with an all-right spherical metric.
Choose a shortest noncontractible circle $\gamma\:\mathbb{S}^1\to\spc{S}$ formed by the edges of~$\spc{S}$.

Note that $\gamma$ forms a one-dimensional subcomplex of $\spc{S}$ which is a closed local geodesic.
Denote by $G$ the subcomplex of $\spc{Q}$ which corresponds to~$\gamma$.

Fix a vertex $v\in G$;
let $G_v$ be the connected component of $v$ in~$G$.
Let $\tilde G$ be a connected component of the inverse image of $G_v$ in $\tilde{\spc{Q}}$
for the universal cover $\tilde{\spc{Q}}\to \spc{Q}$.
Fix a point $\tilde v\in\tilde G$ in the inverse image of~$v$.

\begin{wrapfigure}{r}{25mm}
\vskip0mm
\centering
\includegraphics{mppics/pic-1100}
\end{wrapfigure}
 
Note that 
\begin{clm}{}\label{tilde-G-convex}
$\tilde G$ is a convex set in~$\tilde{\spc{Q}}$.
\end{clm}


Indeed, according to Proposition \ref{prob:cubical-analog},
$\tilde{\spc{Q}}$ is $\CAT(0)$.
By Exercise \ref{ex:locally-convex},
it is sufficient to show that $\tilde G$ is locally convex in $\tilde{\spc{Q}}$,
or equivalently, $G$ is locally convex in~$\spc{Q}$.

Note that the latter can only fail if $\gamma$ contains two vertices, say $\xi$ 
and 
$\zeta$ in $\spc{S}$,
which are joined by an edge not in $\gamma$; 
denote this edge by~$e$.

Each edge of $\spc{S}$ has length~$\tfrac\pi2$.
Therefore each of the two circles formed by $e$ and an arc of $\gamma$
from $\xi$ to $\zeta$ is shorter than~$\gamma$.
Moreover,
at least one of them is noncontractible 
since $\gamma$ is 
noncontractible.
That is, 
$\gamma$ is not a shortest noncontractible circle, a contradiction.
\claimqeds

Further, note that 
$\tilde G$ is homeomorphic to the plane
since $\tilde G$ is 
a two-dimensional manifold without boundary which 
by the above is $\CAT(0)$ and hence is contractible.

Denote by $C_R$ the circle of radius $R$ in $\tilde G$ centered at~$\tilde v$.
All $C_R$ are homotopic to each other in $\tilde G\setminus\{\tilde v\}$ and therefore in $\tilde{\spc{Q}}\setminus \{\tilde v\}$.

Note that the map $\tilde{\spc{Q}}\setminus \{\tilde v\}\to \spc{S}$
which returns the direction of $[{\tilde v}{x}]$  for any $x\ne \tilde v$, maps $C_R$ to a circle homotopic to~$\gamma$.
Therefore $C_R$ is not contractible in $\tilde{\spc{Q}}\setminus \{\tilde v\}$.

If $R$ is large, 
the circle $C_R$  
lies outside of any fixed compact set $K'$ in~$\tilde{\spc{Q}}$.
From above $C_R$ is not contractible in $\tilde{\spc{Q}}\setminus K$
if $K\supset \tilde v$.
It follows that $\tilde{\spc{Q}}$ is not simply connected at infinity, a contradiction.
\qeds

The proof of the following exercise is analogous.
It will be used later in the proof of Proposition~\ref{prop:loc-CAT-mnfld} --- a more geometric version of Proposition~\ref{prop:aspherical}.

\begin{thm}{Exercise}\label{ex:example-pi_infty-new}
Under the assumptions of Lemma~\ref{lem:example-pi_infty}, 
for any vertex $v$ in $\spc{S}$
the complement $\spc{S}\setminus\{v\}$ is simply connected.
\end{thm}

\parit{Proof of \ref{prop:aspherical}.}
Let $\Sigma^{m-1}$ be an $(m-1)$-dimensional smooth homology sphere that  is not simply connected, and bounds a contractible smooth compact $m$-dimensional manifold~$\spc{W}$. 

For $m\ge 5$, the existence of such $(\spc{W}, \Sigma)$
is proved by Michel Kervaire \cite{kervaire}. 
For $m=4$, it follows from the construction of Barry Mazur~\cite{mazur}.

Pick any triangulation $\tau$ of $W$ and let $\spc{S}$ be the resulting subcomplex that triangulates~$\Sigma$.


We can assume that $\spc{S}$ is flag; 
otherwise, pass to the barycentric subdivision 
of $\tau$ and apply Exercise~\ref{ex:baricenric-flag}.


Let $\spc{Q}=\square_{\spc{S}}$ be the cubical analog of~$\spc{S}$.

By Proposition~\ref{prob:cubical-analog},
$\spc{Q}$ is a homology manifold.
It follows that $\spc{Q}$ is a piecewise linear manifold 
with a finite number of singularities at its vertices.


Removing a small contractible neighborhood $V_v$ of each vertex $v$ in $\spc{Q}$,
we can obtain a piecewise linear manifold $\spc{N}$
whose boundary is formed by several copies of~$\Sigma$.

Let us glue a copy of  $\spc{W}$ along its boundary to each copy of $\Sigma$ in the boundary of~$\spc{N}$.
This results in a  closed manifold 
$\spc{M}$ with polyhedral metric which is homotopically equivalent to~$\spc{Q}$.

Indeed, since both $V_v$ and $\spc{W}$ are contractible, the identity map of  their common boundary $\Sigma$ can be extended to a homotopy equivalence $V_v\to\spc{W}$ relative to the boundary.
Therefore the identity map on $\spc{N}$ extends to homotopy equivalences 
$f\: \spc Q\to \spc M$ and $g\:\spc M\to \spc Q$.

Finally, by Lemma~\ref{lem:example-pi_infty},  
the universal cover $\tilde{\spc{Q}}$ of $\spc{Q}$
is not simply connected at infinity.

The same holds for 
the universal cover $\tilde{\spc{M}}$ of $\spc{M}$.
The latter follows since the constructed homotopy equivalences 
$f\: \spc Q\to \spc M$ and $g\:\spc M\z\to \spc Q$ 
lift to {}\emph{proper maps} 
$\tilde f \: \tilde{\spc{Q}}\to \tilde{\spc{M}}$
and $\tilde g \: \tilde{\spc{M}}\to \tilde{\spc{Q}}$;
that is, for any compact sets $A\subset \tilde{\spc{Q}}$ and $B\subset\tilde{\spc{M}}$, the inverse images $\tilde g^{-1}(A)$ and $\tilde f^{-1}(B)$ are compact.
\qeds


The following proposition was proved by
Fredric Ancel, 
Michael Davis,
and Craig Guilbault \cite{ancel-davis-guilbault};
it could be considered as a more geometric version of Proposition~\ref{prop:aspherical}.


\begin{thm}{Proposition}\label{prop:loc-CAT-mnfld}
Given $m\ge 5$, there is a Euclidean polyhedral space $\spc{P}$ such that:
\begin{subthm}{}
$\spc{P}$ is homeomorphic to a closed $m$-dimensional manifold.
\end{subthm}

\begin{subthm}{}
$\spc{P}$ is locally $\CAT(0)$.
\end{subthm}

\begin{subthm}{}
The universal cover of $\spc{P}$ is not simply connected at infinity.
\end{subthm}
\end{thm}

Dale Rolfsen \cite{rolfsen} has shown that there are no three-dimensional examples of that type.
Paul Thurston \cite{thurston} conjectured that the same holds in the four-dimensional case.

\parit{Proof.}
Apply Exercise~\ref{ex:example-pi_infty-new} to the barycentric subdivision of the simplicial complex $\spc{S}$ provided by Exercise~\ref{ex:funny-S}.
\qeds

\begin{thm}{Exercise}\label{ex:funny-S}
Given an integer $m\ge 5$,
construct a finite $(m-1)$-dimensional simplicial complex $\spc{S}$ such that $\Cone\spc{S}$ is homeomorphic to $\EE^m$
and $\pi_1(\spc{S}\setminus\{v\})\ne0$ for some vertex $v$ in~$\spc{S}$.
\end{thm} 

\section{Remarks}

There is a good-looking description of polyhedral $\CAT(\kappa)$ and $\CBB(\kappa)$ spaces \cite[12.2 and 12.5]{alexander-kapovitch-petrunin-2025},
but in fact, it is hard to check even in very simple cases.
For example, the description of those coverings of $\mathbb{S}^3$ branching at three 
great circles which are $\CAT(1)$ requires quite a bit of work \cite{charney-davis-1993} --- try to guess the answer before reading.

Another example is the braid space $\spc{B}_n$ that  is the universal cover of $\CC^n$ infinitely branching in complex hyperplanes $z_i=z_j$ with the induced length metric.
So far it is not known if $\spc{B}_n$ is $\CAT(0)$ for any $n\ge 4$ \cite{panov-petrunin-2016}.
Understanding this space could help to study the braid group.
This circle of questions is closely related to the generalization of the flag condition (\ref{thm:flag}) to  spherical simplices with few acute dihedral angles.


The construction used in the proof of  Proposition~\ref{prop:aspherical} admits a number of modifications,  
several of which are discussed in a survey by Michael Davis \cite{davis-2001}.

A similar argument was used by Michael Davis, 
Tadeusz Ja\-nu\-szkie\-wicz,
and  Jean-Fran\c{c}ois Lafont \cite{davis-januszkiewicz-lafont}.
They constructed a closed smooth four-dimensional manifold $M$ with  universal cover $\tilde M$ diffeomorphic to $\RR^4$, such that $M$ admits a polyhedral metric which is locally $\CAT(0)$, but does not admit a Riemannian metric with nonpositive sectional curvature.
Another example of that type was constructed by Stephan Stadler \cite{stadler}.
There are no lower-dimensional examples of this type ---
the two-dimensional case follows from the  classification of surfaces,
and 
the three-dimensional case follows from the geometrization conjecture.

It is noteworthy that any complete, simply connected Riemannian manifold with nonpositive curvature is homeomorphic to the Euclidean space of the same dimension.
In fact, by the globalization theorem
(\ref{thm:hadamard-cartan}), 
the exponential map at a point of such a manifold is a homeomorphism.
In particular, there is no Riemannian analog of Proposition~\ref{prop:loc-CAT-mnfld}.

Recall that a triangulation of an $m$-dimensional manifold defines a piecewise linear structure if the link  of every simplex $\Delta$ is homeomorphic to the sphere of dimension $m-1-\dim\Delta$.
According to Stone's theorem \cite{stone, davis-januszkiewicz}, the triangulation of $\spc{P}$ in Proposition~\ref{prop:loc-CAT-mnfld} 
cannot be made piecewise linear --- despite  the fact that $\spc{P}$ is a manifold, its triangulation does not induce a piecewise linear structure.

The flag condition also leads to the so-called {}\emph{hyperbolization} procedure, a flexible tool for constructing  aspherical spaces;
a good survey on the subject is given by Ruth Charney and Michael Davis \cite{charney-davis-1995}.

The $\CAT(0)$ property of a cube complex admits interesting (and useful) geometric descriptions if one exchanged the $\ell^2$-metric to a natural $\ell^1$ or $\ell^\infty$ on each cube.

\begin{thm}{Theorem}
The following three conditions are equivalent.

\begin{subthm}{cube-2} A cube complex $Q$ equipped with $\ell^2$-metric is $\CAT(0)$.
\end{subthm}

\begin{subthm}{cube-infty} A cube complex $Q$ equipped with $\ell^\infty$-metric is injective.
\end{subthm}

\begin{subthm}{cube-1} A cube complex $Q$ equipped with $\ell^1$-metric is \index{median space}\emph{median}.
 The latter means that for any three points $x,y,z$ there is a {}\emph{unique} point $m$ (it is called the \index{median}\emph{median} of $x$, $y$, and $z$) and a choice of geodesics such that $[xy]\ni m$, $[xz]\ni m$ and $[yz]\ni m$.
\end{subthm}
\end{thm}

A very readable paper on the subject was written by Brian Bowditch \cite{bowditch-2020}; two easy parts of the theorem are included in the following exercise.

\begin{thm}{Exercise}\label{ex:cube-infty=>cube-2} Prove the implication \ref{SHORT.cube-infty}$\Rightarrow$\ref{SHORT.cube-2} and/or \ref{SHORT.cube-1}$\Rightarrow$\ref{SHORT.cube-2} in the theorem.
\end{thm}


All the topics discussed in this lecture link Alexandrov geometry with the fundamental group.
The theory of {}\emph{hyperbolic groups}, 
a branch of {}\emph{geometric group theory}, 
introduced by 
Mikhael Gromov \cite{gromov-1987},
could be considered as a further step in this direction.


