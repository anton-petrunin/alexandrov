\chapter{Definitions}

The first synthetic description of curvature is due to Abraham Wald \cite{wald} published in 1936;
it was his student work, written under the supervision of Karl Menger. 
This publication was not noticed for about 50 years \cite{berestovskii}.
In 1941, similar definitions were rediscovered by Alexandr Alexandrov \cite{alexandrov:def}.

\section{Notations}

The distance between two points $x$ and $y$ in a metric space $\spc{X}$ will be denoted by $\dist{x}{y}{}$ or $\dist{x}{y}{\spc{X}}$.
The latter notation is used if we need to emphasize 
that the distance is taken in the space~${\spc{X}}$.

We will denote by $\SSS^n$, $\EE^n$, and $\HH^n$ the $n$-dimensional sphere (with angle metric), 
Euclidean space, and Lobachevsky space respectively.
More generally, $\MM^n(\kappa)$ will denote the \emph{model $n$-space} of curvature $\kappa$;
that is,
\begin{itemize}
\item if $\kappa>0$, then $\MM^n(\kappa)$ is the $n$-sphere of radius $\tfrac{1}{\sqrt{\kappa}}$, so $\SSS^n=\MM^n(1)$
\item $\MM^n(0)=\EE^n$,
\item if $\kappa<0$, then $\MM^n(\kappa)$ is the Lobachevsky $n$-space $\HH^n$ rescaled by factor $\tfrac{1}{\sqrt{-\kappa}}$;
in particular $\MM^n(-1)=\HH^n$.
\end{itemize}

\section{Wald's approach}

Wald noticed that a \textit{typical} quadruple $x_1,x_2,x_3,x_4$ of points in a metric space
admits model configurations in $\tilde x_1,\tilde x_2,\tilde x_3,\tilde x_4\in\MM^3(\kappa)$ with
\[\dist{\tilde x_i}{\tilde x_j}{\MM^3(\kappa)}=\dist{x_i}{x_j}{\spc{X}}\]
for $\kappa$ in a closed interval,
say 
\[[\kappa_{\min}(x_1,x_2,x_3,x_4),\kappa_{\max}(x_1,x_2,x_3,x_4)]\subset \RR.\]

\begin{wrapfigure}{r}{33mm}
\vskip-2mm
\centering
\includegraphics{mppics/pic-710}
\end{wrapfigure}

In $\MM^3(\kappa_{\min})$ and $\MM^3(\kappa_{\max})$, the points $\tilde x_1,\tilde x_2,\tilde x_3,\tilde x_4$ form degenerate tetrahedrons shown on the diagram (for $\kappa_{\min}$ it is a convex quadrangle and for $\kappa_{\max}$ --- a triangle with a point inside).
In the interior of the interval, the tetrahedron is nondegenerate.

Moreover, one can use $[-\infty,\infty)$ instead of $\RR$ 
and let
\[\kappa_{\min}(x_1,x_2,x_3,x_4)=-\infty\]
if there is \textit{almost} model quadruple in
$\MM^3(\kappa)$ for $\kappa\to -\infty$;
that is, for any $\eps>0$ there is a quadruple
$\tilde x_1,\tilde x_2,\tilde x_3,\tilde x_4\in\MM^3(\kappa)$
such that $\kappa\le -\tfrac1\eps$, and
\[\dist{\tilde x_i}{\tilde x_j}{\MM^3(\kappa)}\lege\dist{x_i}{x_j}{\spc{X}}\pm\eps\]
for all $i$ and $j$.
In this case the interval 
\[[\kappa_{\min}(x_1,x_2,x_3,x_4),\kappa_{\max}(x_1,x_2,x_3,x_4)]\subset [-\infty,\infty)\]
is defined for \textit{any} quadruple.

We will not use these statements further in the sequel, so we omit the proofs.
The just wanted to describe the first step in the theory.

\begin{thm}{Exercise}
Let $x_1,x_2,x_3,x_4$ be a quadruple in a metric space such that $\kappa_{\min}(x_1,x_2,x_3,x_4)=-\infty$.
Show that two maximal numbers from the following three are equal to each other.
\begin{align*}
a&=\dist{x_1}{x_2}{}+\dist{x_3}{x_4}{},
\\
b&=\dist{x_1}{x_3}{}+\dist{x_2}{x_4}{},
\\
c&=\dist{x_1}{x_4}{}+\dist{x_2}{x_3}{}.
\end{align*}


\end{thm}


\begin{thm}{Exercise}
Suppose that $x_1,x_2,x_3,x_4$ in a metric space
such that
\begin{align*}
\dist{x_1}{x_2}{}=\dist{x_1}{x_3}{}=\dist{x_1}{x_4}{}&=1,
\\
\dist{x_2}{x_3}{}=\dist{x_3}{x_4}{}=\dist{x_4}{x_1}{}&=2.
\end{align*}
Show that 
\[\kappa_{\min}(x_1,x_2,x_3,x_4)=\kappa_{\max}(x_1,x_2,x_3,x_4)=-\infty.\]
\end{thm}

\begin{thm}{Exercise}
Let $x_1,x_2,x_3,x_4$ be a quadruple in $\EE^2$.
Suppose that triangle $[x_1x_2x_3]$ is degerate, but  $[x_2x_3x_4]$ is not.
Show that 
\[\kappa_{\min}(x_1,x_2,x_3,x_4)=\kappa_{\max}(x_1,x_2,x_3,x_4)=0.\]
\end{thm}

\begin{thm}{Wald-style definition}
Let $\kappa\in \RR$.
A metric space $\spc{X}$ has curvature $\ge\kappa$ (or $\le\kappa$) 
if for any quadruple $x_1,x_2,x_3,x_4\in \spc{X}$ we have 
$\kappa_{\max}(x_1,x_2,x_3,x_4)\ge \kappa$ (or $\kappa_{\min}(x_1,x_2,x_3,x_4)\le \kappa$ respectively). 
\end{thm}

\section{Substance}\label{sec:manifesto}

Consider the space $\mathcal{M}_4$ of all isometry classes of 4-point metric spaces.
Each element in $\mathcal{M}_4$ can be described by 6 numbers 
 --- the distances between all 6 pairs of its points, say $\ell_{i,j}$ for $1\le i< j\le 4$ modulo permutations of the index set $(1,2,3,4)$.
These 6 numbers are subject to 12 triangle inequalities; that is,
\[\ell_{i,j}+\ell_{j,k}\ge \ell_{i,k}\]
holds for all $i$, $j$ and $k$, where we assume that $\ell_{j,i}=\ell_{i,j}$, and $\ell_{i,i}=0$.

{

\begin{wrapfigure}{o}{33mm}
\vskip-3mm
\centering
\includegraphics{mppics/pic-700}
\end{wrapfigure}

The space $\mathcal{M}_4$ comes with topology.
It can be defined as a quotient topology of the cone in $\RR^6$ by permutations of the 4 points of the space.

Consider the subset $\mathcal{E}_4\subset \mathcal{M}_4$ of all isometry classes of 4-point metric spaces that admit isometric embeddings into Euclidean space.

}

\begin{thm}{Claim}\label{clm:two-components-of-M4}
The complement $\mathcal{M}_4\setminus \mathcal{E}_4$ has two connected components.
\end{thm}

\begin{thm}{Exercise}
Spend 10 minutes trying to prove the claim.
\end{thm}


The definition of Alexandrov spaces is based on the claim above.
Let us denote one of the components by $\mathcal{P}_4$ and the other by~$\mathcal{N}_4$.
Here $\mathcal{P}$ and $\mathcal{N}$ stand for {}\emph{positive} 
and {}\emph{negative curvature} because spheres have no quadruples of type $\mathcal{N}_4$ and 
hyperbolic space
has no quadruples of type~$\mathcal{P}_4$.

A metric space that has no quadruples of points of type $\mathcal{P}_4$ or $\mathcal{N}_4$
respectively 
is called an Alexandrov space with non-positive ($\CAT(0)$) or non-negative curvature ($\CBB(0)$).

\begin{wrapfigure}{r}{33mm}
\vskip-0mm
\centering
\includegraphics{mppics/pic-710}
\end{wrapfigure}

Let us describe the subdivision into  $\mathcal{P}_4$, $\mathcal{E}_4$, and $\mathcal{N}_4$ intuitively.
Imagine that you move out of $\mathcal{E}_4$ --- your path is a one-parameter family of 4-point metric spaces.
The last thing you see in $\mathcal{E}_4$ is one of the two plane configurations shown on the diagram.
If you see the right configuration then you move into $\mathcal{N}_4$;
if it is the one on the left, then you move into $\mathcal{P}_4$.
More degenerate pictures can be avoided; for example, a triangle with a point on a side.
From such a configuration one may move in $\mathcal{N}_4$ and $\mathcal{P}_4$ (as well as come back to $\mathcal{E}_4$).

Here is an exercise, solving which would force you to rebuild a considerable part of Alexandrov geometry.
It is wise to spend some time thinking about this exercise before proceeding.

\begin{thm}{Advanced exercise}\label{ex:convex-set}
Assume $\spc{X}$ is a complete metric space with length metric (see Section~\ref{sec:length}), 
containing only quadruples of type~$\mathcal{E}_4$.
Show that $\spc{X}$ is isometric to a convex set in a Hilbert space.
\end{thm}

In the definition above, 
one can take $\MM^3(\kappa)$
instead of $\EE^3$.
In this case,
one obtains the definition of spaces with curvature bounded above or below by~$\kappa$ ($\CAT(\kappa)$ or $\CBB(\kappa)$).
The parameter $\kappa$ has three interesting choices $-1$, $0$, and $1$;
the rest can be obtained from these three applying rescaling.

\section{Geodesics, triangles, and angles}

\parbf{Geodesics.}
Let $\spc{X}$ be a metric space 
and $\II$\index{$\II$} a real interval. 
A distance-preserving map $\gamma\:\II\to \spc{X}$ is called a \index{geodesic}\emph{geodesic}%
\footnote{Others call it differently: \textit{shortest path}, \textit{minimizing geodesic}.
Also, note that the meaning of the term \textit{geodesic} is different from what is used in Riemannian geometry, altho they are closely related.}; 
in other words, $\gamma\:\II\to \spc{X}$ is a geodesic if 
\[\dist{\gamma(s)}{\gamma(t)}{\spc{X}}=|s-t|\]
for any pair $s,t\in \II$.

If $\gamma\:[a,b]\to \spc{X}$ is a geodesic such that $p=\gamma(a)$, $q=\gamma(b)$, then we say that $\gamma$ is a geodesic from $p$ to $q$.
In this case, the image of $\gamma$ is denoted by $[p q]$\index{$[{*}{*}]$}, and, with abuse of notations, we also call it a \index{geodesic}\emph{geodesic}.
We may write $[p q]_{\spc{X}}$ 
to emphasize that the geodesic $[p q]$ is in the space  ${\spc{X}}$.

In general, a geodesic from $p$ to $q$ need not exist and if it exists, it need not  be unique.  
However, once we write $[p q]$ we assume that we have chosen such geodesic.

A \index{geodesic path}\emph{geodesic path} is a geodesic with constant-speed parameterization by the unit interval $[0,1]$.

A metric space is called \index{geodesic space}\emph{geodesic} if any pair of its points can be joined by a geodesic.

\parbf{Triangles.}
Given a triple of points $p,q,r$ in a metric space $\spc{X}$, a choice of geodesics $([q r], [r p], [p q])$ will be called a \index{triangle}\emph{triangle}; we will use the short notation 
$\trig p q r=\trig p q r_{\spc{X}}=([q r], [r p], [p q])$\index{$\trig {{*}}{{*}}{{*}}$}.

Given a triple $p,q,r\in \spc{X}$ there may be no triangle 
$\trig p q r$ simply because one of the pairs of these points cannot be joined by a geodesic.
Also, many different triangles with these vertices may exist, any of which can be denoted by $\trig p q r$.
If we write $\trig p q r$, it means that we have chosen such a triangle.

\parbf{Model triangles.}
Given three points $p,q,r$ in a metric space $\spc{X}$,
let us define its \index{model triangle}\emph{model triangle} $\trig{\tilde p}{\tilde q}{\tilde r}$ 
(briefly, 
$\trig{\tilde p}{\tilde q}{\tilde r}=\modtrig(p q r)_{\EE^2}$%
\index{$\modtrig$}) to be a triangle in the Euclidean plane $\EE^2$ such that
\begin{align*}\dist{\tilde p}{\tilde q}{\EE^2}&=\dist{p}{q}{\spc{X}},
&
\quad\dist{\tilde q}{\tilde r}{\EE^2}&=\dist{q}{r}{\spc{X}},
&
\quad\dist{\tilde r}{\tilde p}{\EE^2}&=\dist{r}{p}{\spc{X}}.
\end{align*}

The same way we can define the \index{hyperbolic model triangle}\emph{hyperbolic} and the \index{spherical model triangles}\emph{spherical model triangles} $\modtrig(p q r)_{\HH^2}$, $\modtrig(p q r)_{\SSS^2}$
in the Lobachevsky plane $\HH^2$ and the unit sphere~$\SSS^2$.
In the latter case, the model triangle is said to be defined if in addition
\[\dist{p}{q}{}+\dist{q}{r}{}+\dist{r}{p}{}< 2\cdot\pi.\]
In this case, the model triangle again exists and is unique up to an isometry of~$\SSS^2$.

\parbf{Model angles.}
If 
$\trig{\tilde p}{\tilde q}{\tilde r}=\modtrig(p q r)_{\EE^2}$ 
and $\dist{p}{q}{},\dist{p}{r}{}>0$, 
the angle measure of 
$\trig{\tilde p}{\tilde q}{\tilde r}$ at $\tilde p$ 
will be called the \index{model angle}\emph{model angle} of the triple $p$, $q$, $r$ and will be denoted by
$\angk p q r_{\EE^2}$%
\index{$\tilde\measuredangle$!$\angk{{*}}{{*}}{{*}}$}.

The same way we define $\angk p q r_{\MM^2(\kappa)}$;
in particular, $\angk p q r_{\HH^2}$ and $\angk p q r_{\SSS^2}$.
We may use the notation $\angk p q r$ if it is evident which of the model spaces is meant.

\begin{thm}{Exercise}
Show that for any triple of point $p$, $q$, and $r$,
the function
\[\kappa\mapsto \angk p q r_{\MM^2(\kappa)}\]
is nondecreasing in its domain of definition.
\end{thm}


\parbf{Hinges.} Let $p,x,y\in \spc{X}$ be a triple of points such that $p$ is distinct from $x$ and~$y$.
A pair of geodesics $([p x],[p y])$ will be called  a \index{hinge}\emph{hinge} and will be denoted by 
$\hinge p x y=([p x],[p y])$\index{$\hinge{{*}}{{*}}{{*}}$}.

\section{Definitions}

In this section we write inequalities that describe the sets $\mathcal{E}_4\cup\mathcal{P}_4$ and ~$\mathcal{E}_4\cup\mathcal{N}_4$ from Section \ref{sec:manifesto}.

\parbf{Curvature bounded below.}
Let $p,x,y,z$ be a quadruple of points in a metric space.
If the inequality 
\[\angk  pxy_{\EE^2}+\angk pyz_{\EE^2}+\angk pzx_{\EE^2}
\le 
2\cdot\pi
\eqlbl{eq:CBB-comparison}\]
holds, then we say that the quadruple meets \index{$\CBB$}\emph{$\CBB(0)$ comparison}.

\begin{thm}{Definition}\label{def:CBB}
A metric space $\spc{X}$ has {}\emph{nonnegative curvature} in the sense of Alexandrov (briefly, $\spc{X}\in\CBB(0)$ if $\CBB(0)$ comparison
holds for any quadruple in $\spc{X}$ such that each model angle in \ref{eq:CBB-comparison} is defined. 
\end{thm}

If instead of $\EE^2$, we use $\SSS^2$ or $\HH^2$, then we get the definition of
$\CBB(1)$ and $\CBB(-1)$ comparisons.
Note that $\angk  pxy_{\EE^2}$ and $\angk  pxy_{\HH^2}$ are defined if $p\ne x$, $p\ne y$,
but for $\angk  pxy_{\SSS^2}$ we need in addition, $\dist{p}{x}{}+\dist{p}{y}{}+\dist{x}{y}{}<2\cdot\pi$.

More generally, one may apply this definition to $\MM^2(\kappa)$.
This way we define $\CBB(\kappa)$ comparison for any real $\kappa$.

\begin{thm}{Exercise}
Show that $\EE^n$ is $\CBB(0)$.
\end{thm}

\begin{thm}{Exercise}\label{ex:(3+1)-expanding}
Show that a metric space $\spc{X}$ is $\Alex0$
if and only if for any quadruple of points $p,x_1,x_2,x_3\in \spc{X}$ 
there is a quadruple of points $q,y_1,y_2,y_3\in\EE^3$
such that 
\[\dist{p}{x_i}{\spc{X}}\ge\dist{q}{y_i}{\EE^2} 
\quad \text{and}\quad
\dist{x_i}{x_j}{\spc{X}}\le\dist{y_i}{y_j}{\EE^2}\] 
for all $i$ and $j$.
\end{thm}

\begin{thm}{Exercise}\label{ex:normCBB}
Show that $\RR^2$ with metric induced by a norm is $\Alex0$ if and only if it is isometric to the Euclidean plane $\EE^2$.
\end{thm}

\parbf{Curvature bounded above.}
Given a quadruple of points $p,q,x,y$ in a metric space $\spc{X}$,
consider two model triangles 
$\trig{\tilde p}{\tilde x}{\tilde y}=\modtrig{}(pxy)_{\EE^2}$ 
and 
$\trig{\tilde q}{\tilde x}{\tilde y}\z=\modtrig{}(qxy)_{\EE^2}$ with common side $[\tilde x\tilde y]$.

\begin{wrapfigure}{r}{25mm}
\vskip-4mm
\centering
\includegraphics{mppics/pic-720}
\end{wrapfigure}

If the inequality
\[\dist{p}{q}{\spc{X}}\le \dist{\tilde p}{\tilde z}{\EE^2}+\dist{\tilde z}{\tilde q}{\EE^2}\]
holds for any point $\tilde z\in [\tilde x\tilde y]$, then we say that 
the quadruple $p,q,x,y$ satisfies \index{$\CAT$}\emph{$\CAT(0)$ comparison}.
\label{page:CAT-comparison}

\begin{thm}{Definition}\label{def:CBB}
A metric space $\spc{X}$ has \index{$\CAT$}\emph{nonpositive curvature} in the sense of Alexandrov (briefly, $\spc{X}\in\CAT(0)$) if $\CAT(0)$ comparison holds for any quadruple in $\spc{X}$.
\end{thm}

If we do the same for spherical model triangles  
$\trig{\tilde p}{\tilde x}{\tilde y}=\modtrig{}(pxy)_{\SSS^2}$ 
and 
$\trig{\tilde q}{\tilde x}{\tilde y}=\modtrig{}(qxy)_{\SSS^2}$,
then we arrive at the definition of $\CAT(1)$ comparison.
One of the spherical model triangles might undefined;
it happens if 
\[\dist{p}{x}{}+\dist{p}{y}{}+\dist{x}{y}{}\ge 2\cdot\pi
\quad
\text{or}
\quad
\dist{q}{x}{}+\dist{q}{y}{}+\dist{x}{y}{}\ge 2\cdot\pi.\]
In this case, it is assumed that $\CAT(1)$ comparison automatically holds for this quadruple.

We can do the same for $\MM^2(\kappa)$.
In this case, we arrive at the definition of $\CAT(\kappa)$ comparison.
However, we will mostly consider $\CAT(0)$ comparison and occasionally $\CAT(1)$ comparison;
so, if you see $\CAT(\kappa)$, then it is safe to assume that $\kappa$ is $0$ or~$1$.

Here $\CAT$ is an acronym for Cartan, Alexandrov, and Toponogov,
but usually pronounced as ``cat'' in the sense of ``miauw''.
The term was coined by Mikhael Gromov in 1987.
Originally, Alexandrov used \emph{$\mathfrak{R}_\kappa$ domain};
this term is still in use.

\begin{thm}{Exercise}\label{ex:sba-2+2-short}
Show that a metric space $\spc{U}$ is $\CAT(0)$
if and only if for any quadruple of points 
$p,q,x,y$ in $\spc{U}$ 
there is a quadruple $\tilde p,\tilde q,\tilde x,\tilde y$ in $\EE^2$
such that 
\begin{align*}
\dist{\tilde p}{\tilde q}{}&\ge\dist{p}{q}{},
&
\dist{\tilde x}{\tilde y}{}&\ge\dist{x}{y}{},
\\
\dist{\tilde p}{\tilde x}{}&\le \dist{p}{x}{},
&
\dist{\tilde p}{\tilde y}{}&\le \dist{p}{y}{},
\\
\dist{\tilde q}{\tilde x}{}&\le \dist{q}{x}{},
&
\dist{\tilde q}{\tilde y}{}&\le \dist{q}{y}{}.
\end{align*}

\end{thm}

\begin{thm}{Exercise}
Assume that a quadruple of points in a metric space satisfies $\CBB(0)$ and $\CAT(0)$ comparisons for all labelings.
Show that it is isometric to a quadruple in $\EE^3$.
\end{thm}

\begin{thm}{Exercise}\label{ex:normCBB}
Show that $\RR^2$ with metric induced by a norm is $\CAT(0)$ if and only if it is isometric to the Euclidean plane $\EE^2$.
\end{thm}

The definitions stated in the this section can be applied to any metric space.
However, interesting things happen only for the so-called \textit{geodesic} or at least \textit{length spaces}.

\section{Length and length spaces}\label{sec:length}

\parbf{Length.}
A \index{curve}\emph{curve} is defined as a continuous map from a real interval $\II$ to a metric space.
If $\II=[0,1]$, then the curve is called a \index{path}\emph{path}.

\begin{thm}{Definition}
Let $\spc{X}$ be a metric space and
$\alpha\: \II\to \spc{X}$ be a curve.
We define the \index{length}\emph{length} of $\alpha$ as 
\[
\length \alpha \df \sup_{t_0\le t_1\le\ldots\le t_n}\sum_i \dist{\alpha(t_i)}{\alpha(t_{i-1})}{}.
\]

A curve $\alpha$ is called \index{rectifiable curve}\emph{rectifiable} if $\length \alpha<\infty$.
\end{thm}

The following theorem is assumed to be known;
see \cite{petrunin2023pure,burago-burago-ivanov}.


\begin{thm}{Theorem}\label{thm:length-semicont}
Length is a lower semi-continuous with respect to the pointwise convergence of curves. 

More precisely, assume that a sequence
of curves $\gamma_n\:\II\to \spc{X}$ in a metric space $\spc{X}$ converges pointwise 
to a curve $\gamma_\infty\:\II\to \spc{X}$;
that is, for any fixed $t \in \II$ we have $\gamma_n(t)\z\to\gamma_\infty(t)$ as $n\to\infty$. 
Then 
$$\liminf_{n\to\infty} \length\gamma_n \ge \length\gamma_\infty.\eqlbl{eq:semicont-length}$$
\end{thm}

\begin{wrapfigure}{o}{20 mm}
\vskip-0mm
\centering
\includegraphics{mppics/pic-100}
\end{wrapfigure}

Note that the inequality \ref{eq:semicont-length} might be strict.
For example, the diagonal $\gamma_\infty$ of the unit square 
can be  approximated by stairs-like
polygonal curves $\gamma_n$
with sides parallel to the sides of the square ($\gamma_6$ is on the picture).
In this case
\[\length\gamma_\infty=\sqrt{2}\quad
\text{and}\quad \length\gamma_n=2\]
for any $n$.

\parbf{Length spaces.}
Let $\spc{X}$ be a metric space.
If for any $\eps>0$ and any pair of points $x,y\in\spc{X}$, there is a path $\alpha$ connecting $x$ to $y$ such that
\[\length\alpha< \dist{x}{y}{}+\eps,\]
then $\spc{X}$ is called a \index{length space}\emph{length space} and the metric on $\spc{X}$ is called a \index{length metric}\emph{length metric}.\label{page:length metric}

Evidently, any geodesic space is a length space.

\begin{thm}{Exercise}
Show that any compact length space is geodesic.
\end{thm}


\parbf{Induced length metric.}
Directly from the definition, it follows that if $\alpha\:[0,1]\to\spc{X}$ is a path from $x$ to $y$ 
(that is, $\alpha(0)=x$ and $\alpha(1)=y$), then 
\[\length\alpha\ge \dist{x}{y}{}.\]
Set 
\[\yetdist{x}{y}{}=\inf\{\,\length\alpha\,\}\]
where the greatest lower bound is taken for all paths from $x$ to~$y$.
It is straightforward to check that $(x,y)\mapsto \yetdist{x}{y}{}$ is an \emph{$\infty$-metric};
that is, $(x,y)\mapsto \yetdist{x}{y}{}$ is a metric in the extended positive reals $[0,\infty]$. 
The metric $\yetdist{*}{*}{}$ is called the \index{induced length metric}\emph{induced length metric}.

\begin{thm}{Exercise}\label{ex:compact+connceted}
Let $\spc{X}$ be a complete length space.
Show that for any compact subset $K\subset\spc{X}$
there is a compact path-connected subset $K'\subset\spc{X}$ that contains $K$.  
\end{thm}

\begin{thm}{Exercise}\label{ex:compact=>complete}
Suppose $(\spc{X},\dist{*}{*}{})$ is a complete metric space.
Show that $(\spc{X},\yetdist{*}{*}{})$ is complete.
\end{thm}

Let $A$ be a subset of a metric space $\spc{X}$.
Given two points $x,y\in A$,
consider the value
\[\dist{x}{y}{A}=\inf_{\alpha}\{\,\length\alpha\,\},\]
where the greatest lower bound is taken for all paths $\alpha$ from $x$ to $y$ in~$A$.
In other words, $\dist{*}{*}{A}$ denotes the induced length metric on the subspace $A$.
(The notation $\dist{*}{*}{A}$ conflicts with the previously defined notation for distance $\dist{x}{y}{\spc{X}}$ in a metric space $\spc{X}$.
However, most of the time we will work with ambient length spaces where the meaning will be unambiguous.)

\section{Embedding theorem}

The main part of the following theorem is due to Alexandr Alexandrov~\cite{alexandrov-1948}.
The last part is very difficult; it was proved by Aleksei Pogorelov~\cite{pogorelov}.

\begin{thm}{Theorem}\label{thm:alexandrov+pogorelov}
A metric space $\spc{X}$ is isometric to the surface of a convex body in the Euclidean space if and only if $\spc{X}$ is a geodesic $\CBB(0)$ space that is homeomorphic to $\SSS^2$.

Moreover, $\spc{X}$ determines the convex body up to congruence.
\end{thm}

The convex body above is a compact convex subset in $\EE^3$;
we assume that it does not lie in a line but might degenerate to a plane figure, say $F$.
In the latter case, its surface is defined as two copies of $F$ glued along the boundary.
For nondegenerate convex body $B$, its surface is its boundary $\partial B$ equipped with the induced length metric. 

The only-if part of the theorem is the simplest; we will give a complete proof of it eventually.
The if part will be sketched.
We will not touch the last part.
