\appendix

\chapter{Solutions}

\parbf{\ref{ex:first-var-CBB}.} 
Let $\theta=\mangle\hinge q p x$.
We need to prove two inequalities
\begin{align*}
\dist{y}{\gamma(t)}{}
&\le
\dist{y}{q}{}-t\cdot \cos\theta+o(t),
\\
\dist{y}{\gamma(t)}{}
&\ge
\dist{y}{q}{}-t\cdot \cos\theta+o(t).
\end{align*}
The first one follows from \ref{ex:first-var}; it remains to prove the second one.
Arguing by contradiction, assume there is a sequence $t_n\to 0+$ such that for some fixed $\eps>0$
\[\angk{q}{x_n}{y}<\theta-\eps\]
for any $x_n=\gamma(t_n)$.
Since $\dist{q}{x_n}{}\to0$, we get
\[\mangle\hinge{x_n}{q}{y}>\pi-\theta+\tfrac\eps2,
\quad\text{and therefore}\quad
\mangle\hinge{x_n}{x}{y}<\theta-\tfrac\eps2\eqlbl{eq:<theta-eps/2}
\]
for all large $n$.%
\footnote{If the space is compact, then a subsequence of $[x_ny]$ should converge to a geodesic from $q$ to $y$ that makes angle $<\theta$ to $[qx]$.
It follows that there is a pair of distinct geodesics from $q$ to $y$ which contradicts \ref{ex:pi-angle}.
With the use of ultralimits (see for example \cite{petrunin2023pure}), this argument works in the general case.}

\begin{figure}[ht!]
\centering
\includegraphics{mppics/pic-3005}
\end{figure}

Without loss of generality, we may assume that 
\[\angk qxz>\theta-\tfrac\eps{10}\eqlbl{eq:>theta-eps/10}\]
for some point $z\in \left]qy\right]$.
(If it does not hold, shift $x$ to $q$.)

For each $n$, choose $z_n\in \left]x_ny\right]$ such that $\dist{x_n}{z_n}{}=\dist{q}{z}{}$.
Applying the triangle inequality, \ref{eq:<theta-eps/2}, \ref{eq:>theta-eps/10}, and the $\CBB(0)$ comparison, we get
\[\dist{z}{z_n}{}\ge\dist{a}{z}{}-\dist{a}{z_n}{} >\delta_0\]
for some $\delta_0>0$ and all large $n$.
Hence
\[\mangle\hinge y{x_n}{q}=\mangle\hinge y{z_n}{z}\ge \angk{y}{z_n}{z}>\delta_1,
\quad\text{and therefore}\quad
\mangle\hinge y{x_n}{p}<\pi-\delta_1\]
for some $\delta_1>0$ and all large $n$.
By  $\CBB(0)$ comparison,
\[\dist{q}{x_n}{}<\dist{p}{q}{}-\delta_2\]
for some $\delta_2>0$ and all large $n$.
Since $\dist{q}{x_n}{}\to 0$, we arrive at a contradiction with the triangle inequality.

\parbf{\ref{ex:|antisum|}.}
By the definition of anti-sum, we have
\begin{align*}
\langle u,u\rangle +\langle v,u\rangle +\langle w,u\rangle &\ge 0,
\\
\langle u,v\rangle +\langle v,v\rangle +\langle w,v\rangle &\ge 0,
\\
\langle u,w\rangle +\langle v,w\rangle +\langle w,w\rangle &= 0.
\end{align*}
Add the first two inequalities and subtract the last identity.

\parbf{\ref{prop:two-opp}.}
Applying \ref{prop:opposite}, we get $\langle v,w\rangle=-\langle u,w\rangle=|u|^2$.
Since $|u|\z=|v|\z=|w|$, we get
\begin{align*}
|v-w|^2&=|v|^2+|w|^2-2\cdot \langle v,w\rangle=
\\
&=0.
\end{align*}


\parbf{\ref{ex:3<,>=0}.} 
Note that 
\begin{align*}
\langle u,x\rangle +\langle v,x\rangle +\langle w,x\rangle &\ge 0,
\\
\langle u,-x\rangle +\langle v,-x\rangle +\langle w,-x\rangle &\ge 0.
\end{align*}
By \ref{prop:opposite}, $\langle y,x\rangle=-\langle y ,-x\rangle$ for any $y\in \T_p$.
Hence the result.


\parbf{\ref{ex:-u}.}
If $u=-u^*$, then $|u^*|= |u|$; it remains to prove the converse.

Note that $\langle u^*,u^*\rangle +\langle u,u^*\rangle = 0$
and $\langle u^*,u\rangle +\langle u,u\rangle \ge 0$.
(Hence $|u^*|\le |u|$, but we do not need it.)

If $|u^*|= |u|$, then $|u^*|^2=|u|^2=-\langle u,u^*\rangle$.
Therefore $\mangle(u,u^*)=\pi$ or $u=u^*=0$.
Hence $u=-u^*$.

\parbf{\ref{ex:sba-2+2-short}}; \textit{only-if part.}
Let us start with two model triangles $[\tilde x\tilde y\tilde p]=\modtrig(xyp)$ and $[\tilde x\tilde y\tilde q]=\modtrig(xyq)$ such that $\tilde p$ and $\tilde q$ lie on the opposite sides of the line $\tilde x\tilde y$.

Suppose $[\tilde x \tilde y]$ intersects $[\tilde p\tilde q]$ at a point $\tilde z$.
In this case by $\CAT(0)$ comparison we have that
\[\dist{\tilde p}{\tilde q}{\EE^2}=\dist{\tilde p}{\tilde z}{\EE^2}-\dist{\tilde z}{\tilde q}{\EE^2}\le \dist{p}{q}{\spc{X}}.\]
Hence the problem is solved.

\begin{wrapfigure}{r}{30mm}
\vskip-2mm
\centering
\includegraphics{mppics/pic-6}
\end{wrapfigure}

Now suppose $[\tilde p \tilde q]$ does not intersect $[\tilde x\tilde y]$.
Without loss of generality, we may assume that $[\tilde p \tilde q]$ crosses the line $\tilde x\tilde y$ behind $\tilde x$.

Let us rotate $\tilde p$ around $\tilde x$ so that $\tilde x$ will lie between $\tilde p$ and $\tilde q$.
It will result in decreasing the distance $\dist{\tilde p}{\tilde y}{}$,
by the triangle inequality we have that 
\begin{align*}
\dist{\tilde p}{\tilde q}{\EE^2}&=\dist{\tilde p}{\tilde x}{\EE^2}+\dist{\tilde x}{\tilde q}{\EE^2}=
\\
&=\dist{p}{x}{\spc{X}}+\dist{x}{q}{\spc{X}}\ge
\\
&\ge \dist{p}{q}{\spc{X}}.
\end{align*}

\parit{If part.}
Suppose $\tilde p,\tilde q,\tilde x,\tilde y\in\EE^2$ satisfies the conditions.
We can assume that $\dist{\tilde p}{\tilde x}{}+\dist{\tilde p}{\tilde y}{}+\dist{\tilde q}{\tilde x}{}+\dist{\tilde q}{\tilde y}{}$ takes maximal possible value.
Note that in this case $\tilde p\tilde x\tilde q\tilde y$ is a convex quadrangle and 
\begin{align*}
\dist{\tilde p}{\tilde x}{}&=\dist{p}{x}{},
&
\dist{\tilde p}{\tilde y}{}&=\dist{p}{y}{},
\\
\dist{\tilde q}{\tilde x}{}&=\dist{q}{x}{},
&
\dist{\tilde q}{\tilde y}{}&=\dist{q}{y}{}.
\end{align*}

Since 
\begin{align*}
\dist{\tilde p}{\tilde q}{}&\ge\dist{p}{q}{},
&
\dist{\tilde x}{\tilde y}{}&\ge\dist{x}{y}{},
\intertext{we get}
\mangle\hinge{\tilde p}{\tilde x}{\tilde y}&\ge \angk{p}{x}{y},
&
\mangle\hinge{\tilde p}{\tilde x}{\tilde y}&\ge \angk{p}{x}{y}.
\end{align*}

Suppose $\tilde z=t\cdot \tilde x+(1-t)\cdot \tilde y$ for some $t\in[0,1]$.
Consider the model triangles $[\dot p\dot x\dot y]=\modtrig(pxy)$ and $[\dot q\dot x\dot y]=\modtrig(qxy)$;
let $\dot z=t\cdot \dot x+(1-t)\cdot \dot y$.
The last inequalities imply 
\begin{align*}
\dist{\tilde p}{\tilde z}{}&\le \dist{\dot p}{\dot z}{},
&
\dist{\tilde q}{\tilde z}{}&\le \dist{\dot q}{\dot z}{}.
\end{align*}
Therefore
\begin{align*}
\dist{p}{q}{}
&\le \dist{\tilde p}{\tilde q}{}\le
\\
&\le \dist{\tilde p}{\tilde z}{}+\dist{\tilde q}{\tilde z}{}\le
\\
&\le \dist{\dot p}{\dot z}{}+\dist{\dot q}{\dot z}{}.
\end{align*}


