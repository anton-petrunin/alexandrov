\chapter{Semiconcave functions}

\section{Real-to-real functions}

Choose $\lambda\in \RR$.
Let $s\:\II\to\RR$ be a locally Lipschitz function defined on an interval $\II$.
We say that $s$ is \index{$\lambda$-concave function}\emph{$\lambda$-concave} if $s''\le \lambda$, where the second derivative $s''$ is understood in the sense of distributions.

Equivalently, \textit{$s$ is $\lambda$-concave if the function $h\:t\mapsto s(t)-\lambda\cdot\tfrac{t^2}2$ is concave}.
Concavity can be defined via \index{Jensen inequality}\emph{Jensen inequality}; that is,
\[h(s\cdot t_0+(1-s)\cdot t_1)\ge s\cdot h(t_0)+(1-s)\cdot h(t_1)\]
for any $t_0,t_1\in \II$ and $s\in[0,1]$.
It could be also defined via existence of (local) upper \index{supporting function}\emph{support} at any point:
\textit{for any $t_0\in \II$ there is a linear function $\ell$ that (locally) supports $h$ at $t_0$ from above;
that is, $\ell(t_0)\z= h(t_0)$ and $\ell(t)\ge h(t)$ for any $t$ (in a neighborhood of $t_0$)}.

The equivalence of these definitions is assumed to be known.

\section{Function comparison}


A function on a metric space $\spc{L}$ will usually mean a \textit{locally Lipschitz real-valued function defined in an open subset of $\spc{L}$}.
The domain of definition of a function $f$ will be denoted by $\Dom f$.


Let $f$ be a function on a metric space $\spc{L}$.
We say that $f$ is \index{$\lambda$-concave function}\emph{$\lambda$-concave} (briefly $f''\le \lambda$) if 
for any unit-speed geodesic $\gamma\:\II\z\to \Dom f$
the real-to-real function $t\mapsto f\circ\gamma(t)$ is $\lambda$-concave.

The following proposition is conceptual ---
it reformulates a global geometric condition into an infinitesimal condition on distance functions.

\begin{thm}{Proposition}
A geodesic space $\spc{L}$ is $\Alex{0}$ if and only if $f''\le 1$ for any function $f$ of the following type 
\[f\:x\mapsto \tfrac12\cdot\dist[2]{p}{x}{}.\] 
\end{thm} 

\parit{Proof.}
Choose a unit-speed geodesic $\gamma$ in $\spc{L}$ and two points $x=\gamma(t_0)$, $y=\gamma(t_1)$ for some $t_0<t_1$.
Consider the model triangle $\trig{\tilde p}{\tilde x}{\tilde y}\z=\modtrig(p x y)$.
Let $\tilde \gamma\:[t_0,t_1]\to\EE^2$ be the unit-speed parametrization of $[\tilde x \tilde y]$ from $\tilde x$ to $\tilde y$.

Set
\begin{align*} 
\tilde r(t)&\df\dist{\tilde p}{\tilde\gamma(t)}{},
& 
r(t)&\df\dist{p}{\gamma(t)}{}.
\end{align*}
Clearly, $\tilde r(t_0)=r(t_0)$ and $\tilde r(t_1)=r(t_1)$.
Note that the point-on-side comparison (\ref{point-on-side}) is equivalent to 
\[t_0\le t\le t_1
\qquad\Longrightarrow\qquad
\tilde r(t)\le r(t)
\eqlbl{eq:r=<r}\]
for any $\gamma$ and $t_0<t_1$.

Set
\begin{align*} 
\tilde h(t)&=\tfrac12\cdot \tilde r^2(t) - \tfrac12\cdot t^2,
&
h&=\tfrac12\cdot r^2(t) - \tfrac12\cdot t^2.
\end{align*}
Note that $\tilde h$ is linear,
$\tilde h(t_0)=h(t_0)$ and $\tilde h(t_1)=h(t_1)$.
Observe that the Jensen inequality for the function $h$ is equivalent to \ref{eq:r=<r}.
Hence the proposition follows.
\qeds

\section{Semiconcave functions} 

We will write $f''\le \phi$ if for any point $x\in \Dom f$ and any $\eps>0$ there is a neighborhood $U\ni x$ such that 
the restriction $f|_U$ is $(\phi(x)+\eps)$-concave.
Here we assume that $\phi$ is continuous and defined in $\Dom f$.

If $f''\le \phi$ for some continuous function $\phi$, then $f$ is called  \index{semiconcave}\emph{semiconcave}.


\begin{thm}{Exercise}\label{ex:distfun-semiconcave}
Let $f$ be a distance function on a geodesic $\CBB(0)$ space $\spc{L}$;
that is, $f(x)\equiv\dist{p}{x}{}$ for some $p\in \spc{L}$.
Show that $f''\le \tfrac1f$.
In particular, $f$ is semiconcave in $\spc{L}\setminus\{p\}$.
\end{thm}

\section{Completion}

Given a metric space $\spc{X}$, 
consider the set $\spc{C}$ of all Cauchy sequences in $\spc{X}$.
Note that for any two Cauchy sequences $(x_n)$ and $(y_n)$ the right-hand side in \ref{eq:cauchy-dist} is defined;
moreover, it defines a semimetric on~$\spc{C}$
\[\dist{(x_n)}{(y_n)}{\spc{C}}\df\lim_{n\to\infty}\dist{x_n}{y_n}{\spc{X}}.\eqlbl{eq:cauchy-dist}\]
The corresponding metric space is called the \index{completion}\emph{completion} of $\spc{X}$;
it will be denoted by $\bar{\spc{X}}$.

It is straightforward to check that \textit{completion is complete.}
  
For each point $x\in\spc{X}$, one can consider a constant sequence $x_n=x$ which is Cauchy.
It defines a natural inclusion map $\spc{X}\hookrightarrow \bar{\spc{X}}$.
It is easy to check that this map is distance-preserving.
In particular, we can (and will) consider $\spc{X}$ as a subset of $\bar{\spc{X}}$.
Note that \textit{$\spc{X}$ is a dense subset in its completion $\bar{\spc{X}}$}.

\section{Space of directions} 
\label{sec:space+directions}

Let $\spc{X}$ be a space with defined angles.
Given $p\in \spc{X}$, consider the set $\mathfrak{S}_p$ of all nontrivial unit-speed geodesics starting at $p$.
By \ref{claim:angle-3angle-inq}, the triangle inequality holds for $\mangle$ on $\mathfrak{S}_p$,
that is, $(\mathfrak{S}_p,\mangle)$ 
forms a semimetric space.

The metric space corresponding to  $(\mathfrak{S}_p,\mangle)$ is called the \index{direction!space of geodesic directions}\emph{space of geodesic directions} at $p$, denoted by $\Sigma'_p$ or $\Sigma'_p\spc{X}$.
The elements of $\Sigma'_p$ are called \index{geodesic direction}\emph{geodesic directions} at $p$.
Each geodesic direction is formed by an equivalence class of geodesics starting from $p$ 
for the equivalence relation 
\[[px]\sim[py]\quad \iff\quad \mangle\hinge pxy=0;\]
the direction of $[px]$ is denoted by $\dir px $.\index{$\dir{p}{q}$}
(If $\spc{X}$ is $\CBB$, then by \ref{ex:0-angle}, $[px]\sim[py]$ if and only if $[px]\subset [py]$ or $[px]\supset[py]$.)


The completion of $\Sigma'_p$ is called the \index{direction!space of directions}\emph{space of directions} at $p$ and is denoted by $\Sigma_p$ or $\Sigma_p\spc{X}$.
The elements of $\Sigma_p$ are called \index{direction}\emph{directions} at $p$.

\section{Tangent space}\label{sec: tangent space}

\textbf{Cone construction.}
The \index{cone}\emph{Euclidean cone} $\spc{V}=\Cone\spc{X}$ 
over a metric space $\spc{X}$
is defined as the metric space whose underlying set consists of
equivalence classes in
$[0,\infty)\times \spc{X}$ with the equivalence relation ``$\sim$'' given by $(0,p)\sim (0,q)$ for any points $p,q\in\spc{X}$,
and whose metric is given by the cosine rule
\[
\dist{(s,p)}{(t,q)}{\spc{V}} 
=
\sqrt{s^2+t^2-2\cdot s\cdot t\cdot \cos\theta},
\]
where $\theta= \min\{\pi, \dist{p}{q}{\spc{X}}\}$.

Note that \textit{$\Cone\SSS^n$ is isometric to $\EE^{n+1}$.}
This is a leading example;
further, we generalize several notions of Euclidean space to the Euclidean cones. 

The point in $\spc{V}$ that corresponds $(t,x)\z\in[0,\infty)\times \spc{X}$ will be denoted by $t\cdot x$.
The point in $\spc{V}$ formed by the equivalence class of $\{\0\}\times\spc{X}$ is called the \index{origin of a  cone}\emph{origin} of the cone and is denoted by $\0$ or $\0_{\spc{V}}$.
For $v\in\spc{V}$ the distance $\dist{\0}{v}{\spc{V}}$ is called the norm of $v$ and is denoted by $|v|$ or $|v|_{\spc{V}}$.
The \index{scalar product}\emph{scalar product} $\<v,w\>$
of $v=s\cdot p$ and $w=t\cdot q$
is defined by 
\[\<v,w\>
\df |v|\cdot|w|\cdot\cos\theta
\]
where $\theta= \min\{\pi, \dist{p}{q}{\spc{X}}\}$; we set $\<v,w\>\df0$ if $v=\0$ or $w=\0$.

\parbf{Tangent space.}
The Euclidean cone $\Cone\Sigma_p$ over the space of directions $\Sigma_p$ is called the \index{tangent space}\emph{tangent space} at $p$ and denoted by $\T_p$ or $\T_p\spc{X}$.
The elements of $\T_p\spc{X}$ will be called \index{tangent vector}\emph{tangent vectors} at $p$
(despite the fact that $\T_p$ is only a cone --- not a vector space).
The space of directions $\Sigma_p$ can be (and will be) identified with the unit sphere in~$\T_p$.

\begin{thm}{Exercise}\label{ex:Tan-is-CBB(0)}
Show that tangent spaces of $\CBB(\kappa)$ space are $\CBB(0)$.

\end{thm}


\section{Differential}\index{differential of a function}

Let $\spc{X}$ be a space with defined angles.
Let $f$ be a semiconcave function on $\spc{X}$ and $p\in \Dom f$.
Choose a unit-speed geodesic $\gamma$ that starts at $p$;
let $\xi\in\Sigma_p$ be its direction.
Define 
\[(\dd_pf)(\xi)\df(f\circ\gamma)^+(0),\]
here $(f\circ\gamma)^+$ denoted right derivative of $(f\circ\gamma)$;
it is defined since $f$ is semiconcave.

By the following exercise, the value $(\dd_pf)(\xi)$ is defined; that is, it does not depend on the choice of $\gamma$.
Moreover, $\dd_pf$ is a Lipschitz function on $\Sigma'_p$.
It follows that the function $\dd_pf\:\Sigma_p'\to\RR$ can be extended to a Lipschitz function $\dd_pf\:\Sigma_p\to\RR$.
Further, we can extend it to the tangent space by setting 
\[(\dd_pf)(r\cdot \xi)
\df
r\cdot (\dd_pf)(\xi)\]
for any $r\ge 0$ and $\xi\in\Sigma_p$.
The obtained function $\dd_pf\:\T_p\to\RR$ is Lipschitz;
it is called the \index{differential}\emph{differential} of $f$ at $p$.

\begin{thm}{Exercise}\label{ex:df(xi)}
Let $f$ be a semiconcave function on a geodesic space $\spc{X}$ with defined angles.
Suppose $\gamma_1$ and $\gamma_2$ are unit-speed geodesics that start at $p\in \Dom f$;
denote by $\theta$ the angle between $\gamma_1$ and $\gamma_2$ at $p$.
Show that 
\[|(f\circ\gamma_1)^+(0)-(f\circ\gamma_2)^+(0)|\le L\cdot \theta,\]
where $L$ is the Lipschitz constant of $f$ in a neighborhood of $p$.
\end{thm}

\begin{thm}{Exercise}\label{ex:d(distfun)}
Let $p$ and $q$ be distinct points in a $\CBB(0)$ space.
Denote by $\xi$ the direction of a geodesic $[pq]$ at $p$.
Show that 
\[\dd_p\distfun_q(v)\le -\langle\xi,v\rangle\]
for any $v\in\T_p$.
\end{thm}


\section{Gradient}\label{sec:grad-def}

\begin{thm}{Definition}\label{def:grad} 
Let $f$ be a semiconcave function on a geodesic space $\spc{X}$ with defined angles.
A tangent vector $g\in \T_p$ is called a 
\index{gradient}\emph{gradient} of $f$ at $p$ 
(briefly,  $g\z=\nabla_p f$\index{$\nabla$}) if
\begin{subthm}{}
$(\dd_p f)(w)\le \<g,w\>$ for any $w\in \T_p$, and
\end{subthm}

\begin{subthm}{}
$(\dd_p f)(g) = \<g,g\> .$
\end{subthm}
\end{thm}

\begin{thm}{Proposition}\label{prop:grad-exist}
Suppose that a semiconcave function $f$ is defined in a neighborhood of a point $p$ in a $\CBB(\kappa)$ space.
Then the gradient $\nabla_pf$ is uniquely defined.
\end{thm}


\begin{thm}{Key lemma}\label{lem:ohta} 
Let $f$ be a $\lambda$-concave function that is defined in a neighborhood of a point $p$
in a geodesic $\CBB(\kappa)$ space $\spc{L}$. 
Then for any $u,v\in \T_p$, we have
\[s\cdot \sqrt{|u|^2+2\cdot\<u,v\> +|v|^2}
\ge 
(\dd_p f)(u)+(\dd_p f)(v),\]
where
\[s=\sup\set{(\dd_p f)(\xi)}{\xi\in\Sigma_p}.\]

\end{thm}

\parit{Proof.}
We will assume $\kappa=0$;
the general case requires only minor modifications.
We can assume that $v\ne 0$, $w\ne 0$, and $\alpha=\mangle(u,v)>0$; otherwise, the statement is trivial.

{

\begin{wrapfigure}{r}{34 mm}
\vskip-4mm
\centering
\includegraphics{mppics/pic-1205}
\vskip0mm
\end{wrapfigure}

Prepare a model configuration of five points: $\tilde p$, $\tilde u$, $\tilde v$, $\tilde q$, $\tilde w\in\EE^2$ such that
\begin{itemize}
\item $\mangle\hinge{\tilde p}{\tilde u}{\tilde v}=\alpha$, 
\item $\dist{\tilde p}{\tilde u}{}=|u|$, 
\item $\dist{\tilde p}{\tilde v}{}=|v|$,
\end{itemize}
}
\begin{itemize}
\item $\tilde q$ lies on an extension of $[\tilde p\tilde v]$ so that $\tilde v$ is the midpoint of $[\tilde p\tilde q]$, 
\item $\tilde w$ is the midpoint between $\tilde u$ and ${\tilde v}$.
\end{itemize}
Note that 
\[\dist{\tilde p}{\tilde w}{}
=
\tfrac{1}{2}\cdot\sqrt{|u|^2+2\cdot\<u,v\>+|v|^2}.\]

We can assume that there are geodesics in the directions of $u$ and $v$;
the latter follows since the geodesic space of directions $\Sigma'_p$ is dense in $\Sigma_p$.
Choose geodesics $\gamma_u$ and $\gamma_v$ in the direction in the directions of $u$ and $v$;
let us assume that they are parametrized with speed $|u|$ and $|v|$ respectively.
For all small $t>0$, construct points $u_t,v_t,q_t,w_t\in \spc{L}$ as follows.
\begin{itemize}
\item $v_t=\gamma_v(t)$,\quad  $q_t=\gamma_v(2\cdot t)$
\item $u_t=\gamma_u(t)$.
\item $w_t$ is the midpoint of $[u_t v_t]$.
\end{itemize}
Clearly 
\[\dist{p}{u_t}{}=t\cdot |u|,\qquad \dist{p}{v_t}{}=t\cdot|v|,\qquad \dist{p}{q_t}{}=2\cdot t\cdot|v|.\] 
Since $\mangle(u,v)$ is defined, 
we have 
\[\dist{u_t}{v_t}{}=t\cdot\dist{\tilde u}{\tilde v}{}+o(t),
\qquad
\dist{u_t}{q_t}{}=t\cdot\dist{\tilde u}{\tilde q}{}+o(t).\]

From the point-on-side and hinge comparisons (\ref{point-on-side}$+$\ref{angle}), we have
\[\angk{v_t}p{w_t}
\ge
\angk{v_t}p{u_t}
\ge
\mangle\hinge{\tilde v}{\tilde p}{\tilde u}+\tfrac{o(t)}t\]
and
\[\angk{v_t}{q_t}{w_t}
\ge
\angk{v_t}{q_t}{u_t}
\ge
\mangle\hinge{\tilde v}{\tilde q}{\tilde u}+\tfrac{o(t)}t.\]
Clearly, 
$\mangle\hinge{\tilde v}{\tilde p}{\tilde u}+\mangle\hinge{\tilde v}{\tilde q}{\tilde u}=\pi$. 
From the adjacent angle comparison (\ref{2-sum}), 
$\angk{v_t}p{u_t}\z+\angk{v_t}{u_t}{q_t}\le \pi$.
Hence
$\angk{v_t}p{w_t}
\to
\mangle\hinge{\tilde v}{\tilde p}{\tilde w}$ as $t\to0+$
and thus 
\[\dist{p}{w_t}{}=t\cdot\dist{\tilde p}{\tilde w}{}+o(t).\]

Since $f$ is $\lambda$-concave, we have 
\begin{align*}
2\cdot f(w_t)&\ge f(u_t)+f(v_t)+\tfrac\lambda4\cdot\dist[2]{u_t}{v_t}{}=
\\
&=2\cdot f(p)
+t\cdot [(\dd_p f)(u)+(\dd_p f)(v)]+o(t).
\end{align*}
 
Applying $\lambda$-concavity of $f$, we have
\[(\dd_p f)(\dir p{w_t})
\ge 
\frac{t\cdot[(\dd_p f)(u)+(\dd_p f)(v)]
+o(t)}{2\cdot t\cdot\dist[{{}}]{\tilde p}{\tilde w}{}+o(t)}.\]
The key lemma follows.
\qeds

\begin{thm}{Exercise}\label{ex:first-var-CBB}
Let $\hinge q p x$ be a hinge in  a $\CBB(0)$ space and $y\in \left]qp\right[$.
Suppose that $\gamma$ is the unit speed parametrization of $[qx]$ from $q$ to $x$.
Show that
\[\dist{y}{\gamma(t)}{}
=
\dist{q}{y}{}-t\cdot \cos(\mangle\hinge q p x)+o(t).\]
Conclude that 
\[(\dd_q\distfun_y)(w)=-\langle\dir qp,w\rangle\]
for any $w\in \T_x$
\end{thm}

\parit{Proof of \ref{prop:grad-exist}; uniqueness.} 
If $g,g'\in \T_p$ are two gradients of $f$,
then 
\begin{align*}
\<g,g\>
&=(\dd_p f)(g)\le \<g,g'\>,
&
\<g',g'\>
&=(\dd_p f)(g')\le \<g,g'\>.
\end{align*}
Therefore,
\[\dist[2]{g}{g'}{}=\<g,g\>-2\cdot\<g,g'\>+\<g',g'\>\le0.\] 
It follows that $g=g'$.

\parit{Existence.} 
Note first that if $\dd_p f\le 0$, then one can take $\nabla_p f=\0$.

Otherwise, if $s=\sup\set{(\dd_p f)(\xi)}{\xi\in\Sigma_p}>0$, 
it is sufficient to show that there is  $\overline{\xi}\in \Sigma_p$ such that 
\[
(\dd_p f)\left(\overline{\xi}\right)=s.
\eqlbl{overlinexi}
\]
Indeed, suppose $\overline{\xi}$ exists.
Applying \ref{lem:ohta} for $u=\overline{\xi}$, $v=\eps\cdot w$ with $\eps\to0+$, 
we get
\[(\dd_p f)(w)\le \<w,s\cdot\overline{\xi}\>\] 
for any $w\in\T_p$;
that is, $s\cdot\overline{\xi}$ is the gradient at $p$.

Take a sequence of directions $\xi_n\in \Sigma_p$, such that $(\dd_p f)(\xi_n)\to s$.
Applying \ref{lem:ohta} for $u=\xi_n$ and $v=\xi_m$, we get
\[s
\ge
\frac{(\dd_p f)(\xi_n)+(\dd_p f)(\xi_m)}{\sqrt{2+2\cdot\cos\mangle(\xi_n,\xi_m)}}.\]
Therefore $\mangle(\xi_n,\xi_m)\to0$ as $n,m\to\infty$;
that is, the sequence $\xi_n$ is Cauchy.
Clearly $\overline{\xi}=\lim_n\xi_n$ meets \ref{overlinexi}.
\qeds

\section{Comments}

The function comparison of $\CBB(-1)$ states that 
$f''\le f$ for any function of the type $f=\cosh\circ\distfun_p$.
Similarly, the function comparison of $\CBB(1)$ states that for any point $p$ we have
$f''\le -f$ for the function $f=-\cos\circ\distfun_p$
defined in $\oBall(p,\pi)$.
The meaning of these inequalities is the same --- distance functions in $\CBB(\kappa)$ are more concave than distance functions in $\MM(\kappa)$.
