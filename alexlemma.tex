\chapter{Angles}

\section{Definition}\label{sec:angles}

The angle measure of a hinge $\hinge p x y$ is defined as the following limit
\[\mangle\hinge p x y=\lim_{\bar x,\bar y\to p} \angk p{\bar x}{\bar y},\]
where $\bar x\in\left]p x\right]$ and $\bar y\in\left]p y\right]$.

Note that if $\mangle\hinge p x y$ is defined, then
\[0\le \mangle\hinge p x y\le \pi.\]

\begin{thm}{Exercise}
Suppose that in the above definition, one uses spherical or hyperbolic model angles instead of Euclidean.
Show that it does not change the value $\mangle\hinge p x y$.
\end{thm}


\begin{thm}{Exercise}\label{ex:undefined-angle}
Give an example of a hinge $\hinge p x y$ in a metric space with an undefined angle measure $\mangle\hinge p x y$.
\end{thm}

\section{Triangle inequality}

\begin{thm}{Proposition}
\label{claim:angle-3angle-inq}
Let  $[px_1]$, $[px_2]$, and $[px_3]$ be three geodesics in a metric space.
Suppose all the angle measures $\alpha_{i j}=\mangle\hinge p {x_i}{x_j}$ are defined.
Then 
\[\alpha_{13}\le \alpha_{12}+\alpha_{23}.\]

\end{thm}

\parit{Proof.}
Since $\alpha_{13}\le\pi$, we can assume that $\alpha_{12}+\alpha_{23}< \pi$.
Denote by $\gamma_i$ the unit-speed parametrization of $[px_i]$ from $p$ to $x_i$.
Given any $\eps>0$, for all sufficiently small $t,\tau,s\in\RR_{\ge0}$ we have
\begin{align*}
\dist{\gamma_1(t)}{\gamma_3(\tau)}{}
&\le 
\dist{\gamma_1(t)}{\gamma_2(s)}{}+\dist{\gamma_2(s)}{\gamma_3(\tau)}{}<\\
&<
\sqrt{t^2+s^2-2\cdot t\cdot  s\cdot \cos(\alpha_{12}+\eps)} +
\\
&\quad+\sqrt{s^2+\tau^2-2\cdot s\cdot \tau\cdot \cos(\alpha_{23}+\eps)}\le
\end{align*}

\begin{wrapfigure}{o}{30 mm}
\vskip-6mm
\centering
\includegraphics{mppics/pic-615}
\vskip3mm
\end{wrapfigure}

Below we define 
$s(t,\tau)$ so that for 
$s=s(t,\tau)$, this chain of inequalities can be continued as follows:
\[\le
\sqrt{t^2+\tau^2-2\cdot t\cdot \tau\cdot \cos(\alpha_{12}+\alpha_{23}+2\cdot \eps)}.
\]

Thus for any $\eps>0$, 
\[\alpha_{13}\le \alpha_{12}+\alpha_{23}+2\cdot \eps.\]
Hence the result follows.

To define $s(t,\tau)$, consider three half-lines $\tilde \gamma_1$, $\tilde \gamma_2$, $\tilde \gamma_3$ on a Euclidean plane starting at one point, such that
$\mangle(\tilde \gamma_1,\tilde \gamma_2)\z=\alpha_{12}+\eps$,
$\mangle(\tilde \gamma_2,\tilde \gamma_3)\z=\alpha_{23}+\eps$,
and $\mangle(\tilde \gamma_1,\tilde \gamma_3)\z=\alpha_{12}\z+\alpha_{23}\z+2\cdot \eps$.
We parametrize each half-line by the distance from the starting point.
Given two positive numbers $t,\tau\in\RR_{\ge0}$, let $s=s(t,\tau)$ be 
the number such that 
$\tilde \gamma_2(s)\in[\tilde \gamma_1(t)\ \tilde \gamma_3(\tau)]$. 
Clearly, $s\le\max\{t,\tau\}$, so $t,\tau,s$ may be taken sufficiently small.
\qeds 

\begin{thm}{Exercise}\label{ex:adjacent-angles}
Prove that the sum of adjacent angles is at least $\pi$.

More precisely: suppose two hinges $\hinge pxz$ and $\hinge pyz$ are \index{adjacent hinges}\emph{adjacent};
that is, they share side $[pz]$, and the union of two sides $[px]$ and $[py]$ form a geodesic $[xy]$.
Show that
\[\mangle\hinge pxz+\mangle\hinge pyz\ge \pi\]
whenever  each angle on the left-hand side is defined.

Give an example showing that the inequality can be strict.
\end{thm}

\begin{thm}{Exercise}\label{ex:first-var}
Assume that the angle measure of $\hinge q p x$ is defined.
Let $\gamma$ be the unit speed parametrization of $[qx]$ from $q$ to $x$.
Show that
\[\dist{p}{\gamma(t)}{}
\le
\dist{q}{p}{}-t\cdot \cos(\mangle\hinge q p x)+o(t).\]

\end{thm}

\section{Alexandrov's lemma}

Recall that $[xy]$ denotes a geodesic from $x$ to $y$;
set  
\[
\left]x y\right]=[xy]\setminus\{x\},
\quad
\left[x y\right[=[xy]\setminus\{y\},
\quad
\left]x y\right[=[xy]\setminus\{x,y\}.\]

\begin{thm}{Lemma}
\index{Alexandrov's lemma}
\label{lem:alex}  
Let $p,x,y,z$ be distinct points in a metric space such that $z\in \left]x y\right[$.
Then 
the following expressions for the Euclidean model angles have the same sign:

\begin{subthm}{lem-alex-difference}
$\angk x p y
-\angk x p z$,
\end{subthm} 

\begin{subthm}{lem-alex-angle}
$\angk z p x
+\angk z p y -\pi$.
\end{subthm}

\begin{wrapfigure}{r}{25mm}
\vskip-6mm
\centering
\includegraphics{mppics/pic-730}
\end{wrapfigure}

The same holds for the hyperbolic and spherical model angles, 
but in the latter case, one has to assume in addition that
\[\dist{p}{z}{}+\dist{p}{y}{}+\dist{x}{y}{}< 2\cdot\pi.\]

\end{thm}


\parit{Proof.} 
Consider the model triangle $\trig{\tilde x}{\tilde p}{\tilde z}=\modtrig(x p z)$.
Take 
a point $\tilde y$ on the extension of 
$[\tilde x \tilde z]$ beyond $\tilde z$ so that $\dist{\tilde x}{\tilde y}{}=\dist{x}{y}{}$ (and therefore $\dist{\tilde x}{\tilde z}{}=\dist{x}{z}{}$). 

\begin{wrapfigure}{r}{33mm}
\vskip-0mm
\centering
\includegraphics{mppics/pic-740}
\end{wrapfigure}

Since increasing the opposite side in a plane triangle increases the corresponding angle, 
the following expressions have the same sign:
\begin{enumerate}[(i)]
\item $\mangle\hinge{\tilde x}{\tilde p}{\tilde y}-\angk{x}{p}{y}$,
\item $\dist{\tilde p}{\tilde y}{}-\dist{p}{y}{}$,
\item $\mangle\hinge{\tilde z}{\tilde p}{\tilde y}-\angk{z}{p}{y}$.
\end{enumerate}
Since 
\[\mangle\hinge{\tilde x}{\tilde p}{\tilde y}=\mangle\hinge{\tilde x}{\tilde p}{\tilde z}=\angk{x}{p}{z}\]
and
\[ \mangle\hinge{\tilde z}{\tilde p}{\tilde y}
=\pi-\mangle\hinge{\tilde z}{\tilde x}{\tilde p}
=\pi-\angk{z}{x}{p},\]
the statement follows.

The spherical and hyperbolic cases can be proved the same way.
\qeds

\begin{thm}{Exercise}\label{ex:alex-lemma-cat}
Assume $p,x,y,z$ are as in Alexandrov's lemma.
Show that
\[\angk p x y
\ge
\angk p x z + \angk p z y,\]
with equality if and only if the expressions in \ref{SHORT.lem-alex-difference} and \ref{SHORT.lem-alex-angle} vanish.
\end{thm}

\section{CBB comparison}

Note that 
\[p\in\left]x y\right[
\quad\Longrightarrow\quad
\angk pxy=\pi.
\]
Applying it with Alexandrov's lemma and $\CBB(0)$ comparison, we get the following claim and its corollary.

\begin{thm}{Claim}\label{clm:angle-mono}
If $p,x,y,z$ are points in a $\CBB(0)$ such that $p\in\left]x y\right[$, then 
\[\angk xyz\le \angk xpz.\]
\end{thm}

\begin{wrapfigure}{r}{25mm}
\vskip-0mm
\centering
\includegraphics{mppics/pic-750}
\end{wrapfigure}

\begin{thm}{Exercise}\label{ex:noncreasing}
Let $\hinge p x y$ be a hinge in a $\CBB(0)$ space.
Consider the function
\[f\:(\dist{p}{\bar x}{},\dist{p}{\bar y}{})\mapsto \angk p{\bar x}{\bar y},\]
where $\bar x\in\left]p x\right]$ and $\bar y\in\left]p y\right]$.
Show that $f$ is nonincreasing in each argument.
\end{thm}

Note that \ref{ex:noncreasing} implies the following.

\begin{thm}{Claim}\label{clm:angle-defined}
For any hinge $\hinge p x y$ in a $\CBB(0)$ space,
the angle measure $\mangle\hinge p x y$ is defined, and
\[\mangle\hinge p x y\ge \angk p x y.\]

\end{thm}

\begin{thm}{Exercise}\label{ex:0-angle}
Let $\hinge p x y$ be a hinge in a $\CBB(0)$ space.
Suppose $\mangle\hinge p x y=0$; show that $[px]\subset [py]$ or $[py]\subset [px]$.
\end{thm}

\begin{thm}{Exercise}\label{ex:pi-angle}
Let $[xy]$ be a geodesic in a $\CBB(0)$ space.
Suppose $z\in \left]xy\right[$ show that there is a unique geodesic $[xz]$ and $[xz]\subset [xy]$.
\end{thm}


\begin{thm}{Exercise}\label{ex:adjacent-CBB}
Let $\hinge pxz$ and $\hinge pyz$ be adjacent hinges in a $\CBB(0)$ space.
Show that
\[\mangle\hinge pxz+\mangle\hinge pyz= \pi.\]
\end{thm}


\begin{thm}{Exercise}
\label{ex:pxyvw}
Let 
$p,x,y$ in a $\CBB(0)$ space
and $v,w\in \left]xy\right[$.
Show that  
\[
\angk xyp=\angk xvp
\quad\Longleftrightarrow\quad
\angk xyp=\angk xwp.
\]

\end{thm}

\section{Hinge comparison}

Let $\hinge pxy$ be a hinge in a $\CBB(0)$ space.
By \ref{ex:0-angle}, the angle measure $\mangle\hinge pxy$ is defined and
\[\mangle\hinge pxy\ge \angk pxy.\]

Further, according to \ref{ex:adjacent-CBB}, we have 
\[\mangle\hinge pxz+\mangle\hinge pyz=\pi\]
for adjacent hinges $\hinge pxz$ and $\hinge pyz$ in a $\CBB(0)$ space.

The following theorem implies that a geodesic space is $\CBB(0)$ if
the above conditions hold for all its hinges.

\begin{thm}{Theorem}\label{thm:angle-cbb}
A geodesic space $\spc{L}$ is $\CBB(0)$ if the following conditions hold.

\begin{subthm}{angle-a}
For any hinge $\hinge x p y$ in $\spc{L}$, the angle 
$\mangle\hinge x p y$ is defined and 
\[\mangle\hinge x p y\ge\angk x p y.\]
\end{subthm}

\begin{subthm}{angle-b}
For any two adjacent hinges $\hinge pxz$ and $\hinge pyz$ in $\spc{L}$, we have
\[\mangle\hinge pxz+\mangle\hinge pyz\le\pi.\]
\end{subthm}

\end{thm}

\parit{Proof.}
Consider a point  $w\in \mathopen{]} p z \mathclose{[}$ close to $p$.
From \ref{SHORT.angle-b}, it follows that 
\[\mangle\hinge w x z+ \mangle\hinge w x{p}\le\pi\quad \text{and}\quad \mangle\hinge w y z + \mangle\hinge w y{p}\le\pi.\]

\begin{wrapfigure}{o}{30 mm}
\vskip-4mm
\centering
\includegraphics{mppics/pic-805}
\vskip4mm
\end{wrapfigure}

Since $\mangle\hinge w x y\le \mangle\hinge w x p +\mangle\hinge w y{p}$ (see \ref{claim:angle-3angle-inq}), we get 
\[\mangle\hinge w x z+ \mangle\hinge w y z +\mangle\hinge w x y
\le
2\cdot\pi.\]
Applying \ref{SHORT.angle-a}, 
\[\angk w x z
+ \angk w y z 
+\angk w x y
\le
2\cdot\pi.\]
Passing to the limits $w\to p$, we have
\[\angk p x z 
+ \angk p y z 
+\angk p x y
\le
2\cdot\pi.\]
\qedsf

\section{Equivalent conditions}

The following theorem summarizes \ref{clm:angle-mono}, \ref{clm:angle-defined}, \ref{ex:adjacent-CBB}, \ref{thm:angle-cbb}.

\begin{thm}{Theorem}\label{thm:defs_of_alex} 
Let $\spc{L}$ be a geodesic space.
Then the following conditions are equivalent.

\begin{subthm}{cbb}
$\spc{L}$ is $\CBB(0)$.
\end{subthm}
 

\begin{subthm}{2-sum} 
(adjacent angle comparison\index{comparison!adjacent angle comparison}) for any geodesic $[x y]$ and point $z\in \mathopen{]}x y\mathclose{[}$, $z\ne p$ in $\spc{L}$, we have
\[\angk z p x
+\angk z p y\le \pi.\]
\end{subthm}

\begin{subthm}{point-on-side}
(\index{comparison!point-on-side comparison}point-on-side comparison)
for any geodesic $[x y]$ and $z\in \mathopen{]}x y\mathclose{[}$ in $\spc{L}$, we have
\[\angk x p y\le\angk x p z.\]
\end{subthm}

\begin{subthm}{angle}(hinge comparison\index{comparison!hinge comparison})
\index{hinge comparison}
for any hinge $\hinge x p y$ in $\spc{L}$, the angle 
$\mangle\hinge x p y$ is defined and 
\[\mangle\hinge x p y\ge\angk x p y.\]
Moreover, 
\[\mangle\hinge z p y + \mangle\hinge z p x\le\pi\]
for any adjacent hinges $\hinge z p y$ and $\hinge z p x$.
\end{subthm}

Moreover, the implications \ref{SHORT.cbb}$\Rightarrow$\ref{SHORT.2-sum}$\Rightarrow$\ref{SHORT.point-on-side}$\Rightarrow$\ref{SHORT.angle} hold in any space, not necessarily geodesic.
\end{thm}

\begin{thm}{Advanced Exercise} \label{ex:urysohn}
Construct a geodesic space $\spc{X}\notin\CBB(0)$ that meets the following condition: for any 3 points $p,x,y\in \spc{X}$ there is a geodesic $[x y]$ such that for any $z\in \left]x y\right[$
\[\angk{z}{p}{x}+\angk{z}{p}{y}
\le
\pi.\]
\end{thm}

\section{Function comparison}

\parbf{Real-to-real functions.}
Choose $\lambda\in \RR$.
Let $s\:\II\to\RR$ be a locally Lipschitz function defined on an interval $\II$.
We say that $s$ is \index{$\lambda$-concave function}\emph{$\lambda$-concave} if $s''\le \lambda$, where the second derivative $s''$ is understood in the sense of distributions.

Equivalently, \textit{$s$ is $\lambda$-concave if the function $h\:t\mapsto s(t)-\lambda\cdot\tfrac{t^2}2$ is concave}.
Concavity can be defined via \index{Jensen inequality}\emph{Jensen inequality}; that is,
\[h(s\cdot t_0+(1-s)\cdot t_1)\ge s\cdot h(t_0)+(1-s)\cdot h(t_1)\]
for any $t_0,t_1\in \II$ and $s\in[0,1]$.
It could be also defined via the existence of (local) upper \index{supporting function}\emph{support} at any point:
\textit{for any $t_0\in \II$ there is a linear function $\ell$ that (locally) supports $h$ at $t_0$ from above;
that is, $\ell(t_0)\z= h(t_0)$ and $\ell(t)\ge h(t)$ for any $t$ (in a neighborhood of $t_0$)}.

The equivalence of these definitions is assumed to be known.
We will also use that \textit{$\lambda$-concave functions are one-side differentiable}.

\parbf{Functions on metric space.}
A function on a metric space $\spc{L}$ will usually mean a \textit{locally Lipschitz real-valued function defined in an open subset of $\spc{L}$}.
The domain of definition of a function $f$ will be denoted by $\Dom f$.

Let $f$ be a function on a metric space~$\spc{L}$.
We say that $f$ is \index{$\lambda$-concave function}\emph{$\lambda$-concave} (briefly $f''\le \lambda$) if 
for any unit-speed geodesic $\gamma\:\II\z\to \Dom f$
the real-to-real function $t\mapsto f\circ\gamma(t)$ is $\lambda$-concave.

The following proposition is conceptual ---
it reformulates a global geometric condition into an infinitesimal condition on distance functions.

\begin{thm}{Proposition}\label{comp-kappa}
A geodesic space $\spc{L}$ is $\Alex{0}$ if and only if $f''\le 1$ for any function $f$ of the following type 
\[f\:x\mapsto \tfrac12\cdot\dist[2]{p}{x}{}.\] 
\end{thm} 

\parit{Proof.}
Choose a unit-speed geodesic $\gamma$ in $\spc{L}$ and two points $x=\gamma(t_0)$, $y=\gamma(t_1)$ for some $t_0<t_1$.
Consider the model triangle $\trig{\tilde p}{\tilde x}{\tilde y}\z=\modtrig(p x y)$.
Let $\tilde \gamma\:[t_0,t_1]\to\EE^2$ be the unit-speed parametrization of $[\tilde x \tilde y]$ from $\tilde x$ to $\tilde y$.

Set
\begin{align*} 
\tilde r(t)&\df\dist{\tilde p}{\tilde\gamma(t)}{},
& 
r(t)&\df\dist{p}{\gamma(t)}{}.
\end{align*}
Clearly, $\tilde r(t_0)=r(t_0)$ and $\tilde r(t_1)=r(t_1)$.
Note that the point-on-side comparison (\ref{point-on-side}) is equivalent to 
\[t_0\le t\le t_1
\qquad\Longrightarrow\qquad
\tilde r(t)\le r(t)
\eqlbl{eq:r=<r}\]
for any $\gamma$ and $t_0<t_1$.

Set
\begin{align*} 
\tilde h(t)&=\tfrac12\cdot \tilde r^2(t) - \tfrac12\cdot t^2,
&
h&=\tfrac12\cdot r^2(t) - \tfrac12\cdot t^2.
\end{align*}
Note that $\tilde h$ is linear,
$\tilde h(t_0)=h(t_0)$ and $\tilde h(t_1)=h(t_1)$.
Observe that the Jensen inequality for the function $h$ is equivalent to \ref{eq:r=<r}.
Hence the proposition follows.
\qeds


\section{Comments}

All the discussed statements admit natural generalizations to $\CBB(\kappa)$ spaces.
Most of the time the proof is the same with uglier formulas.

For example, the function comparison of $\CBB(-1)$ states that 
$f''\z\le f$ for any function of the type $f=\cosh\circ\distfun_p$.
Similarly, the function comparison of $\CBB(1)$ states that for any point $p$, we have
$f''\le -f$ for the function $f=-\cos\circ\distfun_p$
defined in $\oBall(p,\pi)$.
The meaning of these inequalities is the same --- distance functions in $\CBB(\kappa)$ are more concave than distance functions in $\MM(\kappa)$.
The inequality $f''\le \phi$ means that for any point $p$ in the domain of definition and any $\eps>0$, there is a neighborhood $U\ni p$ such that $f''\le \phi(p)+\eps$ in $U$.
Here we assume that $f$ and $\phi$ are continuous and defined in open set.
