\chapter{Majorization}\label{sec:resh-kirz}

\section{Formulation}

\begin{thm}{Definition}\label{def:majorize}
Let $\spc{X}$ be a metric space,
$\tilde \alpha$ be a simple closed curve of finite length  in $\EE^2$,
and $D\subset\EE^2$ be a closed region bounded by $\tilde \alpha$.
A length-nonincreasing map $F\:D\to\spc{X}$ is called \index{majorizing map}\emph{majorizing} if it is length-preserving on $\tilde \alpha$.

In this case, we say that $D$ \emph{majorizes} the curve $\alpha=F\circ\tilde \alpha$ under the map $F$.
\end{thm}

The following proposition is a consequence of the definition.

\begin{thm}{Proposition}
\label{prop:majorize-geodesic} 
Let  $\alpha$  be a closed curve in a metric space $\spc{X}$.
Suppose $D\subset\EE^2$ majorizes $\alpha$ under $F\: D \to \spc{X}$.  
Then any geodesic subarc of $\alpha$ is the image under $F$ of a subarc of $\partial_{\EE^2} D$ that is geodesic in the length metric of $D$.

In particular, if $D$ is convex, then the corresponding subarc is a geodesic in $\EE^2$.
\end{thm}

\parit{Proof.} For a geodesic subarc $\gamma\:[a,b]\to\spc{X}$ of $\alpha=F\circ\tilde \alpha$, set
\begin{align*}
\tilde r&=\dist{\tilde \gamma(a)}{\tilde \gamma(b)}{D},
&
\tilde \gamma &= (F|_{\partial D})^{-1}\circ\gamma,
\\
s&=\length \gamma,
&
\tilde s&= \length \tilde \gamma.
\end{align*}
Then
\[\tilde r\ge r = s =\tilde s\ge\tilde r.\]
Therefore $\tilde s=\tilde r$.
\qeds

\begin{thm}{Corollary}\label{cor:maj-triangle}
 Assume a convex region $D\subset \EE^2$ majorizes $\trig p x y$.
Then $D$ is a \emph{solid model triangle} of $\trig p x y$;
that is,
$D=\Conv\trig{\tilde p}{\tilde x}{\tilde y}$ for a model triangle $\trig{\tilde p}{\tilde x}{\tilde y}=\modtrig(p x y)$.
Moreover, the majorizing map sends  $\tilde p$, $\tilde x$ and $\tilde y$ respectively to $p$, $x$ and $y$.
\end{thm}

Now we come to the main theorem of this section.

\begin{thm}{Majorization theorem}\label{thm:majorization}
\label{thm:major}
Any closed rectifiable curve $\alpha$ in a geodesic $\CAT(0)$ space is majorized by a convex plane figure. \end{thm}

\section{Triangles}

The case when $\alpha$ is a triangle, say $\trig p x y$, is the base in the following proof, and it is nontrivial.
In this case, by Corollary~\ref{cor:maj-triangle}, the majorizing convex region the solid model triangle. 

\begin{thm}{Line-of-sight map} \label{def:sight}
Let $p$ be a point and $\alpha$ be a curve of finite length in a geodesic space~$\spc{X}$. 
Let $\mathring\alpha:[0,1]\to\spc{U}$ be the constant-speed parametrization of~$\alpha$.  
If   $\gamma_t\:[0,1]\to\spc{U}$ is a geodesic path from $p$ to $\mathring\alpha(t)$, we say 
\[
[0,1]\times[0,1]\to\spc{U}\:(t,s)\mapsto\gamma_t(s)
\]
is a \index{line-of-sight map}\emph{line-of-sight map from $p$ to $\alpha$}.  
\end{thm}

We will show that there is a majorizing map for $\trig p x y$ whose image $W$ is the image of the line-of-sight map for $[x y]$ from  $p$,
but as one can see from the following example, the line-of-sight map is not majorizing in general.



\begin{wrapfigure}{r}{30 mm}
\vskip-0mm
\centering
\includegraphics{mppics/pic-951}
\end{wrapfigure}

\parbf{Example.} Let $\spc{Q}$ be a solid quadrangle $[p x z y]$ in $\EE^2$ formed by two congruent triangles, which is non-convex at $z$ (as in the picture).  
Equip $\spc{Q}$ with the length metric. 
Then $\spc{Q}$ is $\CAT(0)$
by Reshetnyak gluing  (\ref{thm:gluing}). 
For triangle ${\trig p x y}_\spc{Q}$ in $\spc{Q}$ and its model triangle $\trig{\tilde p}{\tilde x}{\tilde y}$ in $\EE^2$,  
we have 
\[\dist{\tilde x}{\tilde y}{}=\dist{x}{y}{\spc{Q}}=\dist{x}{z}{}+\dist{z}{y}{}.\]
Then the map $F$ defined by matching line-of-sight parameters satisfies $F(\tilde x)=x$ and $\dist{x}{F(\tilde w)}{}>\dist{\tilde x}{\tilde w}{}$ if $\tilde w$ is near the midpoint $\tilde z$ of $[\tilde x\tilde y]$ and lies on $[\tilde p\tilde z]$. 
Indeed, for $\eps=1-s$ we have
\[\dist{\tilde x}{\tilde w}{}
=\dist{\tilde x}{\tilde \gamma_\frac12(s)}{}
=\dist{x}{z}{}+o(\eps)\] and 
\[\dist{x}{F(\tilde w)}{}
=\dist{x}{\gamma_\frac12(s)}{}
=\dist{x}{z}{}-\eps\cdot\cos\mangle\hinge z p x+o(\eps).\]  
Thus $F$ is not majorizing.

\begin{wrapfigure}{r}{38 mm}
\vskip-0mm
\centering
\includegraphics{mppics/pic-811}
\end{wrapfigure}

\begin{thm}{Definition}\label{def:convex-devel}
Let $\tilde \gamma\:\II\to\EE^2$ be a curve and $\tilde p\in\EE^2$ be such that the direction of $[\tilde p\,\tilde \gamma(t)]$ turns monotonically as $t$ grows.

The set formed by all geodesics from  $\tilde p$ to the points on $\tilde \gamma$ is called the \index{development!subgraph/supergraph} \emph{subgraph} of $\tilde \gamma$ with respect to $\tilde p$.

The set of all points $\tilde x\in\EE^2$ such that a geodesic $[\tilde p\tilde x]$ intersects $\tilde \gamma$ is called the \emph{supergraph} of $\tilde \gamma$ with respect to $\tilde p$.

The curve $\tilde \gamma$ is called \index{convex/concave curve with respect to a point}\emph{convex (concave) with respect to} $\tilde p$ if the subgraph (supergraph) of $\tilde \gamma$ with respect to $\tilde p$ is convex.

The curve $\tilde \gamma$ is called 
\emph{locally convex (concave) with respect to $\tilde p$} 
if for any interior value $t_0$ in $\II$ there is a subsegment $(a,b)\subset\II$, $(a,b)\z\ni t_0$, such that the restriction $\tilde \gamma|_{(a,b)}$ is convex (concave) with respect to~$\tilde p$.
\end{thm}

Our first lemma gives a model space construction based on repeated application of the argument in the proof of the inheritance lemma (\ref{lem:inherit-angle}).

\begin{thm}{Lemma}\label{lem:majorize-subgraph}
In $\EE^2$, let  
$\beta$ be a curve from $x$ to $y$ 
that is concave with respect  to $p$.
Let $D$  be the subgraph of $\beta$ with respect to $p$.
\begin{subthm}{curvilinear} 
Then $\beta$ forms a geodesic $[x y]_D$ in $D$ and therefore $\beta$, $[p x]$ and $[p y]$ form a triangle 
${\trig p x y}_D$ in the length metric of $D$.
\end{subthm}
\begin{subthm} {short-to-subgraph}
Let $\trig{\tilde p}{\tilde x}{\tilde y}$ be the model triangle for 
${\trig p x y}_D$.
Then there is a short map 
\[G\:\Conv\trig{\tilde p}{\tilde x}{\tilde y}\to D\]
such that $\tilde p\mapsto p$, $\tilde x\mapsto x$, $\tilde y\mapsto y$, and $G$ is length-preserving on each side of $\trig{\tilde p}{\tilde x}{\tilde y}$.
In particular, $\Conv\trig{\tilde p}{\tilde x}{\tilde y}$ majorizes triangle $[p x y]_D$ in $D$ under~$G$.
\end{subthm}
\end{thm} 


\parit{Proof.}
We prove the lemma for a polygonal line $\beta$;
the general case then follows by approximation.
Namely, since $\beta$ is concave 
it can be approximated by polygonal lines that are concave with respect to $p$, 
with their lengths converging to $\length \beta$. 
Passing to a partial limit we will obtain the needed map $G$.  

Suppose $\beta=x^0x^1\dots x^n$ is a polygonal line with $x^0=x$ and $x^n=y$.
Consider a sequence of polygonal lines $\beta_i=x^0x^1\dots x^{i-1}y_i$ such that $\dist{p}{y_i}{}=\dist{p}{y}{}$ and 
$\beta_i$ has same length as $\beta$; 
that is, 
\[\dist{x^{i-1}}{y_i}{}=\dist{x^{i-1}}{x^{i}}{}+\dist{x^{i}}{x^{i+1}}{}+\dots+\dist{x^{n-1}}{x^n}{}.\]

\begin{figure}[!ht]
\vskip-0mm
\centering
\includegraphics{mppics/pic-955}
\end{figure}

Clearly $\beta_n=\beta$.
Sequentially applying Alexandrov's lemma (\ref{lem:alex}) shows that each of the polygonal lines $\beta_{n-1}, \beta_{n-2},\dots,\beta_1$ is concave with respect to $p$.
Let $D_i$ be the subgraph of $\beta_i$ with respect to $p$.
Applying the argument in the inheritance lemma (\ref{lem:inherit-angle}) gives a short map $G_i\:D_{i}\to D_{i+1}$ that maps $y_{i}\mapsto y_{i+1}$ and does not move $p$ and $x$ (in fact,  $G_i$ is the identity everywhere except on $\Conv\trig{p}{x^{i-1}}{y_i}$).
Thus the composition 
\[G_{n-1}\circ\dots\circ G_1\: D_1\to D_n\] 
is short.
The result follows since $D_1\iso\Conv\trig{\tilde p}{\tilde x}{\tilde y}$.
\qeds

\begin{thm}{Lemma}\label{lem:devel}\label{def:devel}
Let $\spc{X}$ be a metric space, 
$\gamma\:\II\to \spc{X}$ be a $1$-Lipschitz curve,
$p\in \spc{X}$,
and $\tilde p\in\EE^2$.
Then there exists a unique up to rotation curve
$\tilde \gamma\: \II\to \EE^2$, parametrized by arc-length, 
such that
$\dist{\tilde p}{\tilde \gamma(t)}{}\z=\dist{p}{\gamma(t)}{}$ for all $t$
and the direction of
$[\tilde p\tilde \gamma(t)]$ monotonically turns around $\tilde p$ counterclockwise as $t$ increases.
\end{thm}

If $p$, $\tilde p$, $\gamma$, and $\tilde \gamma$ are as above,
then $\tilde \gamma$ is called the \index{development}\emph{development} of $\gamma$ with respect to $p$; 
the point $\tilde p$ is called the \index{development!basepoint of a development}\emph{basepoint} of the development.

\parit{Proof.}
Consider the functions $\rho$, $\theta\:\II\to\RR$ defined as 
\begin{align*}
\rho(t)
&=\dist{p}{\gamma(t)}{},
&
\theta(t)
&=
\int\limits_{t_0}^{t}\frac{\sqrt{1-(\rho')^2}}{\rho},
\end{align*}
where $t_0\in\II$ is a fixed number and $\int$ denotes Lebesgue integral.
Since $\gamma$ is $1$-Lipshitz, so is $\rho(t)$, and thus the function $\theta$ is defined and nondecreasing.

It is straightforward to check that $(\rho,\theta)$ uniquely describe $\tilde \gamma$ in polar coordinates on $\EE^2$ with center at $\tilde p$.
\qeds

\begin{thm}{Exercise}\label{ex:devel-comp-CAT}
A geodesic space $\spc{U}$ is $\CAT(0)$ if and only if development of any geodesic with respect to any point is concave. 
\end{thm}



\begin{thm}{Lemma}\label{lem:majorize-triangle}
Let $\trig{p}{x}{y}$ be a triangle in a geodesic $\CAT(0)$ space $\spc{U}$.
In $\EE^2$, let $\tilde \gamma$ be the $\kappa$-development of $[x y]$ with respect to $p$, where $\tilde \gamma$ has basepoint $\tilde p$ and subgraph $D$.
Consider the map $H\:D\to\spc{U}$ that sends the point with parameter $(t,s)$ under the line-of-sight map for $\tilde \gamma$ with respect to $\tilde p$, to the point with the same parameter under the line-of-sight map $f$ for $[x y]$ with respect to $p$.
Then $H$ is  length-nonincreasing.
In particular, $D$ majorizes triangle $\trig p x y$.
\end{thm}

\parit{Proof.}
Let $\gamma\:[0,T]\to \spc{U}$ be a unit-speed paremetrization of $[xy]$; so, $T=\dist{x}{y}{}$. 
Choose a partition 
\[0=t^0<t^1<\dots<t^n=T,\]
and set $x^i=\gamma(t^i)$. 
Construct a chain of model triangles  $\trig{\tilde p}{\tilde x^{i-1}}{\tilde x^i}\z=
\modtrig(p x^{i-1} {x^i})$, with $\tilde x^0=\tilde x$ and the direction of $[\tilde p\tilde x^i]$ turning counterclockwise as $i$ grows.  
Let $D_n$ be the subgraph with respect to $\tilde p$ of the polygonal line $\tilde x^0\dots \tilde x^n$.


Let  $\delta_n$ be the maximum radius of a circle inscribed in any of the triangles $\trig{\tilde p}{\tilde x^{i-1}}{\tilde x^i}$.  

Now we construct a map $H_n \: D_n\to\spc{U}$  that increases distances by at most  $2\cdot\delta_n$.
Suppose $w\in D_n$.
Then $w$ lies on or inside some triangle $\trig{\tilde p}{\tilde x^{i-1}}{\tilde x^i}$.  
Define $H_n(w)$ by first mapping $w$ to a nearest point on $\trig{\tilde p}{\tilde x^{i-1}}{\tilde x^i}$ (choosing one if there are several), followed by the natural map to the triangle  $\trig {p}{x^{i-1}}{ x^i}$. 

Since triangles in $\spc{U}$ are thin, the restriction of $H_n$ to each triangle $\trig{\tilde p}{\tilde x^{i-1}}{\tilde x^i}$ is short.   
Then the triangle inequality implies that the restriction of $H_n$ to 
\[U_n=\bigcup_{1\le i\le n}\trig{\tilde p}{\tilde x^{i-1}}{\tilde x^i}\]
is short with respect to the length metric on $D_n$. 
Since nearest-point projection from $D_n$ to $U_n$ increases the $D_n$-distance between two points by at most $2\cdot\delta_n$, the map $H_n$ also increases the $D_n$-distance by at most $2\cdot\delta_n$. 

Consider converging sequences $v_n\to v$ and $w_n\to w$ such that $v_n,w_n\in D_n$ and therefore $v,w\in D$.
Note that 
\[\dist{H_n(v_n)}{H_n(w_n)}{} \le \dist{v_n}{w_n}{D_n} + 2\cdot\delta_n,\eqlbl{eq:|H(v)-H(w)|}\]
for each $n$.
Since $\delta_n\to 0$ and geodesics in $\spc{U}$ vary continuously with their endpoints (\ref{thm:alex-patch}), we have $H_n(v_n)\to 
H(v)$ and $H_n(w_n)\to H(w)$.
Therefore the left-hand side in \ref{eq:|H(v)-H(w)|} converges to $\dist{H(v)}{H(w)}{}$ and the right-hand side converges to $\dist{v}{w}{D}$, it follows that $H$ is short.
\qeds

\parit{Proof of \ref{thm:major} for triangles.}
Suppose $\alpha$ is a triangle, say $\trig p x y$.

Let $\tilde \gamma$ be the development of $[x y]$ with respect to $p$, where $\tilde \gamma$ has basepoint $\tilde p$ and subgraph $D$.
By \ref{ex:devel-comp-CAT}, $\tilde \gamma$ is concave.
By \ref{lem:majorize-subgraph},  there is a short map $G\:\Conv\modtrig(p x y)\to D$.
Further, by \ref{lem:majorize-triangle},  $D$ majorizes $\trig p x y$ under a majorizing map $H\:D\to\spc{U}$.
Clearly $H\circ G$ is a majorizing map for $\trig p x y$.
\qeds

\section{Polygons}

In  the following proofs, $x^1 \dots x^n$ ($n\ge 3$) denotes a polygonal line $x^1,\dots,x^n$, and $[x^1\dots x^n ]$ denotes the corresponding (closed) polygon.
For a subset $R$ of the ambient metric space,
we denote by $[x^1\dots x^n ]_R$ a polygon in the length metric of $R$.

\parit{Proof of \ref{thm:major} for polygons.}
We begin by proving the theorem in case $\alpha$ is polygonal.

\begin{wrapfigure}{o}{40 mm}
\vskip-1mm
\centering
\includegraphics{mppics/pic-960}
\vskip0mm
\end{wrapfigure}

Now we claim that any closed $n$-gon $[x^1x^2 \dots x^n ]$ in a $\CAT(0)$ space  is majorized by a convex polygonal region \[R_n=\Conv[\tilde x^1\tilde x^2\dots\tilde x^n]\]
under a map $F_n$ such that $F_n\:\tilde x^i\mapsto x^i$ for each $i$. 

The base case $n=3$ is proved above.
Assume the statement is true for $(n-1)$-gons, $n\ge 4$.  
Then  $[x^1 x^2 \dots x^{n-1}]$  is majorized by a convex polygonal region 
\[R_{n-1}=\Conv[\tilde x^1 \tilde x^2,\dots \tilde x^{n-1}],\] 
in $\EE^2$ under a map $F_{n-1}$ satisfying $F_{n-1}(\tilde x^i)=x^i$ for all $i$. 
Take $\dot x^n\in\EE^2$ such that $\trig{\tilde x^1}{\tilde x^{n-1}}{\dot x^n}=\modtrig(x^1 x^{n-1} x^n)$ 
and this triangle lies on the other side of $[\tilde x^1\tilde x^{n-1}]$ from $R_{n-1}$.  
Let $\dot R\z=\Conv\trig{\tilde x^1}{\tilde x^{n-1}}{\dot x^n}$, 
and $\dot F\:\dot R\to \spc{U}$ be a majorizing map for $\trig { x^1}{x^{n-1}}{ x^n}$ as provided above.

Set 
$R= R_{n-1}\cup \dot R$, where $R$ carries its length metric.
Since $F_n$ and $F$ agree on $[\tilde x^1 \tilde x^{n-1}]$, we may define $F\:R\to\spc{U}$ by 
\[
F(x)=
\begin{cases}
F_{n-1}(x),\quad & x\in R_{n-1},\\
\dot F(x),\quad & x\in \dot R.\\
\end{cases}
\]
Then $F$ is length-nonincreasing and is a majorizing map for $[x^1 x^2 \dots x^n ]$ (as in Definition~\ref{def:majorize}).

If $R$ is a convex subset of $\EE^2$, we are done. 

If $R$ is not convex,  the total internal angle of $R$ at $\tilde x^1$ or $ \tilde x^{n-1} $ or both is $>\pi$.  
By relabeling we may suppose this holds for $\tilde x^{n-1}$.  

The region $R$ is obtained by gluing $R_{n-1}$ to $\dot R$ by $[x^1x^{n-1}]$.
Thus, by Reshetnyak gluing (\ref{thm:gluing}), $R$ carrying its length metric is a $\CAT(0)$-space.  
Moreover $[\tilde x^{n-2}\tilde x^{n-1}]\cup[\tilde x^{n-1} \dot x^n]$ is a geodesic of $R$.
Thus $[\tilde x^1 \tilde x^2 \dots \tilde x^{n-2} \dot x^n]_R$ is a closed $(n-1)$-gon in $R$, to which the induction hypothesis applies. The resulting short map from a convex region in $ \EE^2$ to~$R$, followed by $F$,  is the desired majorizing map.
\qeds

If $p_1\dots p_n$ is a polygon, then values $\theta_i=\pi-\mangle\hinge{p_i}{p_{i-1}}{p_{i+1}}$ for all $i\pmod n$ are called \emph{external angles} of the polygon.
The following exercise is a generalization of Fenchel's theorem.

\begin{thm}{Exercise}\label{ex:fenchel}
Show that the sum of external angles of any polygon in a complete length $\CAT(0)$ space cannot be smaller than $2\cdot\pi$. 
\end{thm}

The following exercise is a version of the Fáry--Milnor theorem for $\CAT(0)$ spaces.

\begin{thm}{Very advanced exercise}\label{ex:FM}
Suppose that a simple polygon $\beta$ in a complete length $\CAT(0)$ space does not bound an embedded disc.
Show that the sum of external angles of $\beta$ cannot be smaller than $4\cdot\pi$.

Give an example of such a polygon $\beta$ with the sum of external angles exactly $4\cdot\pi$.
\end{thm}

\begin{thm}{Exercise}\label{ex:arm-lemma}
Prove the following generalization of the arm lemma (\ref{lem:arm}).
\end{thm}

  
\begin{thm}{Arm lemma}\label{lem:arm+}
Let $P=[x^0x^1\dots x^{n+1}]$ be a polygon in a geodesic $\CAT(0)$ space $\spc{U}$.
Suppose $\tilde P=[\tilde x^0\tilde x^1\dots \tilde x^{n+1}]$ is a convex  polygon in $\EE^2$
such that 
\[
\dist{\tilde x^i}{\tilde x^{i-1}}{\EE^2}
=
\dist{x^i}{x^{i-1}}{\spc{U}}
\quad \text{and}\quad 
\mangle\hinge{x^i}{x^{i-1}}{x^{i+1}}\ge\mangle\hinge{\tilde x^i}{\tilde x^{i-1}}{\tilde x^{i+1}}
\eqlbl{eq:arm}
\]
for all $i$.
Then 
$\dist{\tilde x^0}{\tilde x^{n+1}}{\EE^2}
\le
\dist{x^0}{x^{n+1}}{\spc{U}}$.
\end{thm}

\section{General case}

If the space is proper, then the general case follows applying polygonal case to inscribed polygonal lines and passing to the limit.
The statement holds for any geodesic $\CAT(0)$ space but one need to be more careful \cite{alexander-kapovitch-petrunin-2025}.

The following exercise is the rigidity case 
of the majorization theorem.

{\sloppy 

\begin{thm}{Exercise}\label{ex:isometric-majorization}
Let $\spc{U}$ be a geodesic $\CAT(0)$ space
and $\alpha\:[0,\ell]\to\spc{U}$ be a closed curve with arclength parametrization.
Assume there is a closed convex curve $\tilde \alpha\:[0,\ell]\to\EE^2$ such that 
\[\dist{\alpha(t_0)}{\alpha(t_1)}{\spc{U}}=\dist{\tilde \alpha(t_0)}{\tilde \alpha(t_1)}{\EE^2}\]
for any $t_0$ and $t_1$.
Show that there is a distance-preserving map $F\:\Conv\tilde \alpha\to \spc{U}$
such that $F\:\tilde \alpha(t)\mapsto \alpha(t)$ for any $t$.
\end{thm}

}

\begin{thm}{Exercise}\label{ex:bishop}
Two majorizations $F\:D\to \spc{U}$ and $F'\:D'\to \spc{U}$ will be called \index{majorizing map!equivalent majorizations}\emph{equivalent} if $F'=F\circ\iota$ for an isometry $\iota\:D\to D'$.

Show that a closed rectifiable curve in a $\CAT(0)$ space has an isometric majorization map if and only if the majorization map is unique up to equivalence.
\end{thm}


\section{Comments}

The statements in this section can be generalized to $\CAT(\pm1)$ spaces;
in the $\CAT(1)$ case one has to assume that the closed curve has length at most $2\cdot\pi$.

The majorization theorem was proved by Yuriy Reshetnyak \cite{reshetnyak:major};
our proof uses a trick that we learned from the lectures of Werner Ballmann \cite{ballmann:lectures}.
Another proof can be built on generalized Kirszbraun's theorem, but it works only for complete spaces.

The definition of develpoment appears in \cite{alexandrov:devel}
and an earlier form of it can be found in \cite{liberman}.

\begin{thm}{Open problem}
Let $\alpha$ be a closed rectifiable curve in a $\CAT(0)$ space $\spc{U}$.
Note that if $\alpha$ is a geodesic triangle or it bounds an isometric copy of convex plane figure in $\spc{U}$, then $\alpha$ has a unique (up to congruence) majorizing convex figure.

What about the converse?
\end{thm}
