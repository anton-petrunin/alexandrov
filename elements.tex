\chapter{Nonnegative curvature: definition}

\section{Distances and geodesics}

\parbf{Distances.}
The distance between two points $x$ and $y$ in a metric space $\spc{X}$ will be denoted by $\dist{x}{y}{}$ or $\dist{x}{y}{\spc{X}}$.
The latter notation is used if we need to emphasize 
that the distance is taken in the space~${\spc{X}}$.
The function $(x,y)\mapsto \dist{x}{y}{\spc{X}}$ is called \emph{metric};
it has to meet the following conditions for any three points $x,y,z\in \spc{X}$:

\begin{subthm}{metric>=0}
$\dist{x}{y}{\spc{X}}\ge 0$,
\end{subthm}

\begin{subthm}{metric=0} $\dist{x}{y}{\spc{X}}= 0$ $\iff$ $x=y$,
\end{subthm}

\begin{subthm}{metric:sym} $\dist{x}{y}{\spc{X}}=\dist{y}{x}{\spc{X}}$,
\end{subthm}

\begin{subthm}{metric:triangle} $\dist{x}{y}{\spc{X}}+\dist{y}{z}{\spc{X}}\ge\dist{x}{z}{\spc{X}}$.
\end{subthm}

\parbf{Geodesics.}
Let $\II$\index{$\II$} be a real interval. 
A distance-preserving map $\gamma$ from $\II$ to a metric space $\spc{X}$ is called a \index{geodesic}\emph{geodesic}%
\footnote{Others call it differently: \textit{shortest path}, \textit{minimizing geodesic}.
Also, note that the meaning of the term \textit{geodesic} is different from what is used in Riemannian geometry, altho they are closely related.}; 
in other words, $\gamma\:\II\to \spc{X}$ is a geodesic if 
\[\dist{\gamma(s)}{\gamma(t)}{\spc{X}}=|s-t|\]
for any pair $s,t\in \II$.

If $\gamma\:[a,b]\to \spc{X}$ is a geodesic such that $p=\gamma(a)$, $q=\gamma(b)$, then we say that $\gamma$ is a geodesic from $p$ to $q$.
In this case, the image of $\gamma$ is denoted by $[p q]$\index{$[{*}{*}]$}, and, with abuse of notations, we also call it a \index{geodesic}\emph{geodesic}.
We may write $[p q]_{\spc{X}}$ 
to emphasize that the geodesic $[p q]$ is in the space  ${\spc{X}}$.

In general, a geodesic from $p$ to $q$ need not exist and if it exists, it need not  be unique.  
However, once we write $[p q]$ we assume that we have chosen such geodesic.

\parbf{Geodesic path.}
A \index{geodesic path}\emph{geodesic path} is a geodesic with constant-speed parameterization by the unit interval $[0,1]$.

\parbf{Geodesic space.}
A metric space is called \index{geodesic space}\emph{geodesic} if any pair of its points can be joined by a geodesic.

\section{Triangles, hinges, and angles}

\parbf{Triangles.}
Given a triple of points $p,q,r$ in a metric space $\spc{X}$, a choice of geodesics $([q r], [r p], [p q])$ will be called a \index{triangle}\emph{triangle}; we will use the short notation 
$\trig p q r=\trig p q r_{\spc{X}}=([q r], [r p], [p q])$\index{$\trig {{*}}{{*}}{{*}}$}.

Given a triple $p,q,r\in \spc{X}$ there may be no triangle 
$\trig p q r$ simply because one of the pairs of these points cannot be joined by a geodesic.
Also, many different triangles with these vertices may exist, any of which can be denoted by $\trig p q r$.
If we write $\trig p q r$, it means that we have chosen such a triangle.

\parbf{Model triangles.}
Given three points $p,q,r$ in a metric space $\spc{X}$,
let us define its \index{model triangle}\emph{model triangle} $\trig{\tilde p}{\tilde q}{\tilde r}$ 
(briefly, 
$\trig{\tilde p}{\tilde q}{\tilde r}=\modtrig(p q r)_{\EE^2}$%
\index{$\modtrig$!$\modtrig({*}{*}{*})_{\EE^2}$}) to be a triangle in the Euclidean plane $\EE^2$ such that
\begin{align*}\dist{\tilde p}{\tilde q}{\EE^2}&=\dist{p}{q}{\spc{X}},
&
\quad\dist{\tilde q}{\tilde r}{\EE^2}&=\dist{q}{r}{\spc{X}},
&
\quad\dist{\tilde r}{\tilde p}{\EE^2}&=\dist{r}{p}{\spc{X}}.
\end{align*}

The same way we can define the \index{hyperbolic model triangle}\emph{hyperbolic} and the \index{spherical model triangles}\emph{spherical model triangles} $\modtrig(p q r)_{\HH^2}$, $\modtrig(p q r)_{\SSS^2}$
in the Lobachevsky plane $\HH^2$ and the unit sphere~$\SSS^2$.
In the latter case, the model triangle is said to be defined if in addition
\[\dist{p}{q}{}+\dist{q}{r}{}+\dist{r}{p}{}< 2\cdot\pi.\]
In this case, the model triangle again exists and is unique up to an isometry of~$\SSS^2$.

\parbf{Model angles.}
If 
$\trig{\tilde p}{\tilde q}{\tilde r}=\modtrig(p q r)_{\EE^2}$ 
and $\dist{p}{q}{},\dist{p}{r}{}>0$, 
the angle measure of 
$\trig{\tilde p}{\tilde q}{\tilde r}$ at $\tilde p$ 
will be called the \index{model angle}\emph{model angle} of the triple $p$, $q$, $r$ and will be denoted by
$\angk p q r_{\EE^2}$%
\index{$\tilde\measuredangle$!$\angk{{*}}{{*}}{{*}}$}.
The same way we define $\angk p q r_{\HH^2}$ and $\angk p q r_{\SSS^2}$;
in the latter case  we assume in addition that the model triangle $\modtrig(p q r)_{\SSS^2}$ is defined.

We may use the notation $\angk p q r$ if it is evident which of the model spaces $\HH^2$, $\EE^2$ or $\SSS^2$ is meant.

\parbf{Hinges.} Let $p,x,y\in \spc{X}$ be a triple of points such that $p$ is distinct from $x$ and~$y$.
A pair of geodesics $([p x],[p y])$ will be called  a \index{hinge}\emph{hinge} and will be denoted by 
$\hinge p x y=([p x],[p y])$\index{$\hinge{{*}}{{*}}{{*}}$}.

\section{Baby Toponogov}

Recall that \emph{polyhedral space} is a geodesic space that admits a finite triangulation such that each simplex is isometric to a simplex in a Euclidean space.
If, in addition, it is homeomorphic to a surface (without boundary), then it is called a \emph{polyhedral surface}.
A point on a polyhedral surface with nonzero curvature is called an \emph{essential vertex}.
Any other point on the surface will be called \emph{regular}.
Note that \textit{any regular point has a neighborhood that is isometric to an open set in the Euclidean plane}.

\begin{thm}{Exercise}\label{ex:poly+geod}
Let $P$ be a non-negatively curved polyhedral surface.

\begin{subthm}{}
Show that a geodesic in $P$ cannot pass thru an essential vertex.
\end{subthm}

\begin{subthm}{}
Show that if two geodesics in $P$ intersect at two points, 
then these are the endpoints for both geodesics.
\end{subthm}

\end{thm}

The next theorem gives a global geometric property of 
non-negatively curved polyhedral surface.

Given a hinge $\hinge pxy$ in a polyhedral surface $P$, denote by $\mangle\hinge pxy$ the minimal angle that the hinge cuts from $P$ at $p$.
(Soon we will give a more general definition of $\mangle\hinge pxy$.)

\begin{thm}{Theorem}\label{thm:poly-cbb}
Let $P$ be a polyhedral surface.
Assume $P$ has non-negative curvature at each point (see \ref{sec:Alexandrov-existence}).
Then 
\[\mangle\hinge pxy\ge\angk pxy\]
for any hinge $\hinge pxy$ in $P$.
\end{thm}

The following exercise will be used in the proof.

\begin{thm}{Exercise}\label{ex:concave-loc}
Let $f\:[0,\ell]\to\RR$ be a continuous function such that for any $t\in \left]0,\ell\right[$ there is a linear function $h$ that locally supports $f$ from above;
that is, $h(t_0)=f(t_0)$, and there is $\eps>0$ such that $h(t)\ge f(t)$ if $|t-t_0|<\eps$.
Show that $f$ is concave.
\end{thm}


\parit{Proof.}
Let $[pxy]$ be a triangle in $P$ and let $[\tilde p\tilde x\tilde y]$ be the model triangle of $[pxy]$.
Set $\ell=|x-y|_P=|\tilde x-\tilde y|_{\EE^2}$.

Denote by $\gamma(t)$ and $\tilde \gamma(t)$ the geodesics $[xy]$ and $[\tilde x\tilde y]$ parametrized by length starting from $x$ and $\tilde x$, respectively.
Observe that it is sufficient to show that 
$$| p- \gamma(t)|\le|\tilde p-\tilde \gamma(t)| 
\eqlbl{eq:comp-gamma}$$
for any $t$ in $[0,\ell]$.

We may assume that $p$ is a regular point;
otherwise, move it slightly and apply approximation.


From the cosine law, we get that the function 
$$\tilde f(t)=|\tilde p-\tilde \gamma(t)|^2-t^2$$
is linear.
Consider the function
$$f(t)=|p- \gamma(t)|^2-t^2.$$
Note that $f(0)=\tilde f(0)$, $f(\ell)=\tilde f(\ell)$, and the inequality~\ref{eq:comp-gamma} is equivalent to
$$f(t)\ge \tilde f(t).
\eqlbl{eq:comp-f}$$
By Jensen's inequality, \ref{eq:comp-f} holds if $f$ is concave.

By \ref{ex:poly+geod}, 
$\gamma(t_0)$ is regular.
Since $p$ is regular,
a geodesic $[p\gamma(t)]$ contains only regular points.
Therefore for small $\eps>0$,
 the $\eps$-neighborhood of $[p\gamma(t)]$, say $\Omega$, contains only regular points. 
We may assume that $\Omega$ is homeomorphic to a disc;
in this case, there is a locally distance preserving embedding $\iota\:\Omega\to\EE^2$.
Note the image $\iota[p\gamma(t)]$ is a line segment that 
and $\iota(\Omega)$ is the $\eps$-neighborhood of $\iota[p\gamma(t)]$ in $\EE^2$;
in particular, $\iota(\Omega)$ is convex.
Thus $\iota(\Omega)$ contains a triangle with  base $\iota[\gamma(t_0-\eps)\ \gamma(t_0+\eps)]$  and vertex $\iota(p)$.

Clearly, for any $t\in[t_0-\eps,t_0+\eps]$ 
we have 
$$|\iota(p)-\iota(\gamma(t))|\ge|p-\gamma(t)|.$$
Note that
the function
$$h(t)= |\iota(p)-\iota(\gamma(t))|^2-t^2$$
is linear.
From above, $h$ supports $f$ locally  at $t_0$.
It remains to apply~\ref{ex:concave-loc}.
\qeds

\section{Definition}

\begin{thm}{Definition}\label{def:CBB}
A metric space $\spc{X}$ has \emph{nonnegative curvature} in the sense of Alexandrov (briefly, $\spc{X}\in\CBB(0)$) if the inequality 
\[\angk  pxy_{\EE^2}+\angk pyz_{\EE^2}+\angk pzx_{\EE^2}
\le 
2\cdot\pi
\eqlbl{eq:CBB-comparison}\]
holds for any quadruple $p,x,y,z\in\spc{X}$ such that $p$ is distinct from $x$, $y$, and $z$. 

The inequality \ref{eq:CBB-comparison} is called \emph{$\CBB(0)$ comparison} for the quadruple $p,x,y,z$.
If instead of $\EE^2$, we use $\SSS^2$ or $\HH^2$, then we get the definition of
$\CBB(1)$ and $\CBB(-1)$ comparisons.
\end{thm}

While this definition can be applied to any metric space,
it is usually applied to geodesic spaces (or, at least, length spaces that will be defined later).



\begin{thm}{Exercise}\label{ex:polyCBB}
Show that a polyhedral surface is $\CBB(0)$ if and only if it has nonnegative curvature in the sense of \ref{sec:Alexandrov-existence}. 
\end{thm}

The following theorem generalizes \ref{thm:alexandrov-uni'} and \ref{thm:exist}.

\begin{thm}{Theorem}
A metric space $\spc{X}$ is isometric to the surface of a convex body in the Euclidean space if and only if $\spc{X}$ is a geodesic $\CBB(0)$ space that is homeomorphic to $\SSS^2$.

Moreover, $\spc{X}$ determines the convex body up to congruence.
\end{thm}

As before, a convex body can degenerate to a plane figure $F$;
in this case, its surface is defined as two copies of $F$ glued along the boundary.

The main part is due to Alexadr Alexandov;
its proof is an application of \ref{thm:exist} together with approximation.
The last part is very difficult; it was proved by Aleksei Pogorelov.

Eventually, we will prove the only-if part of the theorem, which is the simplest part of the theorem;
it requires only \ref{ex:sphere-with-pos} which is
the only-if part of \ref{thm:exist}.
To do this we will need to introduce the convergence of subsets in Euclidean space (Hausdorff convergence) and convergence of metric spaces (Gromov--Hausdorff convergence); it will be done in the next lecture.

\begin{thm}{Exercise}\label{ex:(3+1)-expanding}
Show that a metric space $\spc{X}$ is $\Alex0$
if for any quadruple of points $p,x_1,x_2,x_3\in \spc{X}$ 
there is a quadruple of points $q,y_1,y_2,y_3\in\EE^2$
such that 
\[\dist{p}{x_i}{\spc{X}}\ge\dist{q}{y_i}{\EE^2} 
\quad \text{and}\quad
\dist{x_i}{x_j}{\spc{X}}\le\dist{y_i}{y_j}{\EE^2}\] 
for all $i$ and $j$.
\end{thm}

\begin{thm}{Exercise}\label{ex:normCBB}
Show that $\RR^2$ with metric induced by a norm is $\Alex0$ if and only if it is isometric to the Euclidean plane $\EE^2$.
\end{thm}

\section{Comments}

Let us give a more conceptual way to think about the comparison inequality in \ref{def:CBB} and an analogous inequality for upper curvature bound that will appear later.


Consider the space $\mathcal{M}_4$ of all isometry classes of 4-point metric spaces.
Each element in $\mathcal{M}_4$ can be described by 6 numbers 
 --- the distances between all 6 pairs of its points, say $\ell_{i,j}$ for $1\le i< j\le 4$ modulo permutations of the index set $(1,2,3,4)$.
These 6 numbers are subject to 12 triangle inequalities; that is,
\[\ell_{i,j}+\ell_{j,k}\ge \ell_{i,k}\]
holds for all $i$, $j$ and $k$, where we assume that $\ell_{j,i}=\ell_{i,j}$ and $\ell_{i,i}=0$.

The space $\mathcal{M}_4$ comes with topology.
It can be defined as a quotient of the cone in $\RR^6$ by permutations of the 4-points of the space.

\begin{wrapfigure}[7]{o}{33mm}
\vskip-0mm
\centering
\includegraphics{mppics/pic-700}
\end{wrapfigure}

Consider the subset $\mathcal{E}_4\subset \mathcal{M}_4$ of all isometry classes of 4-point metric spaces that admit isometric embeddings into Euclidean space.

\begin{thm}{Advanced exercise}\label{clm:two-components-of-M4}
The complement $\mathcal{M}_4\setminus \mathcal{E}_4$ has two connected components.
\end{thm}

Let us denote one of the components by $\mathcal{P}_4$ and the other by~$\mathcal{N}_4$.
Here $\mathcal{P}$ and $\mathcal{N}$ stand for {}\emph{positive} 
and {}\emph{negative curvature} because spheres have no quadruples of type $\mathcal{N}_4$ and 
Lobachevsky space
has no quadruples of type~$\mathcal{P}_4$.

A metric space 
that has no quadruples of points of type $\mathcal{P}_4$ or $\mathcal{N}_4$
respectively 
is called $\CAT(0)$ and $\CBB(0)$.

\begin{wrapfigure}{r}{33mm}
\vskip-4mm
\centering
\includegraphics{mppics/pic-710}
\end{wrapfigure}

Let us describe the subdivision into $\mathcal{P}_4$, $\mathcal{E}_4$, and $\mathcal{N}_4$ intuitively.
Imagine that you move out of $\mathcal{E}_4$ --- your path is a one-parameter family of 4-point metric spaces.
The last thing you see in $\mathcal{E}_4$ is one of the two plane configurations shown on the diagram.
If you see the left configuration then you move into $\mathcal{N}_4$;
if it is the one on the right, then you move into $\mathcal{P}_4$.
More degenerate pictures can be avoided; for example, a triangle with a point on a side.
From such a configuration one may move in $\mathcal{N}_4$ and $\mathcal{P}_4$ (as well as come back to $\mathcal{E}_4$).

Here is an exercise, solving which would force you to rebuild a considerable part of Alexandrov geometry.
It might be helpful to spend some time thinking about this exercise before proceeding.

\begin{thm}{Advanced exercise}\label{ex:convex-set}
Assume $\spc{X}$ is a geodesic space, 
containing only quadruples of type~$\mathcal{E}_4$.
Show that $\spc{X}$ is isometric to a convex set in a Hilbert space.
\end{thm}

In the definition above, 
instead of  Euclidean space 
one can take  
Lobachevsky space of curvature~$-1$.
In this case,
one obtains the definition of spaces with curvature bounded above or below by~$-1$ ($\CAT(-1)$ or $\CBB(-1)$).

To define spaces with curvature bounded above or below by $1$ ($\CAT(1)$ or $\CBB(1)$),
one has to take the unit 3-sphere 
and specify that only the quadruples of points such that each of the four triangles has perimeter 
less than $2\cdot\pi$ are checked.
