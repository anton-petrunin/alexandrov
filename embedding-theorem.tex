\chapter{Alexandrov embedding theorem}

This lecture contains selected material from Alexandrov's book~\cite{alexandrov} which is very readable.

We prove the Cauchy theorem and then modify it to prove the Alexandrov uniqueness theorem.
Further, we sketch a proof of the Alexandrov embedding theorem which is the starting point of Alexandrov geometry.


\section{Cauchy theorem}

Further, \emph{surfaces} of convex polyhedra will be considered with \emph{intrinsic distance}; it is defined as the length of a shortest path on the surface between points.
Shortest paths parametrized by arclength are called \emph{geodesics} (this term has a slightly different meaning in Riemannian geometry is different).
 
\begin{thm}{Theorem}\label{thm:cauchy} Let $K$ and $K'$ be two non-degenerate convex polyhedra in $\RR^3$;
denote their surfaces 
by $P$ and $P'$.
Suppose there is an isometry $P\to P'$ that sends each face of $K$ to a face of $K'$.
Then $K$ is congruent to $K'$.
\end{thm}

\parit{Proof.} 
Consider the graph $\Gamma$ formed by the edges of $K$ ;
the edges of $K'$ form the same graph.
 
For an edge $e$ in $\Gamma$, denote by $\alpha_e$ and $\alpha'_e$ the corresponding dihedral angle in $K$ and $K'$ respectively.
Mark an edge $e$ of $\Gamma$ with 
$({+})$ if $\alpha_e < \alpha'_e$ and with $({-})$ if $\alpha_e > \alpha'_e$.

Now remove from $\Gamma$ everything which was not marked;
that is, leave only the edges marked by $(+)$ or $(-)$ and their endpoints.

Note that the theorem follows if $\Gamma$ is an empty graph;
assume the contrary.

Note that $\Gamma$ is embedded into $P$, which is homeomorphic to the sphere.
In particular, the edges coming from one vertex have a natural cyclical order. 
Given a vertex $v$ of $\Gamma$, we can count the \emph{number of sign changes} around $v$;
that is, the number of consequent pairs edges with different signs. 

\begin{thm}{Local lemma}\label{lem:local}
For any vertex of $\Gamma$ the number of sign changes is at least $4$.
\end{thm}

In other words, at each vertex of $\Gamma$, one can choose 4 edges marked by $(+)$, $(-)$, $(+)$, and $(-)$ in the same cyclical order.
Note that the local lemma contradicts the following.

\begin{thm}{Global lemma}\label{lem:global}
Let $\Gamma$ be a nonempty sub-graph of the graph formed by the edges of a convex polyhedron. Then it is impossible to mark all of the edges of $\Gamma$ by $(+)$ or $(-)$ 
such that the number of sign changes around each vertex of $\Gamma$ is at least $4$.
\end{thm}

It remains to prove these two lemmas.
\qeds

\begin{thm}{Exercise}\label{ex:disc-bend}
Consider two polyhedral surfaces in $\RR^3$ glued by the rule on the diagrams from regular polygons.
Assume that each forms a part of a surface of a convex polyhedron.
Use the local lemma to show the following.


\begin{subthm}{}
Show that the first configuration is rigid; that is, one can not fix the position of the pentagon and continuously move the remaining 5 vertices in a new position so that each triangle moves by a one-parameter family of isometries of $\RR^3$.
\end{subthm}

\begin{subthm}{}
Show that the second configuration has a rotational symmetry with the axis passing through the midpoint of the marked edge.
\end{subthm}

\end{thm}





\begin{figure}[ht!]
\begin{lpic}[t(-12mm),b(-5mm),r(0mm),l(-3mm)]{pics/penta(0.15)}
\end{lpic}
\ 
\begin{lpic}[t(-10mm),b(-15mm),r(0mm),l(0mm)]{pics/cauchy-zalgaller(0.15)}
\end{lpic}
\end{figure}


\section{Arm lemma and local lemma}

To prove the local lemma, we will need the following.

\begin{thm}{Arm lemma}\label{lem:arm}
Assume that $A=[a_0 a_1\dots a_n]$ is a convex polygon in $\RR^2$
and $A'=[a'_0 a'_1\dots a'_n]$ be a closed broken line in $\RR^3$
such that 
$$|a_i-a_{i+1}|=|a'_i-a'_{i+1}|$$ for any $i\in\{0,\dots,n-1\}$
and 
$$\measuredangle a_i\le \measuredangle a'_i$$ 
for each $i\in\{1,\dots,n-1\}$.
Then 
$$|a_0-a_n|\le |a'_0-a'_n|$$
and equality holds if and only if $A$ is congruent to $A'$.
\end{thm}

One may view the broken lines $[a_0a_1\dots a_n]$ and $[a'_0a'_1\dots a'_n]$ as a robot's arm in two positions.
The arm lemma states that when the arm opens, 
the distance between the ``shoulder'' and ``tips of the fingers'' increases. 

\begin{thm}{Exercise}
Show that the arm lemma does not hold if 
instead of the convexity,
one only the local convexity;
that is, if you go along the broken line $a_0 a_1\dots a_n$, then you only turn left.
\end{thm}

\begin{thm}{Exercise}\label{ex:cauchy}
Suppose $A=[a_1\dots a_n]$ and $A'=[a'_1\dots a'_n]$ be noncongruent convex plane polygons with equal corresponding sides.
Mark each vertex $a_i$ with plus (minus) if the interior angle of $A$ at $a_i$ is smaller (respectively bigger) than the interior angle of $A'$ at $a_i'$.
Show that there are at least 4 sign changes around $A$.

Give an example showing the statement does not hold without assuming convexity.

\end{thm}


In the proof, we will use the following exercise which is the triangle inequality angles. 

\begin{thm}{Exercise}\label{ex:angle-triangle}
Let $w_1,w_2,w_3$ be unit vectors in $\RR^3$.
Denote by $\theta_{i,j}$ the angle between the vectors $v_i$ and $v_j$.
Then 
$$\theta_{1,3}\le \theta_{1,2}+\theta_{2,3}$$
and in case of equality, the vectors $w_1,w_2,w_3$ lie in a plane.
\end{thm}




\parit{Proof.}
We will view $\RR^2$ as the $xy$-plane in $\RR^3$; 
so both $A$ and $A'$ lie in $\RR^3$.
Let $a_m$ be the vertex of $A$ that lies on the maximal distance to the line $(a_0a_n)$.

Let us shift indexes of $a_i$ and $a'_i$ down by $m$,
so that 
\begin{align*}
a_{-m}&:=a_0,
&
a'_{-m}&:=a'_0,
\\
&\vdots&&\vdots
\\
a_{0}&:=a_m,
&
a'_{0}&:=a'_m,
\\
&\vdots&&\vdots
\\
a_k&:=a_n&a'_k&:=a'_n,
\end{align*}
where $k=n-m$.
(Here the symbol ``$:=$'' means \emph{an assignment} as in programming.)

Without loss of generality, we may assume that
\begin{itemize}
\item $a_0=a'_0$ and they both coincide with the origin $(0,0,0)\in\RR^3$;
\item all $a_i$ lie in the $xy$-plane and the $x$-axis is parallel to the line $(a_{-m}a_k)$;
\item the angle $\measuredangle a'_0$ lies in $xy$-plane and contains the angle $\measuredangle a_0$ inside
and the directions to $a'_{-1}$,$a_{-1}$, $a_{1}$ and $a'_{1}$ from $a_0$ appear in the same cyclic order.
\end{itemize}

\begin{wrapfigure}{r}{58mm}
\begin{lpic}[t(-20mm),b(-20mm),r(0mm),l(-3mm)]{pics/arm(0.3)}
\lbl[tr]{69,67;$a_0$}
\lbl[tl]{135,107;$a_1$}
\lbl[t]{135,67;\small{$x_1$}}
\lbl[l]{167,142;$a_2$}
\lbl[t]{168,67;\small{$x_2$}}
\lbl[l]{158,209;$a_3$}
\lbl[t]{156,67;\small{$x_3$}}
\lbl[l]{22,155;$a_{-1}$}
\lbl[t]{22,67;\small{$x_{-1}$}}
\lbl[lt]{46,206;$a_{-2}$}
\lbl[t]{46,67;\small{$x_{-2}$}}
\lbl{88,74;\small{$\sigma_1$}}
\lbl{117,76;$\sigma_2$}
\lbl{186,80;$\sigma_3$}
\end{lpic}
\end{wrapfigure}

Denote by $x_i$ and $x'_i$ the projections of $a_i$ and $a'_i$ to $x$-axis.
We can assume in addition that $x_k\ge x_{-m}$.
In this case 
$$|a_k-a_{-m}|=x_k-x_{-m}$$
and, since the projection is a distance non-expanding, we also have
$$|a'_k-a'_{-m}|\ge x'_k-x'_{-m}.$$ 

Therefore it is sufficient to show
that 
$$x'_k-x'_{-m}\ge x_k-x_{-m}.$$
The latter holds if
$$x'_i-x'_{i-1}\ge x_i-x_{i-1}.\eqlbl{eq:|bb|=<|aa|}$$
for each $i$.

It remains to prove \ref{eq:|bb|=<|aa|}.
Let us assume that $i>0$; 
the case $i\le 0$ is similar.
Let us
\begin{itemize}
\item denote by $\sigma_i$ the angle between the vector $w_i=a_{i}-a_{i-1}$ and the $x$-axis;
\item denote by $\sigma'_i$ the angle between the vector $w'_i=a'_{i}-a'_{i-1}$ and the $x$-axis.
\end{itemize}
Note that
$$\begin{aligned}
x_i-x_{i-1}&=|a_i-a_{i-1}|\cdot\cos\sigma_i,
\\
x'_i-x'_{i-1}&=|a_i-a_{i-1}|\cdot\cos\sigma'_i
\end{aligned}
\eqlbl{eq:proj}$$
for each $i>0$.
By construction $\sigma_1\ge \sigma'_1$.
Note that $\measuredangle (w_{i-1},w_i)\z=\pi -\measuredangle a_i$.
From convexity of $[a_1 a_1\dots a_i]$, we have
$$\sigma_i=\sigma_1+(\pi-\measuredangle a_1)+\dots+(\pi-\measuredangle a_i)$$
 for any $i>0$.
Since $\measuredangle (w'_{i-1},w'_i)=\pi -\measuredangle a'_i$.
Applying \ref{ex:angle-triangle} several times,
we get
$$\sigma'_i\le\sigma'_1+(\pi-\measuredangle a'_1)+\dots+(\pi-\measuredangle a'_i).$$
Since $\measuredangle a_j\le \measuredangle a'_j$ for each $j$, summing it up we get
$$\sigma_i\ge \sigma'_i.$$
Applying \ref{eq:proj}, we get \ref{eq:|bb|=<|aa|}.

In the case of equality, we have $\sigma_i=\sigma'_i$,
which implies $\measuredangle a_i=\measuredangle a'_i$ for each $i$.
This also implies that all $a'_i$ lie in $xy$-plane.
The latter easily follows from the equality case in \ref{ex:angle-triangle}.
\qeds
 
\parit{Proof of the local lemma (\ref{lem:local}).}
Assume that the local lemma does not hold at the vertex $v$ of $\Gamma$.
Let cut from $P$ a small pyramid $\Delta$ 
with vertex $v$.
Then one can choose two points $a$ and $b$ on the base of $\Delta$
so that on one side of the segments $[va]$ and $[vb]$ we have only pluses
and on the other side only minuses.

%\begin{wrapfigure}{r}{53mm}
%\begin{lpic}[t(-15mm),b(-12mm),r(0mm),l(0mm)]{pics/4-changes(0.25)}
%\lbl[wtl]{110,153;$v$}
%\lbl[tr]{70,88;$a_0=a$}
%\lbl[tr]{14,128;$a_1$}
%\lbl[br]{19,181;$a_2$}
%\lbl[br]{73,228;$a_3$}
%\lbl[bl]{139,222;$a_4$}
%\lbl[bl]{170,195;$b=a_5$}
%\lbl[w]{60,143;\tiny{$(+)$}}
%\lbl[w]{90,190;\tiny{$(+)$}}
%\lbl[w]{155,127;\tiny{$(-)$}}
%\end{lpic}
%\end{wrapfigure}

The base of $\Delta$ has two broken lines with ends at $a$ and $b$.
Assume that 
$$a=a_0,\ a_1,\ \dots,\ a_n=b$$ 
is the broken that has only pluses.
Denote by 
$$a'=a_0',\ a_1',\ \dots,\ a'_n=b'$$ 
the corresponding points in $P'$.
Since each marked edge passing through $a_i$ has a $(+)$ on it or nothing, 
we have $$\measuredangle a_i\le\measuredangle a'_i$$
for each $i$.
\begin{thm}{Exercise}
Prove the last statement. 
\end{thm}
By the construction we have $|a_i-a_{i-1}|=|a'_i-a'_{i-1}|$ for all $i$.
By the arm lemma (\ref{lem:arm}), 
we get 
\begin{clm}{}\label{clm:ab<ab}
$|a-b|\le |a'-b'|$ and equality holds if no edge from $v$ is marked.
\end{clm}
Repeating the same construction exchanging the places of $K$ and $K'$ gives 
\begin{clm}{}\label{clm:ab>ab}
$|a-b|\ge |a'-b'|$ and equality holds no edge from $v$ is marked with a $(-)$.
\end{clm}

The claims 
\ref{clm:ab<ab} and \ref{clm:ab>ab} 
together imply $|a-b|=|a'-b'|$ 
and it follows that no edge at $v$ is marked;
that is, $v$ is not a vertex of $\Gamma$
--- a contradiction.
\qeds

\section{Global lemma}

Before going into the proof, we suggest to do the following.

\begin{thm}{Exercise}\label{ex:octahedron}
Try to mark the edges of an octahedron
by pluses and minuses
such that there would be 4 sign changes at each vertex.

Show that this is impossible.
\end{thm}

The proof of the global lemma is based on counting the sign changes 
in two ways;
first while moving around each vertex of $\Gamma$ 
and second while moving around each of the regions separated by $\Gamma$
on the surface~$P$. 

If two edges are adjacent at a vertex,
then they are also adjacent in a region. 
The converse is true as well. 
Therefore, both countings give the same number.

\parit{Proof of \ref{lem:global}.}
We can assume that $\Gamma$ is connected;
that is, one can get from any vertex to any other vertex by walking along edges.
(If not, pass to a connected component of $\Gamma$.)
Denote by $k$ and $l$ the number of vertices and edges respectively in $\Gamma$.
Denote by $m$ the number of \emph{regions} which $\Gamma$ cuts from $P$.
Since $\Gamma$ is connected, each region is homeomorphic to an open disc.%
\begin{thm}{Exercise}
Prove the last statement.
\end{thm}
Now we can apply Euler's formula
$$k-l+m=2.
\eqlbl{eq:cauchy:euler}$$

Denote by $s$ the total number of sign changes in $\Gamma$ for all vertices. 
By the local lemma (\ref{lem:local}), we have 
$$ 4\cdot k\le s.\eqlbl{eq:S>=4k}$$

Now let us get an upper bound on $s$ by counting the number of sign changes when you go around
each region. 
Denote by $m_n$ the number of regions which are bounded by $n$ edges;
if an edge appears twice when it is counted twice.
Note that each region is bounded by at least $3$ edges;
therefore
$$m=m_3+m_4+m_5+\dots\eqlbl{eq:3-4-5}$$
Counting edges and using the fact that each edge belongs to exactly two regions, we get
$$2\cdot l=3\cdot m_3+ 4\cdot m_4+5\cdot m_5+\dots$$
Combining this with Euler's formula (\ref{eq:cauchy:euler}), we get
$$4\cdot k=8+2\cdot m_3+4\cdot m_4+6\cdot m_5+\dots
\eqlbl{eq:k=2+}$$
Observe that the number of sign changes in $n$-gon regions has to be even and $\le n$.
Therefore
$$s \le 2\cdot m_3 + 4\cdot m_4 + 4\cdot m_5 + 6\cdot m_6+\dots
\eqlbl{eq:23-44-45}$$
Clearly, \ref{eq:S>=4k} and \ref{eq:23-44-45} contradict \ref{eq:k=2+}.
\qeds


\section{Uniqueness}

Alexandrov's uniqueness theorem states that the conclusion of the Cauchy theorem (\ref{thm:cauchy}) still holds if one removes the phrase ``which sends each face of $K$ to a face of $K'$'' from it.

\begin{thm}{Theorem}\label{thm:alexandrov-uni'}
Any two convex polyhedra in $\RR^3$ with isometric surfaces are congruent.
\end{thm}

Let us describe the necessary modifications in the proof of the Cauchy theorem.

\parit{Modifications in the proof.}
Suppose $\iota\:P\to P'$ be an isometry between surfaces of $K$ and $K'$.
Mark in $P$ all the edges of $K$ and all the $\iota$-preimages of edges of $K'$, which will further be called fake edges.
These lines divide $P$ into convex polygons, say $\{Q_1,\dots,Q_m\}$, and the restriction of $\iota$ to each $Q_i$ is a rigid motion.
These polygons play the role of faces in the proof above.

A vertex of $Q_i$ can be a vertex of $K$ or it can be a fake vertex;
that is, lie on the intersection of an edge and a fake edge.
For the first type of vertex, the local lemma can be proved in the same way. 
For a fake vertex $v$, it is easy to see that both parts of the edge coming through $v$ are marked with $(+)$
while both of the fake edges at $v$ are marked with $(-)$.
Therefore, the local lemma holds for the fake vertices as well.

\begin{center}
\begin{lpic}[t(-35mm),b(-30mm),r(0mm),l(0mm)]{pics/fake(0.35)}
\lbl[r]{10,120;$a$}
\lbl[r]{116,120;$a'$}
\lbl[rb]{34,170;$b$}
\lbl[rb]{143,170;$b'$}
\lbl[bl]{86,145;$c$}
\lbl[tl]{202,167;$c'$}
\lbl[tl]{79,108;$d$}
\lbl[tl]{158,108;$d'$}
\lbl[rt]{54,139;$v$}
\lbl[lt]{162,143;$v'$}
\lbl[]{100,86;A fake vertex $v\in K$ and the corresponding point $v'\in K'$.}
\end{lpic}
\end{center}

What remains in the proof needs no modifications.
\qeds

\section{Existence}

Let $P$ be a polyhedral metric on the sphere.
The curvature of a point $p\in P$ is defined as $2\cdot \pi -\theta$,
where $\theta$ is the total angle around $p$.

\begin{thm}{Exercise}\label{ex:sphere-with-pos}
Suppose $P$ is the surface of a convex polyhedron.
Show that $P$ is homeomorphic to a sphere and it has nonnegative curvature at every point.
\end{thm}

Alexandrov's theorem states that the converse holds if one includes in the consideration \emph{twice covered polygons}.
In other words, we assume that a polyhedron can degenerate to a plane polygon;
in this case, its surface is defined as two copies of the polygon glued along the boundaries.

Further, we assume that a polyhedron can degenerate into a plane polygon.

\begin{thm}{Theorem}\label{thm:exist}
A polyhedral metric on the sphere is isometric to the surface of a convex polyhedron if and only if it has nonnegative curvature at each point.
\end{thm}

By \ref{thm:alexandrov-uni'} and \ref{thm:exist}, a convex polyhedron is completely defined by the intrinsic metric of its surface.
In particular, knowing the metric we could find the position of the edges.
However, in practice, it is not easy to do.
For example, the surface glued from a rectangle as shown on the diagram defines a tetrahedron.
Some of the glued lines appear inside facets of the tetrahedron and some edges (dashed lines) do not follow the sides of the rectangle.

{

\begin{wrapfigure}{r}{30mm}
\vskip-0mm
\centering
\includegraphics{mppics/pic-10}
\vskip-0mm
\end{wrapfigure}

The theorem was proved by A. D. Alexandrov in 1941 \cite{alexandrov-1941};
we will present a sketch of his proof.
A complete proof is nicely written by A. D. Alexandrov in his book~\cite{alexandrov}.
Yet another proof was found by Yu.~A.~Volkov in his thesis \cite{volkov};
it uses a deformation of three-dimensional polyhedral space.

}

\paragraph{Space of polyhedrons.}
Let us denote by $\mathbf{K}$ the space of all convex polyhedrons in the Euclidean space,
including polyhedrons that degenerate to a plane polygon.
Polyhedra in $\mathbf{K}$ will be considered up to a motion of the space, 
and the whole space $\mathbf{K}$ will be considered with the natural topology (so far an intuitive meaning of closeness of two polyhedrons should be sufficient). 

Further, denote by $\mathbf{K}_n$ the polyhedrons in $\mathbf{K}$ with exactly $n$ vertices.
Since any polyhedron has at least 3 vertices, the space $\mathbf{K}$ admits a subdivision into a countable number of subsets $\mathbf{K}_3,\mathbf{K}_4,\dots$

\paragraph{Space of polyhedral metrics.}
The space of polyhedral metrics on the sphere with the sum of angles around each point at most $2\cdot\pi$ will be denoted by $\mathbf{P}$.
The metrics in $\mathbf{P}$ will be considered up to an isometry, and the whole space $\mathbf{P}$ will be equipped with the natural topology (again, an intuitive meaning of closeness of two metrics is sufficient).

A point on the sphere with positive curvature will be called an \emph{essential vertex}.
The subset of $\mathbf{P}$ of all metrics with exactly $n$ essential vertices will be denoted by $\mathbf{P}_n$.
It is easy to see that any metric in $\mathbf{P}$ has at least 3 essential vertices.
Therefore $\mathbf{P}$ is subdivided into countably many subsets
 $\mathbf{P}_3,\mathbf{P}_4,\dots$

\paragraph{From a polyhedron to its surface.}

By \ref{ex:sphere-with-pos}, passing from a polyhedron to its surface defines a map
\[\iota\:\mathbf{K}\to \mathbf{P}.\]

Note that the number of vertices of a polyhedron is equal to the number of essential vertices of its surface.
In other words, $\iota(\mathbf{K}_n)\subset \mathbf{P}_n$ for any $n\ge 3$.

Using the introduced notation, we can give the following reformulation for both theorems \ref{thm:alexandrov-uni'} and \ref{thm:exist}.

\begin{thm}{Reformulation}
For any integer $n\ge 3$,
the map $\iota$ is a bijection from $\mathbf{K}_n$ to~$\mathbf{P}_n$.
\end{thm}

The proof is based on a construction of a one-parameter family of polyhedrons that starts at an arbitrary polyhedron
and ends at a polyhedron with its surface isometric to the given one.
This type of argument is called the \emph{continuity method}; it is often used in the theory of differential equations.

\medskip

The two parts of the first formulation will be proved separately.

\parit{Sketch.}
By \ref{thm:alexandrov-uni'}, the map $\iota\:\mathbf{K}_n\to\mathbf{P}_n$ is injective.
Let us prove that it is surjective.
This part of the proof is subdivided into the following lemmas:

\begin{thm}{Lemma}
For any integer $n\ge 3$, the space $\mathbf{P}_n$ is connected.
\end{thm}

The proof of this lemma is not complicated, but it requires ingenuity;
it can be done by the direct construction of a one-parameter family of metrics in $\mathbf{P}_n$ that connects two given metrics.
Such a family can be obtained by а sequential application of the following construction and its inverse.

Let $P$ be a sphere with metric from $\mathbf{P}_n$.
Suppose $v$ and $w$ are essential vertices in $P$.
Let us cut $P$ along a geodesic from $v$ to $w$.
Note that the geodesic cannot pass thru an essential vertex of $P$.
Further, note that there is a three-parameter family of patches that can be used to patch the cut so that the obtained metric remains in $\mathbf{P}_n$;
in particular, the obtained metric has exactly $n$ essential vertices (after the patching, the vertices $v$ and $w$ may become inessential).


\begin{thm}{Lemma}
The map $\iota\:\mathbf{K}_n\to\mathbf{P}_n$ is open, 
that is, it maps any open set in $\mathbf{K}_n$ to an open set in $\mathbf{P}_n$.

In particular, for any $n\ge 3$, the image $\iota(\mathbf{K}_n)$ is open in~$\mathbf{P}_n$.
\end{thm}

This statement is very close to the so-called \emph{invariance of domain theorem};
the latter states that a continuous injective map between manifolds of the same dimension is open.

Recall that $\iota$ is injective.
The proof of the invariance of domain theorem can be adapted to our case since both spaces $\mathbf{K}_n$ and $\mathbf{P}_n$ are $(3\cdot n-6)$-dimensional and both look like manifolds, altho, formally speaking, they are \emph{not} manifolds.
In a more technical language, $\mathbf{K}_n$ and $\mathbf{P}_n$ have the natural structure of $(3\cdot n-6)$-dimensional \emph{orbifolds},
and the map $\iota$ respects the \emph{orbifold structure}.

We will only show that both spaces $\mathbf{K}_n$ and $\mathbf{P}_n$ are $(3\cdot n-6)$-dimensional.

Choose a polyhedron $K$ in $\mathbf{K}_n$.
Note that $K$ is uniquely determined by the $3\cdot n$ coordinates of its $n$ vertices.
We can assume that the first vertex is the origin, the second has two vanishing coordinates and the third has one vanishing coordinate; therefore, all polyhedrons in $\mathbf{K}_n$ that lie sufficiently close to $K$ can be described by $3\cdot n-6$ parameters.
If $K$ has no symmetries then this description can be made one-to-one;
in this case, a neighborhood of $K$ in $\mathbf{K}_n$ is a $(3\cdot n-6)$-dimensional manifold.
If $K$ has a nontrivial symmetry group, then this description is not one-to-one but it does not have an impact on the dimension of $\mathbf{K}_n$.

The case of polyhedral metrics is analogous.
We need to construct a subdivision of the sphere into plane triangles using only essential vertices.
By Euler's formula, there are exactly $3\cdot n-6$ edges in this subdivision.
Note that the lengths of edges completely describe the metric, and slight changes in these lengths produce a metric with the same property.

\begin{thm}{Lemma}
The map $\iota\:\mathbf{K}_n\to\mathbf{P}_n$ is closed;
that is, the image of a closed set in $\mathbf{K}_n$ is closed in $\mathbf{P}_n$.

In particular, for any $n\ge 3$, the set $\iota(\mathbf{K}_n)$ is closed in~$\mathbf{P}_n$.
\end{thm}

Choose a closed set $Z$ in $\mathbf{K}_n$.
Denote by $\bar Z$ the closure of $Z$ in $\mathbf{K}$; note that $Z=\mathbf{K}_n\cap \bar Z$.
Assume $K_1,K_2,\dots\in Z$ is a sequence of polyhedrons that converges to a polyhedron $K_\infty\in\bar Z$.
Note that $\iota(K_n)$ converges to $\iota(K_\infty)$ in $\mathbf{P}$.
In particular, $\iota(\bar Z)$ is closed in $\mathbf{P}$.

Since $\iota(\mathbf{K}_n)\subset \mathbf{P}_n$ for any $n\ge 3$, we have $\iota (Z)=\iota(\bar Z)\cap \mathbf{P}_n$;
that is, $\iota (Z)$ is closed in $\mathbf{P}_n$. 

\medskip

Summarizing, $\iota(\mathbf{K}_n)$ is a nonempty closed and open set in $\mathbf{P}_n$, and $\mathbf{P}_n$ is connected for any $n\ge 3$.
Therefore, $\iota(\mathbf{K}_n)=\mathbf{P}_n$; that is, $\iota\:\mathbf{K}_n\z\to\mathbf{P}_n$ is surjective.
\qeds

\section{Comments}

In Euclid's Elements, 
solids were called equal if the same holds for their faces, but no proof was given.
Adrien-Marie Legendre became interested in it towards the end of the 18th century and
talked to his colleague Joseph-Louis Lagrange about it, who in turn suggested this problem to Augustin-Louis Cauchy in 1813 who soon proved it.
In 1950, Alexandrov understood that the condition of the equality of faces can be weakened.

We present a proof that has only minor modifications of Alexandrov's original proof in \cite{alexandrov}.


\parit{Arm lemma.}
Original Cauchy's proof \cite{cauchy}
also used a version of the arm lemma, 
but its proof contained a small error.

The presented proof of arm lemma is due to Stanisław Zaremba.
This and yet a couple of other beautiful proofs can be found in the letters between Isaac Schoenber and Stanisław Zaremba \cite{schoenberg-zaremba}.

The following variation of the arm lemma makes sense for nonconvex spherical polygons.
It is due to Viktor Zalgaller \cite{zalgaller}.
It can be used in the proof of uniqueness theorem instead of the standard arm lemma.

\begin{thm}{Another arm lemma}
Let $A=a_1\dots a_n$ and $A'\z=a'_1\z\dots a'_n$ be two spherical $n$-gons (not necessarily convex).
Assume that $A$ lies in a half-sphere,
the corresponding sides of $A$ and $A'$ are equal
and each angle of $A$ is at least the corresponding angle in $A'$.
Then $A$ is congruent to~$A'$. 
\end{thm}

\parit{Global lemma.}
A more visual proof of the global lemma is given in \cite[II \S 1.3]{alexandrov}.
