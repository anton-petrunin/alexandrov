\chapter{CBB: globalization}

\section{Hinge comparison}

Let $\hinge pxy$ be a hinge in a $\CBB(0)$ space.
By \ref{ex:0-angle}, the angle measure $\mangle\hinge pxy$ is defined and
\[\mangle\hinge pxy\ge \angk pxy.\]

Further, according to \ref{ex:adjacent-CBB}, we have 
\[\mangle\hinge pxz+\mangle\hinge pyz=\pi\]
for adjacent hinges $\hinge pxz$ and $\hinge pyz$ in a $\CBB(0)$ space.

The following theorem implies that a geodesic space is $\CBB(0)$ if
the above conditions hold for all its hinges.

\begin{thm}{Theorem}\label{thm:angle-cbb}
A geodesic space $\spc{L}$ is $\CBB(0)$ if the following conditions hold.

\begin{subthm}{angle-a}
For any hinge $\hinge x p y$ in $\spc{L}$, the angle 
$\mangle\hinge x p y$ is defined and 
\[\mangle\hinge x p y\ge\angk x p y.\]
\end{subthm}

\begin{subthm}{angle-b}
For any two adjacent hinges $\hinge pxz$ and $\hinge pyz$ in $\spc{L}$, we have
\[\mangle\hinge pxz+\mangle\hinge pyz\le\pi.\]
\end{subthm}

\end{thm}

\parit{Proof.}
Consider a point  $w\in \mathopen{]} p z \mathclose{[}$ close to $p$.
From \ref{SHORT.angle-b}, it follows that 
\[\mangle\hinge w x z+ \mangle\hinge w x{p}\le\pi\quad \text{and}\quad \mangle\hinge w y z + \mangle\hinge w y{p}\le\pi.\]

\begin{wrapfigure}{o}{30 mm}
\vskip-0mm
\centering
\includegraphics{mppics/pic-805}
\vskip4mm
\end{wrapfigure}

Since $\mangle\hinge w x y\le \mangle\hinge w x p +\mangle\hinge w y{p}$ (see \ref{claim:angle-3angle-inq}), we get 
\[\mangle\hinge w x z+ \mangle\hinge w y z +\mangle\hinge w x y
\le
2\cdot\pi.\]
Applying \ref{SHORT.angle-a}, 
\[\angk w x z
+ \angk w y z 
+\angk w x y
\le
2\cdot\pi.\]
Passing to the limits $w\to p$, we have
\[\angk p x z 
+ \angk p y z 
+\angk p x y
\le
2\cdot\pi.\]
\qedsf

\section{Equivalent conditions}

The following theorem summarizes \ref{clm:angle-mono}, \ref{clm:angle-defined}, \ref{ex:adjacent-CBB}, \ref{thm:angle-cbb}.

\begin{thm}{Theorem}\label{thm:defs_of_alex} 
Let $\spc{L}$ be a geodesic space.
Then the following conditions are equivalent.

\begin{subthm}{cbb}
$\spc{L}$ is $\CBB(0)$.
\end{subthm}
 

\begin{subthm}{2-sum} 
(adjacent angle comparison\index{comparison!adjacent angle comparison}) for any geodesic $[x y]$ and point $z\in \mathopen{]}x y\mathclose{[}$, $z\ne p$ in $\spc{L}$, we have
\[\angk z p x
+\angk z p y\le \pi.\]
\end{subthm}

\begin{subthm}{point-on-side}
(\index{comparison!point-on-side comparison}point-on-side comparison)
for any geodesic $[x y]$ and $z\in \mathopen{]}x y\mathclose{[}$ in $\spc{L}$, we have
\[\angk x p y\le\angk x p z.\]
\end{subthm}

\begin{subthm}{angle}(hinge comparison\index{comparison!hinge comparison})
\index{hinge comparison}
for any hinge $\hinge x p y$ in $\spc{L}$, the angle 
$\mangle\hinge x p y$ is defined and 
\[\mangle\hinge x p y\ge\angk x p y.\]
Moreover, 
\[\mangle\hinge z p y + \mangle\hinge z p x\le\pi\]
for any adjacent hinges $\hinge z p y$ and $\hinge z p x$.
\end{subthm}

Moreover, the implications \ref{SHORT.cbb}$\Rightarrow$\ref{SHORT.2-sum}$\Rightarrow$\ref{SHORT.point-on-side}$\Rightarrow$\ref{SHORT.angle} hold in any space, not necessarily geodesic.
\end{thm}

\begin{thm}{Advanced Exercise} \label{ex:urysohn}
Construct a geodesic space $\spc{X}\notin\CBB(0)$ that meets the following condition: for any 3 points $p,x,y\in \spc{X}$ there is a geodesic $[x y]$ such that for any $z\in \left]x y\right[$
\[\angk{z}{p}{x}+\angk{z}{p}{y}
\le
\pi.\]
\end{thm}

\section{Globalization}

A metric space $\spc{L}$ is \index{locally $\CBB(0)$}\emph{locally $\CBB(0)$} if any point $p\in\spc{L}$ admits a neighborhood $U\ni p$ such that the  $\CBB(0)$ comparison holds for any quadruple of points in $U$.

\begin{thm}{Globalization theorem}\label{thm:glob} 
Any locally $\CBB(0)$ compact geodesic space is $\CBB(0)$.
\end{thm}

\parit{Proof modulo the key lemma.}
Let $\spc{L}$ be a locally $\CBB(0)$ compact geodesic space.
Note that condition \ref{angle-b} holds in $\spc{L}$ (the proof is the same as for $\CBB(0)$ space).
It remains to prove that \ref{angle-a} holds in $\spc{L}$;
that is,
\[\mangle\hinge x p y\ge\angk x p y.
\eqlbl{eq:mod-angle-CBB-comp-glob}\]

First note that \ref{eq:mod-angle-CBB-comp-glob} holds for hinges in a small neighborhood of any point;
this can be proved the same way as \ref{clm:angle-defined} and \ref{ex:adjacent-CBB}, applying the local version of $\CBB(0)$ comparison.
Since $\spc{L}$ is compact, there is $\eps>0$ such that \ref{eq:mod-angle-CBB-comp-glob} holds if $\dist{x}{p}{}+\dist{p}{y}{}<\eps$.
Applying the key lemma several times we get that \ref{eq:mod-angle-CBB-comp-glob} holds for any given hinge.
\qeds

\begin{thm}{Key lemma}\label{key-lem:globalization} 
Let $\spc{L}$ be a locally $\CBB(0)$ geodesic space. 
Assume that the comparison
\[\mangle\hinge x p q
\ge\angk x p q\]
holds for any hinge $\hinge x p q$ with 
$\dist{x}{y}{}+\dist{x}{q}{}
<
\frac{2}{3}\cdot\ell$.
Then comparison
\[\mangle\hinge x p q
\ge\angk x p q\] 
holds for any hinge $\hinge x p q$ with $\dist{x}{ p}{}+\dist{x}{q}{}<\ell$.
\end{thm}

Let $\hinge x p q$ be a hinge in a metric space $\spc{L}$ with defined angle measure.
Denote by $\side \hinge x p q$ its \index{model side}\emph{model side};
this is the opposite side in a flat triangle with the same angle and two adjacent sides as in $\hinge x p q$.

\begin{wrapfigure}{r}{44mm}
\centering
\includegraphics{mppics/pic-105}
\end{wrapfigure}

More precisely,
consider the model hinge $\hinge {\tilde x} {\tilde p} {\tilde q}$ in $\EE^2$ that is defined by 
\begin{align*}
\mangle\hinge {\tilde x} {\tilde p} {\tilde q}_{\EE^2}&=\mangle\hinge x p q_{\spc{L}},
\\
\dist{\tilde x} {\tilde p}{\EE^2}&=\dist{x} {p}{\spc{L}},
\\
\dist{\tilde x} {\tilde q}{\EE^2}&=\dist{x} {q}{\spc{L}};
\intertext{then}
\side \hinge x p q_{\spc{L}}
&\df
\dist{\tilde p}{\tilde q}{\EE^2}.
\end{align*}

Note that 
\[\side \hinge x p q \ge\dist{p}{q}{}
\quad\Longleftrightarrow\quad
\mangle\hinge x p q\ge \angk x p q.
\]
We will use it in the following proof.

\parit{Proof.} 
It is sufficient to prove the inequality
\[\side \hinge x p q
\ge\dist{p}{q}{}\eqlbl{eq:thm:=def-loc*}\] 
for any hinge $\hinge x p q$ with $\dist{x}{p}{}+\dist{x}{q}{}<\ell$.

Consider a hinge $\hinge x p q$ such that 
\[\tfrac{2}{3}\cdot\ell \le\dist{p}{x}{}\z+\dist{x}{q}{}< \ell.\]
First, let us construct a new smaller hinge $\hinge{x'}p q$ with
\[
\dist{p}{x}{}+\dist{x}{q}{}\ge\dist{p}{x'}{}+\dist{x'}{q}{},
\eqlbl{eq:thm:=def-loc-fourstar}\]
such that 
\[\side \hinge x p q
\ge\side \hinge{x'}p q.
\eqlbl{eq:thm:=def-loc-fivestar}\]

\parit{Construction.}
Assume $\dist{x}{q}{}\ge\dist{x}{p}{}$; otherwise switch the roles of $p$ and $q$ in the following construction.
Take $x'\in [x q]$ such that 
\[\dist{p}{x}{}+3\cdot\dist[{{}}]{x}{x'}{}
=\tfrac{2}{3}\cdot\ell. \eqlbl{3|xx'|}\]
Choose a geodesic $[x' p]$ and consider the  hinge $\hinge{x'}p q$ formed by $[x'p]$ and $[x' q]\subset [x q]$.
Then \ref{eq:thm:=def-loc-fourstar} follows from the triangle inequality.

Further, note that 
\begin{align*}
\dist{p}{x}{}\z+\dist{x}{x'}{}&<\tfrac{2}{3}\cdot\ell,
&
\dist{p}{x'}{}\z+\dist{x'}{x}{}&<\tfrac{2}{3}\cdot\ell.
\end{align*}
In particular, 
\[\mangle\hinge x p{x'}
\ge\angk x p{x'}
\quad \text{and}\quad 
\mangle\hinge {x'}p x
\ge\angk {x'}p x.
\eqlbl{eq:thm:=def-loc-threestar}\]

{

\begin{wrapfigure}{r}{30 mm}
\vskip-6mm
\centering
\includegraphics{mppics/pic-820}
\vskip-4mm
\end{wrapfigure}

Now, let 
$\trig{\tilde x}{\tilde x'}{\tilde p}=\modtrig(x x' p)$.
Take $\tilde  q$ on the extension of $[\tilde  x\tilde  x']$ beyond $x'$ such that $\dist{\tilde x}{\tilde q}{}\z=\dist{x}{q}{}$ (and therefore $\dist{\tilde x'}{\tilde q}{}=\dist{x'}{q}{}$).
By~\ref{eq:thm:=def-loc-threestar},
\[\mangle\hinge x p q
=\mangle\hinge  x p{x'}\ge\angk x p{x'}\quad \Rightarrow\quad 
\side \hinge x q p\ge\dist{\tilde p}{\tilde q}{}.\]
Hence
\begin{align*}
\mangle\hinge{\tilde x'}{\tilde p}{\tilde q}&= 
\pi
-\angk{x'}p x
\ge
\\
&\ge
\pi-\mangle\hinge{x'}p x
=
\\
&=
\mangle\hinge{x'}p q,
\end{align*}
and \ref{eq:thm:=def-loc-fivestar} follows.

}

\medskip

Let us continue the proof.
Set $x_0=x$.
Let us apply inductively the above construction to get a sequence of hinges  $\hinge{x_n}p q$ with $x_{n+1}=x_n'$.
From \ref{eq:thm:=def-loc-fivestar}, we have that the sequence  $s_n\z=\side \hinge{x_n}p q$ is nonincreasing.
\begin{figure}[ht!]
\centering
\includegraphics{mppics/pic-825}
\end{figure}

The sequence might terminate at some $n$ only if $\dist{p}{x_n}{}+\dist{x_n}{q}{}\z< \tfrac{2}{3}\cdot\ell $.
In this case, by the assumptions of the lemma, $\side \hinge{x_n}p q\ge\dist{p}{q}{}$.
Since the sequence $s_n$ is nonincreasing, inequality \ref{eq:thm:=def-loc*} follows.

Otherwise, the sequence $r_n=\dist{p}{x_n}{}+\dist{x_n}{q}{}$ is nonincreasing, and $r_n\ge\tfrac{2}{3}\cdot\ell$ for all $n$.
Note that by construction, the distances
$\dist{x_n}{x_{n+1}}{}$, $\dist{x_n}{p}{}$, and $\dist{x_n}{q}{}$ are bounded away from zero for all large $n$.
Indeed, since on each step, we move $x_n$ toward to the point $p$ or $q$ that is further away, the distances $\dist{x_n}{p}{}$ and $\dist{x_n}{q}{}$ become about the same.
Namely, by \ref{3|xx'|}, we have that $\dist{p}{x_n}{}-\dist{x_n}{q}{}\le \tfrac29\cdot\ell$ for all large $n$.
Since $\dist{p}{x_n}{}+\dist{x_n}{q}{}\ge \tfrac23\cdot\ell$, we have $\dist{x_n}{p}{}\ge \tfrac\ell{100}$ and $\dist{x_n}{q}{}\ge \tfrac\ell{100}$.
Further, since $r_n\ge\tfrac{2}{3}\cdot\ell$, \ref{3|xx'|} implies that $\dist{x_n}{x_{n+1}}{}>\tfrac\ell{100}$.


Since the sequence $r_n$ is nonincreasing, it converges.
In particular, $r_n-r_{n+1}\to 0$ as $n\to\infty$.
It follows that $\angk{x_n}{p_n}{x_{n+1}}\to \pi$,
where $p_n=p$ if $x_{n+1}\in [x_nq]$, and otherwise $p_n=q$.
Since $\mangle\hinge{x_n}{p_n}{x_{n+1}}\ge\angk{x_n}{p_n}{x_{n+1}}$, we have
$\mangle\hinge{x_n}{p_n}{x_{n+1}}\to \pi$  as $n\to\infty$.

It follows that
\[r_n-s_n=\dist{p}{x_n}{}+\dist{x_n}{q}{}-\side \hinge{x_n}p q\to 0.\] 
Together with the triangle inequality
\[
\dist{p}{x_n}{}+\dist{x_n}{q}{}\ge\dist{p}{q}{}
\]
this yields
\[\lim_{n\to\infty}\side \hinge{x_n}p q\ge \dist{p}{q}{}.\]
Applying monotonicity of the sequence  $s_n=\side \hinge{x_n}p q$, we obtain \ref{eq:thm:=def-loc*}.
\qeds

\section{On general curvature bound}

The globalization theorem  can be generalized to $\CBB(\kappa)$ spaces for any real $\kappa$.
The case $\kappa\le 0$ is proved the same way, but the case $\kappa>0$ requires minor modifications.

\begin{thm}{Exercise}\label{ex:mono-mod-angle}
Suppose $\kappa\le \Kappa$.
Show that 
\[\angk x y z_{\MM(\kappa)}\le \angk x y z_{\MM(\Kappa)}\]
if the right-hand side is defined.

Conclude that any $\CBB(\Kappa)$ space is locally $\CBB(\kappa)$.
\end{thm}

The exercise and the globalization theorem (here we need a more general version \ref{ex:noncomplete-globalization}) imply that \textit{any geodesic $\CBB(\Kappa)$ space is  $\CBB(\kappa)$}.
Recall that $\CBB(\kappa)$ stands for \textit{curvature bounded below by $\kappa$}; so, for geodesic spaces it makes sense.
However, as you can see from the following exercise, it does not make much sense in general.

\begin{thm}{Exercise}\label{ex:CBB(1)notitCBB(0)}
Let $\spc{X}$ be the set $\{p,x_1,x_2,x_3\}$ with the metric defined by
\[\dist{p}{x_i}{}=\pi,\quad\dist{x_i}{x_j}{}=2\cdot\pi\]
for all $i\ne j$.
Show that $\spc{X}$ is $\CBB(1)$, but not $\CBB(0)$.
\end{thm}

\begin{thm}{Exercise}\label{ex:RisCBB(1)}
Let $p$ and $q$ be points in a $\CBB(1)$ geodesic space $\spc{L}$.
Suppose $\dist{p}{q}{}>\pi$.
Denote by $m$ the midpoint of $[pq]$.
Show that for any hinge $\hinge mxp$ we have
either $\mangle\hinge mxp=0$ or $\mangle\hinge mxp=\pi$.
Conclude that $\spc{L}$ is isometric to a real interval or a circle.

\end{thm}


\section{Remarks} 

The globalization theorem is also known as the \textit{generalized Toponogov theorem}.

Recall that a metric space $\spc{X}$ is called \index{complete space}\emph{complete} if any Cauchy sequence of points in $\spc{X}$ converges.
The compactness condition in our version of the theorem can be traded to completeness by using the following exercise.

\begin{thm}{Exercise}\label{ex:alm-min}
Let $\spc{X}$ be a complete metric space.
Suppose $r\:\spc{X}\to \RR$ is a positive continuous function.
Show that for any $\eps>0$ there is a point $p\in \spc{X}$ such that 
\[r(x)> (1-\eps)\cdot r(p)\] 
for any $x\in \cBall[p,\tfrac{1}{\eps}\cdot r(p)]$.

\end{thm}

Let us mention two more general versions of the globalization theorem.

Recall that \index{length space}\emph{length space} is a metric space such that any two points $p$ and $q$ can be connected by a path with length arbitrarily close to $|p-q|$.
Note that any geodesic space is length, but not the other way around.
The following theorem was proved already in the paper of Michael Gromov, Yuriy Burago, and Grigory Perelman~\cite{burago-gromov-perelman}.

\begin{thm}{Theorem}\label{ex:noncomplete-globalization}
Any complete locally $\Alex\kappa$ length space is $\Alex\kappa$.

\end{thm}

The next result is mine \cite{petrunin:globalization}.

\begin{thm}{Theorem}\label{ex:noncomplete-globalization}
Any locally $\Alex\kappa$ geodesic space is $\Alex\kappa$.

\end{thm}

In the two-dimensional case,
the globalization theorem was proved by Paolo Pizzetti \cite{pizzetti};
later it was reproved independently by Alexandr Alexandrov \cite{alexandrov:devel}. %is it right ref?? 
Victor Toponogov \cite{toponogov-globalization+splitting} proved it for Riemannian manifolds of all dimensions.

I took the proof from our book \cite{alexander-kapovitch-petrunin-bigbook} (with generality reduction).
It uses simplifications obtained by Conrad Plaut \cite{plaut:dimension} and Dmitry Burago, Yuriy Burago, and Sergei Ivanov \cite{burago-burago-ivanov}.
The same proof was rediscovered independently by Urs Lang and Viktor Schroeder \cite{lang-schroeder:globalization}.
Another simplified version was obtained by
Katsuhiro Shiohama \cite{shiohama}.

The question whether \ref{angle-a} suffices to conclude that $\spc{L}$ is $\Alex\kappa$ is a long-standing open problem (possibly dating back to Alexandrov);
in print, it was first stated in \cite[footnote in 4.1.5]{burago-burago-ivanov}.


\begin{thm}{Open question}\label{open:hinge-}
Let $\spc{L}$ be a complete geodesic space (you can also assume that $\spc{L}$ is homeomorphic to $\mathbb{S}^2$ or $\RR^2$) 
such that for any hinge $\hinge x p y$ in $\spc{L}$, 
the angle $\mangle\hinge x p y$ is defined and 
\[\mangle\hinge x p y\ge\angk x p y.\]
Is it true that $\spc{L}$ is $\Alex{0}$?
\end{thm}




