\chapter{Quotients}

\section{Quotient space}

Suppose that group $G$ acts isometrically on a metric space $\spc{X}$.
Note that
\[\dist{G\cdot x}{G\cdot y}{\spc{X}/G}
\df
\inf
\set{\dist{x}{g\cdot y}{\spc{X}}}{g\in G}\]
defines a semimetric on the orbit space $\spc{X}/G$.
Moreover, it is a genuine metric if the orbits of the action are closed.

\begin{thm}{Theorem}\label{thm:CBB/G}
Assume that group $G$ acts isometrically on a proper $\Alex0$ space $\spc{L}$ and has closed orbits.
Then the quotient space $\spc{L}/G$ is $\Alex0$.

\end{thm}

\parit{Proof.}
Denote by $\sigma\:\spc{L}\to \spc{L}/G$ the quotient map.

Fix a quadruple of points $p,x_1,x_2,x_3\in \spc{L}/G$.
Choose an arbitrary $\hat p\in \spc{L}$ such that $\sigma(\hat{p})=p$.
Since $\spc{L}$ is proper, we can choose the points $\hat{x}_1,\hat{x}_2,\hat{x}_3\in \spc{L}$ such that $\sigma(\hat x_i)=x_i$ and
\[\dist{p}{x_i}{\spc{L}/G}
=
\dist{\hat{p}}{\hat{x}_i}{\spc{L}}\]
for all $i$.

Note that 
\[\dist{x_i}{x_j}{\spc{L}/G}
\le 
\dist{\hat{x}_i}{\hat{x}_j}{\spc{L}}
\]
for all $i$ and $j$.
Therefore 
\[\angk p{x_i}{x_j}
\le
\angk {\hat{p}}{\hat{x}_i}{\hat{x}_j}
\eqlbl{eq:angles-M-L}\]
holds for all $i$ and $j$.

By $\Alex0$ comparison in $\spc{L}$,
we have
\[\angk {\hat{p}}{\hat{x}_1}{\hat{x}_2}
+\angk {\hat{p}}{\hat{x}_2}{\hat{x}_3}
+\angk {\hat{p}}{\hat{x}_3}{\hat{x}_1}
\le 
2\cdot\pi.\]
Applying  \ref{eq:angles-M-L}, 
we get 
\[\angk p{x_1}{x_2}
+\angk p{x_2}{x_3}
+\angk p{x_3}{x_1}< 2\cdot\pi;\]
that is,
the $\Alex0$ comparison holds for this quadruple in $\spc{L}/G$.
\qeds

\begin{thm}{Advanced exercise}
Let $G$ be a compact Lie group with a bi-invariant Riemannian metric.
Show that $G$ is isometric to a quotient of the Hilbert space by isometric group action.

Conclude that $G\in\CBB(0)$.
\end{thm}

\section{Submetries}

A map $\sigma\:\spc{X}\to\spc{Y}$ between the metric spaces $\spc{X}$ and $\spc{Y}$
is called a \index{submetry}\emph{submetry} if 
\[\sigma(\oBall(p,r)_\spc{X})=\oBall(\sigma(p),r)_{\spc{Y}}\]
for any $p\in \spc{X}$ and $r\ge 0$.

Suppose $G$ and $\spc{L}$ are as in \ref{thm:CBB/G}.
Observe that the quotient map $\sigma\:\spc{L}\to \spc{L}/G$ is a submetry.
Moreover, the proof above (with minor changes) works for any submetries.
Therefore we get the following.

\begin{thm}{Generalization}\label{thm:submetry-CBB}
Let $\sigma\:\spc{L}\to\spc{M}$ be a submetry.
Suppose $\spc{L}$ is a $\Alex{0}$ space, then so is $\spc{M}$.
\end{thm}

Theorem \ref{thm:CBB/G} admits a straightforward generalization to $\CBB(-1)$ spaces.
In the $\CBB(1)$ the proof produces a slightly weaker statement ---  \textit{any open $\tfrac\pi2$-ball in the quotient of $\CBB(1)$ is $\CBB(1)$};
in particular, the quotient space is \textit{locally} $\CBB(1)$.
If the space is geodesic, then the globalization theorem implies that it is globally  $\CBB(1)$.
The same holds for the targets of submetry from a  $\CBB(1)$ space.

\section{Hopf's conjecture}

Recall that Hopf's conjecture states that \textit{$\mathbb{S}^2\times\mathbb{S}^2$ does not admit a Riemannian metric with positive sectional curvature}.

\begin{thm}{Theorem}
There is no Riemannian metric on $\SSS^2\times\SSS^2$ with sectional curvature $\ge 1$ and nontrivial isometric $\SSS^1$-action.
\end{thm}

Recall that a complete Riemannian manifold has sectional curvature $\ge 1$ if and only if the corresponding metric space is $\CBB(1)$.

We will give a ruf sketch of proof that will use many statements and notions that are not yet introduced.
Nevertheless, you should get its intuitive idea.

The statements include
the \textit{Doubling theorem} and \textit{Splitting theorem}.
The first say that \textit{doubling of a finite-dimensional geodesic $\CBB(\kappa)$ space is a geodesic $\CBB(\kappa)$ space}.
The second says that \textit{a geodesic $\CBB(0)$ space that contains a line splits isometrically as $\RR\times \spc{X}$};
here \emph{line} is a both-side infinite geodesic.

In the proof, we will use the following exercise and two theorem that will be proved latter:
;


\begin{thm}{Exercise}\label{ex:S^3/S^1}
Suppose $\SSS^1\acts\SSS^3$ be an isometric action without fixed points 
and $\Sigma=\SSS^3/\SSS^1$ is its quotient space.
Then there is a distance noncontracting map $\Sigma\to \tfrac12\cdot \SSS^2$, where $\tfrac12\cdot \SSS^2$ is the standard 2-sphere rescaled with factor $\tfrac12$.
\end{thm}


\parit{Sketch.}
Let $\spc{M}=(\SSS^2\times\SSS^2,g)$ be a counterexample.

Recall that $\spc{M}$ is $\CBB(1)$.
By \ref{thm:CBB/G}, the quotient space $\spc{L}=\spc{M}/\SSS^1$ is $\CBB(1)$;
evidently, $\spc{L}$ is 3-dimensional.

Denote by $F\subset \spc{M}$ the fixed point set of the $\SSS^1$-action.
Each connected component of $F$ is either an isolated point or a 2-dimensional submanifold in $\spc{M}$;
the latter has to have positive curvature and therefore it is either $\SSS^2$ or $\RP^2$.
Notice that 
\begin{itemize}
 \item each isolated point contributes 1 to the Euler characteristic of $\spc{M}$,
 \item each sphere contributes 2 to the Euler characteristic of $\spc{M}$, and
 \item each projective plane contributes 1 to the Euler characteristic of $\spc{M}$.
\end{itemize}
Since $\chi(\spc{M})=4$, we are in one of the following three cases:
\begin{itemize}
 \item $F$ has exactly 4 isolated points,
 \item $F$ has one 2-dimensional submanifold and at least 2 isolated points,
 \item $F$ has two 2-dimensional submanifolds.
\end{itemize}
Each case is covered separately.

\parit{Case 1.}
Suppose $F$ has exactly 4 isolated points $x_1$, $x_2$, $x_3$, and $x_4$.
Denote by $y_1$, $y_2$, $y_3$, and $y_4$ the corresponding points in $\spc{L}$.
Note that $\Sigma_{y_1}\spc{L}$ is isometric to a quotient of $\SSS^3$ by an isometric $\SSS^1$-action without fixed points.
It follows that each angle $\mangle\hinge{y_i}{y_j}{y_k}\le \tfrac\pi2$ for any three distinct points 
$y_i$, $y_j$, $y_k$.
In particular, all four triangles $[y_1y_2y_3]$, $[y_1y_2y_4]$, $[y_1y_3y_4]$, and $[y_2y_3y_4]$ are nondegenerate.
By comparison, the sum of angles in each triangle is strictly bigger than $\pi$.

Denote by $\sigma$ the sum of all 12 angles in 4 triangles $[y_1y_2y_3]$, $[y_1y_2y_4]$, $[y_1y_3y_4]$, and $[y_2y_3y_4]$.
From above,
\[\sigma>4\cdot\pi.\]

On the other hand, by \ref{ex:S^3/S^1} any triangle in $\Sigma_{y_1}\spc{L}$ has perimeter at most $\pi$.
In particular, 
\[\mangle\hinge{y_1}{y_2}{y_3}+\mangle\hinge{y_1}{y_3}{y_4}+\mangle\hinge{y_1}{y_4}{y_2}\le \pi.\]
Applying the same argument in $\Sigma_{y_2}\spc{L}$, $\Sigma_{y_3}\spc{L}$, and $\Sigma_{y_4}\spc{L}$, we get 
\[\sigma\le 4\cdot\pi\]
--- a contradiction.

\parit{Case 2.}
Let $F$ contains one surface $S$.
Note that the projection of $S$ to $\spc{L}$ forms its boundary $\partial \spc{L}$.
Note that doubling $\hat {\spc{L}}$ of $\spc{L}$ across its boundary has 4 singular points --- each singular point of $\spc{L}$ corresponds to two singular points of $\hat {\spc{L}}$.

Latter we will show that the doubling is a geodesic $\CBB(1)$ space;
this is the so called \emph{Doubling theorem}.
Therefore we arrive at a contradiction the same way as in the first case.

\parit{Case 3.}
Suppose $F$ contains at least two surfaces.
Then $\partial\spc{L}$ has at least two connected components; choose two of them $A$ and $B$.
Denote by $\gamma$ a geodesic that minimizes the distance from $A$ to $B$.

Let 
\[\dots \spc{L}_{-1},\spc{L}_{0},\spc{L}_{1},\dots\] two-side infinite sequence of copies on $\partial\spc{L}$.
Let us glue $\spc{L}_{i}$ to $\spc{L}_{i+1}$ along $A$ if $i$ is even and along $B$ if $i$ is odd.
Every point in the obtained space $\spc{N}$ has a neighborhood that is isometric to a neighborhood of the corresponding point in $\spc{L}$ or its doubling.
By the globalization theorem, $\spc{N}$ is $\CBB(1)$.

Note that the copies of $\gamma$ in $\spc{L}_{i}$ form a line in $\spc{N}$.
By the line splitting theorem, $\spc{N}$ is isometric to a product $\spc{N}'\oplus \RR$.
The latter is impossible for $\CBB(1)$ space --- a contradiction.
(Here we used that dimension of $\spc{N}$ is bigger than 1.
According to our definitions, $\RR$ is $\CBB(1)$;
it splits trivially, but such examples exist only in dimension~1.)
\qeds





