\chapter{CBB: first steps}


In this lecture, we start to study metric spaces that satisfy $\CBB$ comparison [see \ref{def:CBB}].
Most of the covered material
will not be used further, it served as a motivation for $\CBB$ comparison.

\section{Quotients and submetries}

\begin{thm}{Theorem}\label{thm:CBB/G}
Assume that group $G$ acts isometrically on a $\Alex0$ space $\spc{L}$ and has closed orbits.
Then the quotient space $\spc{L}/G$ is $\Alex0$.

\end{thm}

\parit{Proof.}
Denote by $\sigma\:\spc{L}\to \spc{L}/G$ the quotient map.

Fix a quadruple of points $p,x_1,x_2,x_3\in \spc{L}/G$.
Choose an arbitrary $\hat p\in \spc{L}$ such that $\sigma(\hat{p})=p$.
Note that we can choose the points $\hat{x}_1,\hat{x}_2,\hat{x}_3\in \spc{L}$ such that $\sigma(\hat x_i)=x_i$ and
\[\dist{p}{x_i}{\spc{L}/G}
\lege
\dist{\hat{p}}{\hat{x}_i}{\spc{L}}
\pm\delta\]
for all $i$ and any fixed $\delta>0$.

Given $\eps>0$, the value $\delta$ can be chosen in such a way that the inequality
\[\angk p{x_i}{x_j}
<
\angk {\hat{p}}{\hat{x}_i}{\hat{x}_j}+\eps
\eqlbl{eq:angles-M-L}\]
holds for all $i$ and $j$.

By $\Alex0$ comparison in $\spc{L}$,
we have
\[\angk {\hat{p}}{\hat{x}_1}{\hat{x}_2}
+\angk {\hat{p}}{\hat{x}_2}{\hat{x}_3}
+\angk {\hat{p}}{\hat{x}_3}{\hat{x}_1}
\le 
2\cdot\pi.\]
Applying  \ref{eq:angles-M-L}, 
we get 
\[\angk p{x_1}{x_2}
+\angk p{x_2}{x_3}
+\angk p{x_3}{x_1}< 2\cdot\pi+3\cdot\eps.\]
Since $\eps>0$ is arbitrary we have 
\[\angk p{x_1}{x_2}
+\angk p{x_2}{x_3}
+\angk p{x_3}{x_1}\le 2\cdot\pi;\]
that is,
the $\Alex0$ comparison holds for this quadruple in $\spc{L}/G$.
\qeds

A map $\sigma\:\spc{X}\to\spc{Y}$ between the metric spaces $\spc{X}$ and $\spc{Y}$
is called a \index{submetry}\emph{submetry} if 
\[\sigma(\oBall(p,r)_\spc{X})=\oBall(\sigma(p),r)_{\spc{Y}}\]
for any $p\in \spc{X}$ and $r\ge 0$.

Suppose $G$ and $\spc{L}$ are as in \ref{thm:CBB/G}.
Observe that the quotient map $\sigma\:\spc{L}\to \spc{L}/G$ is a submetry.
Moreover, the proof above works for any submetry.
Therefore we get the following.

\begin{thm}{Generalization}\label{thm:submetry-CBB}
Let $\sigma\:\spc{L}\to\spc{M}$ be a submetry.
Suppose $\spc{L}$ is a $\Alex{0}$ space, then so is $\spc{M}$.
\end{thm}

\begin{thm}{Advanced exercise}
Let $G$ be a compact Lie group with a bi-invariant Riemannian metric.
Show that $G$ is isometric to a quotient of the Hilbert space by isometric group action.

Conclude that $G\in\CBB(0)$.
\end{thm}






