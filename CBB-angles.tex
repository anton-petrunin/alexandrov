\chapter{Quotients}

\section{Quotient space}

Suppose that group $G$ acts isometrically on a metric space $\spc{X}$.
Note that
\[\dist{G\cdot x}{G\cdot y}{\spc{X}/G}
\df
\inf
\set{\dist{x}{g\cdot y}{\spc{X}}}{g\in G}\]
defines a semimetric on the orbit space $\spc{X}/G$.
Moreover, it is a genuine metric if the orbits of the action are closed.

\begin{thm}{Theorem}\label{thm:CBB/G}
Assume that group $G$ acts isometrically on a proper $\Alex0$ space $\spc{L}$ and has closed orbits.
Then the quotient space $\spc{L}/G$ is $\Alex0$.

\end{thm}

\parit{Proof.}
Denote by $\sigma\:\spc{L}\to \spc{L}/G$ the quotient map.

Fix a quadruple of points $p,x_1,x_2,x_3\in \spc{L}/G$.
Choose an arbitrary $\hat p\in \spc{L}$ such that $\sigma(\hat{p})=p$.
Since $\spc{L}$ is proper, we can choose the points $\hat{x}_i\in \spc{L}$ such that $\sigma(\hat x_i)=x_i$ and
\[\dist{p}{x_i}{\spc{L}/G}
=
\dist{\hat{p}}{\hat{x}_i}{\spc{L}}\]
for all $i$.

Note that 
\[\dist{x_i}{x_j}{\spc{L}/G}
\le 
\dist{\hat{x}_i}{\hat{x}_j}{\spc{L}}
\]
for all $i$ and $j$.
Therefore 
\[\angk p{x_i}{x_j}
\le
\angk {\hat{p}}{\hat{x}_i}{\hat{x}_j}
\eqlbl{eq:angles-M-L}\]
holds for all $i$ and $j$.

By $\Alex0$ comparison in $\spc{L}$,
we have
\[\angk {\hat{p}}{\hat{x}_1}{\hat{x}_2}
+\angk {\hat{p}}{\hat{x}_2}{\hat{x}_3}
+\angk {\hat{p}}{\hat{x}_3}{\hat{x}_1}
\le 
2\cdot\pi.\]
Applying  \ref{eq:angles-M-L}, 
we get 
\[\angk p{x_1}{x_2}
+\angk p{x_2}{x_3}
+\angk p{x_3}{x_1}< 2\cdot\pi;\]
that is,
the $\Alex0$ comparison holds for any quadruple in $\spc{L}/G$.
\qeds

\begin{thm}{Advanced exercise}
Let $G$ be a compact Lie group with a bi-invariant Riemannian metric.
Show that $G$ is isometric to a quotient of the Hilbert space by an isometric group action.

Conclude that $G\in\CBB(0)$.
\end{thm}

\section{Generalizations}

A map $\sigma\:\spc{X}\to\spc{Y}$ between the metric spaces $\spc{X}$ and $\spc{Y}$
is called a \index{submetry}\emph{submetry} if 
\[\sigma(\oBall(p,r)_\spc{X})=\oBall(\sigma(p),r)_{\spc{Y}}\]
for any $p\in \spc{X}$ and $r\ge 0$.

Suppose $G$ and $\spc{L}$ are as in \ref{thm:CBB/G}.
Observe that the quotient map $\sigma\:\spc{L}\to \spc{L}/G$ is a submetry.
The following two exercises show that this is not the only source of submetries. 

\begin{thm}{Exercise}\label{ex:sumbetries(S^2)}
Construct submetries
\begin{subthm}{}
$\sigma_1\:\mathbb{S}^2\to[0,\pi]$
\end{subthm}
\begin{subthm}{}
$\sigma_2\:\mathbb{S}^2\to[0,\tfrac\pi2]$
\end{subthm}
\begin{subthm}{}
$\sigma_n\:\mathbb{S}^2\to[0,\tfrac\pi n]$ (for integer $n\ge 1$)
\end{subthm}
such that the fibers $\sigma_n^{-1}\{x\}$ are connected and have empty interior for any $x$.
\end{thm}

\begin{thm}{Exercise}
Let $\sigma\:\EE^2\to [0,\infty)$ is a submetry.
Show that $K\z=\sigma^{-1}\{0\}$ is a closed convex set in $\EE^2$ and $\sigma(x)=\distfun_Kx$.
\end{thm}

The proof of \ref{thm:CBB/G} works for submetries.
Therefore we get the following.

\begin{thm}{Generalization}\label{thm:submetry-CBB}
Let $\sigma\:\spc{L}\to\spc{M}$ be a submetry.
Suppose $\spc{L}$ is a $\Alex{0}$ space, then so is $\spc{M}$.
\end{thm}

Theorems \ref{thm:CBB/G} and \ref{thm:submetry-CBB} admit straightforward generalizations to $\CBB(-1)$ spaces.
In the $\CBB(1)$ the proof produces a slightly weaker statement ---  \textit{any open $\tfrac\pi2$-ball in the quotient of $\CBB(1)$ is $\CBB(1)$};
in particular, the quotient space is \textit{locally} $\CBB(1)$.
If the space is geodesic, then the globalization theorem implies that it is globally  $\CBB(1)$.
The same holds for the targets of submetry from a  $\CBB(1)$ space.
In other words, if $\spc{L}$ geodesic spaces, then analogs of \ref{thm:CBB/G} and \ref{thm:submetry-CBB} hold for $\CBB(\kappa)$ spaces with arbitrary $\kappa$.

\section{Hopf's conjecture}

Recall that Hopf's conjecture states that \textit{$\mathbb{S}^2\times\mathbb{S}^2$ does not admit a Riemannian metric with positive sectional curvature}.
The following partial result was obtained by Wu-Yi Hsiang and Bruce Kleiner \cite{hsiang-kleiner}.

\begin{thm}{Theorem}\label{thm:hsiang-kleiner}
There is no Riemannian metric on $\SSS^2\times\SSS^2$ with sectional curvature $\ge 1$ and nontrivial isometric $\SSS^1$-action.
\end{thm}

We will give a ruf sketch that will use many statements and notions that are not yet introduced.
Nevertheless, it should be possible to follow the proof with an intuitive understanding of the notions (boundary and dimension of a quotient space).
The statements include 
\begin{itemize}
\item The \emph{Toponogov comparison theorem} and the \emph{globalization theorem}.
The former says  that \textit{a complete Riemannian manifold has sectional curvature $\ge \kappa$ if and only if the corresponding metric space is $\CBB(\kappa)$}.
The later says that \textit{a complete geodesic locally $\CBB(\kappa)$ space is $\CBB(\kappa)$}.
\item \emph{Doubling theorem:} \textit{Doubling of a finite-dimensional geodesic $\CBB(\kappa)$ space is a geodesic $\CBB(\kappa)$ space}
\item \emph{Splitting theorem:} \textit{A geodesic $\CBB(0)$ space that contains a line splits isometrically as $\RR\times \spc{X}$};
here the \emph{line} is a both-sided infinite geodesic.
\end{itemize}

In addition, we will use the following exercise that will be proved latter:


\begin{thm}{Exercise}\label{ex:S^3/S^1}
Suppose $\SSS^1\acts\SSS^3$ is an isometric action without fixed points 
and $\Sigma=\SSS^3/\SSS^1$ is its quotient space.
Then there is a distance noncontracting map $\Sigma\to \tfrac12\cdot \SSS^2$, where $\tfrac12\cdot \SSS^2$ is the standard 2-sphere rescaled with a factor $\tfrac12$.
\end{thm}


\parit{Sketch of \ref{thm:hsiang-kleiner}.}
Assume $\spc{M}=(\SSS^2\times\SSS^2,g)$ is a counterexample.
By Toponogov theorem, $\spc{M}$ is $\CBB(1)$.
By \ref{thm:CBB/G}, the quotient space $\spc{L}\z=\spc{M}/\SSS^1$ is $\CBB(1)$;
evidently, $\spc{L}$ is 3-dimensional.

Denote by $F\subset \spc{M}$ the fixed point set of the $\SSS^1$-action.
Each connected component of $F$ is either an isolated point or a 2-dimensional geodesic submanifold in $\spc{M}$;
the latter has to have positive curvature and therefore it is either $\SSS^2$ or $\RP^2$.
Notice that 
\begin{itemize}
 \item each isolated point contributes 1 to the Euler characteristic of~$\spc{M}$,
 \item each sphere contributes 2 to the Euler characteristic of $\spc{M}$, and
 \item each projective plane contributes 1 to the Euler characteristic of~$\spc{M}$.
\end{itemize}
Since $\chi(\spc{M})=4$, we are in one of the following three cases:
\begin{itemize}
 \item $F$ has exactly 4 isolated points,
 \item $F$ has one 2-dimensional submanifold and at least 2 isolated points,
 \item $F$ has at least two 2-dimensional submanifolds.
\end{itemize}
Each case is covered separately.

\parit{Case 1.}
Suppose $F$ has exactly 4 isolated points $x_1$, $x_2$, $x_3$, and $x_4$.
Denote by $y_1$, $y_2$, $y_3$, and $y_4$ the corresponding points in $\spc{L}$.
Note that $\Sigma_{y_i}\spc{L}$ is isometric to a quotient of $\SSS^3$ by an isometric $\SSS^1$-action without fixed points.

By \ref{ex:S^3/S^1}, each angle $\mangle\hinge{y_i}{y_j}{y_k}\le \tfrac\pi2$ for any three distinct points 
$y_i$, $y_j$, $y_k$.
In particular, all four triangles $[y_1y_2y_3]$, $[y_1y_2y_4]$, $[y_1y_3y_4]$, and $[y_2y_3y_4]$ are nondegenerate.
By the comparison, the sum of angles in each triangle is strictly greater than $\pi$.

Denote by $\sigma$ the sum of all 12 angles in 4 triangles $[y_1y_2y_3]$, $[y_1y_2y_4]$, $[y_1y_3y_4]$, and $[y_2y_3y_4]$.
From above,
\[\sigma>4\cdot\pi.\]

On the other hand, by \ref{ex:S^3/S^1} any triangle in $\Sigma_{y_1}\spc{L}$ has perimeter at most $\pi$.
In particular, 
\[\mangle\hinge{y_1}{y_2}{y_3}+\mangle\hinge{y_1}{y_3}{y_4}+\mangle\hinge{y_1}{y_4}{y_2}\le \pi.\]
Apply the same argument in $\Sigma_{y_2}\spc{L}$, $\Sigma_{y_3}\spc{L}$, and $\Sigma_{y_4}\spc{L}$.
Adding the results we get 
\[\sigma\le 4\cdot\pi\]
--- a contradiction.

\parit{Case 2.}
Let $F$ contain one surface $S$.
Note that the projection of $S$ to $\spc{L}$ forms its boundary $\partial \spc{L}$.
Note that doubling $\hat {\spc{L}}$ of $\spc{L}$ across its boundary has 4 singular points --- each singular point of $\spc{L}$ corresponds to two singular points of $\hat {\spc{L}}$.

By the \emph{Doubling theorem}, $\hat {\spc{L}}$ is a geodesic $\CBB(1)$ space.
Therefore we arrive at a contradiction the same way as in the first case.

\parit{Case 3.}
Suppose $F$ contains at least two surfaces.
Then $\partial\spc{L}$ has at least two connected components; choose two of them $A$ and $B$.
Denote by $\gamma$ a geodesic that minimizes the distance from $A$ to $B$.

Let 
\[\dots,\spc{L}_{-1},\spc{L}_{0},\spc{L}_{1},\dots\]
be a two-side infinite sequence of copies on $\partial\spc{L}$.
Let us glue $\spc{L}_{i}$ to $\spc{L}_{i+1}$ along $A$ if $i$ is even and along $B$ if $i$ is odd.

By the \emph{Doubling theorem}, every point in the obtained space $\spc{N}$ has a neighborhood that is isometric to a neighborhood of the corresponding point in $\spc{L}$ or its doubling.
By the \emph{globalization theorem}, $\spc{N}$ is $\CBB(1)$.

Note that the copies of $\gamma$ in $\spc{L}_{i}$ form a line in $\spc{N}$.
By the \emph{splitting theorem}, $\spc{N}$ is isometric to a product $\spc{N}'\oplus \RR$.
The latter is impossible for a $\CBB(1)$ space --- a contradiction.
(Here we used that the dimension of $\spc{N}$ is bigger than 1.
According to our definitions, $\RR$ is $\CBB(1)$;
it splits trivially, but such examples exist only in dimension~1.)
\qeds


\section{Erd\H{o}s' problem rediscovered}

A point $p$ in a $\CBB(\kappa)$ space is called \emph{extremal} if $\mangle\hinge pxy$ for any hinge $\hinge pxy$ with vertex at $p$. 

\begin{thm}{Theorem}\label{thm:extr-point}
Let $\spc{L}$ is an $m$-dimensional $\CBB(0)$ space.
Then it has at most $2^m$ one-point extremal sets.
\end{thm}


The proof is a translation of proof of classical problem in combinatoric geometry to Alexandrov's language.

\begin{thm}{Erd\H{o}s' problem}
Let $F$ be a set of points in $\EE^m$ such that any triangle formed by three distinct points in $F$ has no obtuse angles.
Then number of elements in $F$ can not exeed $2^m$.

Moreover, if $|F|=2^m$ then $F$ consists of vertexes of right parallelepiped.
\end{thm}

This problem was posed by Paul Erd\H{o}s  \cite{erdos} and solved by Ludwig Danzer and Branko Gr\"unbaum \cite{danzer-gruenbaum};
they also showed that in case of equality the points in $F$ have to be placed in vertices of $m$-dimensional rectangle.
Grigori Perelman noticed that after proper definitions, the same proof works in Alexandrov spaces \cite{perelman-Erdos}; so it proves \ref{thm:extr-point}.

\parit{Proof of \ref{thm:extr-point}.}
Let $\{p_i\}$, $i\in\{1,2,\dots,N\}$ be the one-point extreaml sets.
For each $p_i$ consider its open Voronoi domain $V_i$; that is, 
\[V_i=\set{x\in \spc{L}}{\dist{p_i}{x}{}<\dist{p_j}{x}{}\ \text{for any}\ j\not=i}.\]
Clearly $V_i\cap V_j=\emptyset$ if $i\not=j$.
Note that $\vol_mV_i>\frac{1}{2^m}\vol_m \spc{L}$.

Suppose  $0<\alpha<\tfrac12$.
Given a point $x\in\spc{L}$, choose a geodesic $[p_ix]$ and denote by $x_i$ the point on $[p_ix]$ such that $\dist{p_i}{x_i}{}=\alpha\cdot\dist{p_i}{x}{}$;
let $\map_i\:x\to x_i$ be the corresponding map.
By the comparison, 
\[\dist{x_i}{y_i}{}\ge\alpha\cdot \dist{x}{y}{}\]
for any $x,y\in \spc{L}$.
Therefore 
\[\vol(\map_i \spc{L})\ge\alpha^m\cdot\vol \spc{L}.\]

For any $x\in \spc{L}$, we have $x_i\in V_i$.
Indeed, assume $x_i\notin V_i$,
then threre is $p_j$ such that $\dist{p_i}{x_i}{}\ge\dist{p_j}{x_i}{}$.
Then from comparison, we have $\angk{p_j}{p_i}{x}_{\EE^2}>\tfrac\pi2$;
that is, $p_j$ does not form a one-point extremal set.

It follows that $\alpha^m\cdot N\le 1$ for any $0<\alpha<\tfrac12$,
and hence 
\[N\le 2^m.\]
\qedsf

The question of classifying all $m$-dimensional $\CBB(0)$ spaces $\spc{L}$ with the maximal number extremal points turns out to be more delicate.
One such example is the quotient $\TT^m/(\ZZ_2)$, where $\TT^m$ is a flat torus and $\ZZ_2$-action is induced by a reflection across a point;
this action has $2^m$ isolated fixed points; each corresponds to an extremal point in the quotient $\TT^m/(\ZZ_2)$.
It was shown by Nina Lebedeva \cite{lebedeva} that \textit{every $m$-dimensional geodesic $\CBB(0)$ space is a quotient of Euclidean space by a crystallographic action}; see the definition below.

\section{Crystallographic actions}

An isometric action $\Gamma\acts \EE^m$ is called \emph{crystallographic} if it is 
\emph{properly discontinuous} (that is, for any compact set $K\subset \EE^m$ and $x\in \EE^m$ there only finitely many $g\in \Gamma$ such that $g\cdot x\in K$) and \emph{cocompact} (that is, the quotient space $\spc{L}=\EE^m/\Gamma$ is compact).

Let $F$ be a maximal finite subgroup of $\Gamma$
that is, there is no finite group $H$ such that $F<H<\Gamma$.
Denote by $\#(\Gamma)$ the number of maximal finite subgroups of $\Gamma$ up to conjugation.

\begin{thm}{Open question}
Let $\Gamma\acts \EE^m$ be a crystallographic action.
Is it true that $\#(\Gamma)\le 2^m$?
\end{thm}

Note that any finite subgroup of $\Gamma$ fixes an affine subspace $A_F$ in $\EE^m$;
if this subspace is zero-dimnsonal, then we say that $F$ \emph{fixes isolated point}.
Denote by $\#_0(\Gamma)$ the number of maximal finite subgroups of $\Gamma$ that fixes isolated point (up to conjugation).
The following statement gives a partial answer to the question above.

\begin{thm}{Proposition}
Let $\Gamma\acts \EE^m$ be a crystallographic action.
Then $\#_0(\Gamma)\le 2^m$.
\end{thm}

\begin{thm}{Lemma}
Let $\Gamma\acts \EE^m$ be a crystallographic action and $F$ be a maximal finite subgroup of $\Gamma$ that fixes an isolated point $p$.
Then the image of $p$ in the quotient space $\spc{L}=\EE^m/\Gamma$ is an extremal point.
\end{thm}

\parit{Proof.}
Let $q$ be the image of $p$.
Suppose $q$ is not extremal;
that is, $\mangle \hinge qxy>\tfrac\pi2$ for some hinge $\hinge qxy$ in $\spc{L}$.

Choose $x_1,y_1\in \EE^m$ such that $\dist{p}{x_1}{\EE^m}=\dist{q}{x}{\spc{L}}$ and $\dist{p}{y_1}{\EE^m}=\dist{q}{y}{\spc{L}}$.
Note that $\mangle \hinge p{x_1}{y_1}\ge \mangle \hinge qxy>\tfrac\pi2$.
Moreover $\mangle \hinge p{x_1}{g\cdot y_1}\ge \mangle \hinge qxy>\tfrac\pi2$ for any $g\in F$.

Denote by $z$ the barycenter of $g\cdot y_1$
\qeds

The following proposition gives a metric description of $E_F$.
\begin{thm}{Claim}
Let $\Gamma\acts \EE^m$ is a crystallographic action and $F$ be a maximal finite sub
Let $p\in \spc{L}\setminus E_F$
\end{thm}




It follows that the corresponding extremal subset $E_F$ in $\spc{L}=\EE^m/\Gamma$ is proper --- not the whole space nor the empty set.

If $F$ is a maximal finite subgroup then the corresponging extremal set $E_F$ is primitive;
that is $E_F$ contains no other extreaml sets except itself and the empty set.
This disussion leads to the following observation.

\begin{thm}{Observation}
Let $\Gamma\acts \EE^m$ be a crystallographic action on the Euclidean space.
Then number $\#(\Gamma)$ of the maximal finite subgroups of $\Gamma$ up to conjugation 
equals to the number of primitive extreamal sets in the quatient space $\EE^m/\Gamma$.
\end{thm}

It expected that  $\#(\Gamma)\le 2^m$.
According to the observation above the latter would follow if 
the number of primitive extreaml sets in any $m$-dimensional compact length $\CBB{}{0}$ space does not exeed $2^m$.


It follows that in order to estimate number of maximal finite subgroups in $\Gamma$ it is sufficient to estimate the number of primitive extremal sets in an $m$-dimensional compact length $\CBB{}{0}$.

The following theorem was proved in ???.
