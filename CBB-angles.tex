\chapter{Quotients}

\section{Quotient space}

Suppose that a group $G$ acts isometrically on a metric space $\spc{X}$.
Note that
\[\dist{G\cdot x}{G\cdot y}{\spc{X}/G}
\df
\inf
\set{\dist{x}{g\cdot y}{\spc{X}}}{g\in G}\]
defines a semimetric on the orbit space $\spc{X}/G$.
Moreover, it is a genuine metric if the orbits of the action are closed.

\begin{thm}{Theorem}\label{thm:CBB/G}
Assume that group $G$ acts isometrically on a proper $\Alex0$ space $\spc{L}$ and has closed orbits.
Then the quotient space $\spc{L}/G$ is $\Alex0$.

\end{thm}

\parit{Proof.}
Denote by $\sigma\:\spc{L}\to \spc{L}/G$ the quotient map.

Fix a quadruple of points $p,x_1,x_2,x_3\in \spc{L}/G$.
Choose an arbitrary $\hat p\in \spc{L}$ such that $\sigma(\hat{p})=p$.
Since $\spc{L}$ is proper, we can choose the points $\hat{x}_i\in \spc{L}$ such that $\sigma(\hat x_i)=x_i$ and
\[\dist{p}{x_i}{\spc{L}/G}
=
\dist{\hat{p}}{\hat{x}_i}{\spc{L}}\]
for all $i$.

Note that 
\[\dist{x_i}{x_j}{\spc{L}/G}
\le 
\dist{\hat{x}_i}{\hat{x}_j}{\spc{L}}
\]
for all $i$ and $j$.
Therefore 
\[\angk p{x_i}{x_j}
\le
\angk {\hat{p}}{\hat{x}_i}{\hat{x}_j}
\eqlbl{eq:angles-M-L}\]
holds for all $i$ and $j$.

By $\Alex0$ comparison in $\spc{L}$,
we have
\[\angk {\hat{p}}{\hat{x}_1}{\hat{x}_2}
+\angk {\hat{p}}{\hat{x}_2}{\hat{x}_3}
+\angk {\hat{p}}{\hat{x}_3}{\hat{x}_1}
\le 
2\cdot\pi.\]
Applying  \ref{eq:angles-M-L}, 
we get 
\[\angk p{x_1}{x_2}
+\angk p{x_2}{x_3}
+\angk p{x_3}{x_1}< 2\cdot\pi;\]
that is,
the $\Alex0$ comparison holds for any quadruple in $\spc{L}/G$.
\qeds

\begin{thm}{Advanced exercise}
Let $G$ be a compact Lie group with a bi-invariant Riemannian metric.
Show that $G$ is isometric to a quotient of the Hilbert space by an isometric group action.

Conclude that $G\in\CBB(0)$.
\end{thm}

\section{Generalizations}

A map $\sigma\:\spc{X}\to\spc{Y}$ between the metric spaces $\spc{X}$ and $\spc{Y}$
is called a \index{submetry}\emph{submetry} if 
\[\sigma(\oBall(p,r)_\spc{X})=\oBall(\sigma(p),r)_{\spc{Y}}\]
for any $p\in \spc{X}$ and $r\ge 0$.

Suppose $G$ and $\spc{L}$ are as in \ref{thm:CBB/G}.
Observe that the quotient map $\sigma\:\spc{L}\to \spc{L}/G$ is a submetry.
The following two exercises show that this is not the only source of submetries. 

\begin{thm}{Exercise}\label{ex:sumbetries(S^2)}
Construct submetries
\begin{subthm}{}
$\sigma_1\:\mathbb{S}^2\to[0,\pi]$
\end{subthm}
\begin{subthm}{}
$\sigma_2\:\mathbb{S}^2\to[0,\tfrac\pi2]$
\end{subthm}
\begin{subthm}{}
$\sigma_n\:\mathbb{S}^2\to[0,\tfrac\pi n]$ (for integer $n\ge 1$)
\end{subthm}
such that the fibers $\sigma_n^{-1}\{x\}$ are connected and have an empty interior for any $x$.
\end{thm}

\begin{thm}{Exercise}
Let $\sigma\:\EE^2\to [0,\infty)$ be a submetry.
Show that $K\z=\sigma^{-1}\{0\}$ is a closed convex set in $\EE^2$ and $\sigma(x)=\distfun_Kx$.
\end{thm}

The proof of \ref{thm:CBB/G} works for submetries.
Therefore we get the following.

\begin{thm}{Generalization}\label{thm:submetry-CBB}
Let $\sigma\:\spc{L}\to\spc{M}$ be a submetry.
Suppose $\spc{L}$ is a $\Alex{0}$ space, then so is $\spc{M}$.
\end{thm}

Theorems \ref{thm:CBB/G} and \ref{thm:submetry-CBB} admit straightforward generalizations to $\CBB(-1)$ spaces.
In the $\CBB(1)$ case, the proof produces a slightly weaker statement ---  \textit{any open $\tfrac\pi2$-ball in the quotient of $\CBB(1)$ is $\CBB(1)$};
in particular, the quotient space is \textit{locally} $\CBB(1)$.
If the space is geodesic, then the globalization theorem implies that it is globally  $\CBB(1)$.
The same holds for the targets of submetry from a  $\CBB(1)$ space.
In other words, if $\spc{L}$ is a geodesic space, then analogs of \ref{thm:CBB/G} and \ref{thm:submetry-CBB} hold for $\CBB(\kappa)$ spaces with arbitrary $\kappa$.

\section{Hopf's conjecture}

Recall that Hopf's conjecture states that \textit{$\mathbb{S}^2\times\mathbb{S}^2$ does not admit a Riemannian metric with positive sectional curvature}.
The following partial result was obtained by Wu-Yi Hsiang and Bruce Kleiner \cite{hsiang-kleiner}.

\begin{thm}{Theorem}\label{thm:hsiang-kleiner}
There is no Riemannian metric on $\SSS^2\times\SSS^2$ with sectional curvature $\ge 1$ and a nontrivial isometric $\SSS^1$-action.
\end{thm}

We will give a ruf sketch that will use many statements and notions that are not yet introduced.
Nevertheless, it should be possible to follow the proof with an intuitive understanding of the notions (boundary and dimension of a quotient space).
The statements include 
\begin{itemize}
\item The \emph{Toponogov comparison theorem} and the \emph{globalization theorem}.
The former says  that \textit{a complete Riemannian manifold has sectional curvature $\ge \kappa$ if and only if the corresponding metric space is $\CBB(\kappa)$}.
The later says that \textit{a complete geodesic locally $\CBB(\kappa)$ space is $\CBB(\kappa)$}.
\item \emph{Doubling theorem:} \textit{Doubling of a finite-dimensional geodesic $\CBB(\kappa)$ space is a geodesic $\CBB(\kappa)$ space}
\item \emph{Splitting theorem:} \textit{A geodesic $\CBB(0)$ space that contains a line splits isometrically as $\RR\times \spc{X}$};
here the \emph{line} is a both-sided infinite geodesic.
\end{itemize}

In addition, we will use the following exercise that will be proved latter:


\begin{thm}{Exercise}\label{ex:S^3/S^1}
Suppose $\SSS^1\acts\SSS^3$ is an isometric action without fixed points 
and $\Sigma=\SSS^3/\SSS^1$ is its quotient space.
Then there is a distance noncontracting map $\Sigma\to \tfrac12\cdot \SSS^2$, where $\tfrac12\cdot \SSS^2$ is the standard 2-sphere rescaled with a factor $\tfrac12$.
\end{thm}


\parit{Sketch of \ref{thm:hsiang-kleiner}.}
Assume $\spc{M}=(\SSS^2\times\SSS^2,g)$ is a counterexample.
By the Toponogov theorem, $\spc{M}$ is $\CBB(1)$.
By \ref{thm:CBB/G}, the quotient space $\spc{L}\z=\spc{M}/\SSS^1$ is $\CBB(1)$;
evidently, $\spc{L}$ is 3-dimensional.

Denote by $F\subset \spc{M}$ the fixed point set of the $\SSS^1$-action.
Each connected component of $F$ is either an isolated point or a 2-dimensional geodesic submanifold in $\spc{M}$;
the latter has to have positive curvature and therefore it is either $\SSS^2$ or $\RP^2$.
Notice that 
\begin{itemize}
 \item each isolated point contributes 1 to the Euler characteristic of~$\spc{M}$,
 \item each sphere contributes 2 to the Euler characteristic of $\spc{M}$, and
 \item each projective plane contributes 1 to the Euler characteristic of~$\spc{M}$.
\end{itemize}
Since $\chi(\spc{M})=4$, we are in one of the following three cases:
\begin{itemize}
 \item $F$ has exactly 4 isolated points,
 \item $F$ has one 2-dimensional submanifold and at least 2 isolated points,
 \item $F$ has at least two 2-dimensional submanifolds.
\end{itemize}
Each case is covered separately.

\parit{Case 1.}
Suppose $F$ has exactly 4 isolated points $x_1$, $x_2$, $x_3$, and $x_4$.
Denote by $y_1$, $y_2$, $y_3$, and $y_4$ the corresponding points in $\spc{L}$.
Note that $\Sigma_{y_i}\spc{L}$ is isometric to a quotient of $\SSS^3$ by an isometric $\SSS^1$-action without fixed points.

By \ref{ex:S^3/S^1}, each angle $\mangle\hinge{y_i}{y_j}{y_k}\le \tfrac\pi2$ for any three distinct points 
$y_i$, $y_j$, $y_k$.
In particular, all four triangles $[y_1y_2y_3]$, $[y_1y_2y_4]$, $[y_1y_3y_4]$, and $[y_2y_3y_4]$ are nondegenerate.
By the comparison, the sum of angles in each triangle is strictly greater than $\pi$.

Denote by $\sigma$ the sum of all 12 angles in 4 triangles $[y_1y_2y_3]$, $[y_1y_2y_4]$, $[y_1y_3y_4]$, and $[y_2y_3y_4]$.
From above,
\[\sigma>4\cdot\pi.\]

On the other hand, by \ref{ex:S^3/S^1} any triangle in $\Sigma_{y_1}\spc{L}$ has perimeter at most $\pi$.
In particular, 
\[\mangle\hinge{y_1}{y_2}{y_3}+\mangle\hinge{y_1}{y_3}{y_4}+\mangle\hinge{y_1}{y_4}{y_2}\le \pi.\]
Apply the same argument in $\Sigma_{y_2}\spc{L}$, $\Sigma_{y_3}\spc{L}$, and $\Sigma_{y_4}\spc{L}$.
Adding the results we get 
\[\sigma\le 4\cdot\pi\]
--- a contradiction.

\parit{Case 2.}
Let $F$ contain one surface $S$.
Note that the projection of $S$ to $\spc{L}$ forms its boundary $\partial \spc{L}$.
Note that doubling $\hat {\spc{L}}$ of $\spc{L}$ across its boundary has 4 singular points --- each singular point of $\spc{L}$ corresponds to two singular points of $\hat {\spc{L}}$.

By the \emph{Doubling theorem}, $\hat {\spc{L}}$ is a geodesic $\CBB(1)$ space.
Therefore we arrive at a contradiction the same way as in the first case.

\parit{Case 3.}
Suppose $F$ contains at least two surfaces.
Then $\partial\spc{L}$ has at least two connected components; choose two of them $A$ and $B$.
Denote by $\gamma$ a geodesic that minimizes the distance from $A$ to $B$.

Let 
\[\dots,\spc{L}_{-1},\spc{L}_{0},\spc{L}_{1},\dots\]
be a two-sided infinite sequence of copies on $\partial\spc{L}$.
Let us glue $\spc{L}_{i}$ to $\spc{L}_{i+1}$ along $A$ if $i$ is even and along $B$ if $i$ is odd.

By the \emph{Doubling theorem}, every point in the obtained space $\spc{N}$ has a neighborhood that is isometric to a neighborhood of the corresponding point in $\spc{L}$ or its doubling.
By the \emph{globalization theorem}, $\spc{N}$ is $\CBB(1)$.

Note that the copies of $\gamma$ in $\spc{L}_{i}$ form a line in $\spc{N}$.
By the \emph{splitting theorem}, $\spc{N}$ is isometric to a product $\spc{N}'\oplus \RR$.
The latter is impossible for a $\CBB(1)$ space --- a contradiction.
(Here we used that the dimension of $\spc{N}$ is bigger than 1.
According to our definitions, $\RR$ is $\CBB(1)$;
it splits trivially, but such examples exist only in dimension~1.)
\qeds


\section{Erd\H{o}s' problem rediscovered}

A point $p$ in a $\CBB(\kappa)$ space is called \emph{extremal} if $\mangle\hinge pxy\le \tfrac\pi2$ for any hinge $\hinge pxy$ with the vertex at $p$. 

\begin{thm}{Theorem}\label{thm:extr-point}
Let $\spc{L}$ be a compact $m$-dimensional $\CBB(0)$ space.
Then it has at most $2^m$ one-point extremal sets.
\end{thm}


The proof is a translation of the proof of a classical problem in discrete geometry to Alexandrov's language.

\begin{thm}{Erd\H{o}s' problem}
Let $F$ be a set of points in $\EE^m$ such that any triangle formed by three distinct points in $F$ has no obtuse angles.
Then  $|F|\le2^m$.
Moreover, if $|F|=2^m$ then $F$ consists of the vertexes of an $m$-dimensional rectangle.
\end{thm}

This problem was posed by Paul Erd\H{o}s  \cite{erdos} and solved by Ludwig Danzer and Branko Gr\"unbaum \cite{danzer-gruenbaum}.
Grigori Perelman noticed that after proper definitions, the same proof works in Alexandrov spaces \cite{perelman-Erdos}; so it proves \ref{thm:extr-point}.
We will use the notion of volume;
it is not defined so far and should be understood intuitively.

\parit{Proof of \ref{thm:extr-point}.}
Let $\{p_1,\dots,p_N\}$ be extremal points in $\spc{L}$.
For each $p_i$ consider its open \emph{Voronoi domain} $V_i$; that is, 
\[V_i=\set{x\in \spc{L}}{\dist{p_i}{x}{}<\dist{p_j}{x}{}\ \text{for any}\ j\not=i}.\]
Clearly $V_i\cap V_j=\emptyset$ if $i\not=j$.

Suppose  $0<\alpha\le 1$.
Given a point $x\in\spc{L}$, choose a geodesic $[p_ix]$ and denote by $x_i$ the point on $[p_ix]$ such that $\dist{p_i}{x_i}{}=\alpha\cdot\dist{p_i}{x}{}$;
let $\map_i\:x\to x_i$ be the corresponding map.
By the comparison, 
\[\dist{x_i}{y_i}{}\ge\alpha\cdot \dist{x}{y}{}\]
for any $x$, $y$, and $i$.
Therefore 
\[\vol(\map_i \spc{L})\ge\alpha^m\cdot\vol \spc{L}.\]

Suppose $\alpha<\tfrac12$.
Then $x_i\in V_i$ for any $x\in \spc{L}$.
Indeed, assume $x_i\notin V_i$,
then threre is $p_j$ such that $\dist{p_i}{x_i}{}\ge\dist{p_j}{x_i}{}$.
Then from comparison, we have $\angk{p_j}{p_i}{x}_{\EE^2}>\tfrac\pi2$;
that is, $p_j$ does not form a one-point extremal set.
It follows that $\vol V_i\ge\alpha^m\cdot\vol \spc{L}$
for any $0<\alpha<\tfrac12$ and hence 
\[\vol V_i\ge\tfrac1{2^m}\cdot\vol \spc{L}\]
and hence 
\[N\le 2^m.\]
\qedsf

\section{Crystallographic actions}


An isometric action $\Gamma\acts \EE^m$ is called \emph{crystallographic} if it is 
\emph{properly discontinuous} (that is, for any compact set $K\subset \EE^m$ and $x\in \EE^m$ there only finitely many $g\in \Gamma$ such that $g\cdot x\in K$) and \emph{cocompact} (that is, the quotient space $\spc{L}=\EE^m/\Gamma$ is compact).

Let $F$ be a maximal finite subgroup of $\Gamma$;
that is, if $H$ is a finite group $H$ such that $F<H<\Gamma$, then $F=H$.
Denote by $\#(\Gamma)$ the number of maximal finite subgroups of $\Gamma$ up to conjugation.

\begin{thm}{Open question}
Let $\Gamma\acts \EE^m$ be a crystallographic action.
Is it true that $\#(\Gamma)\le 2^m$?
\end{thm}

Note that any finite subgroup $F$ of $\Gamma$ fixes an affine subspace $A_F$ in $\EE^m$.
If $F$ is maximal, then $A_F$ completely describes $F$.
Denote by $\#_k(\Gamma)$ the number of maximal finite subgroups $F<\Gamma$ (up to conjugation) such that $\dim A_F=k$.

Choose a finite subgroup $F<\Gamma$; consider a conjugate subgroup $F'=g \cdot F \cdot g^{-1}$.
Note that $A_{F'}=g\cdot A_F$.
In particular, the subspaces $A_F$ and $A_{F'}$ have the same image in the quotient space $\spc{L}=\EE^m/\Gamma$.
It follows that to count subgroups up to conjugation, we need to count the images of their fixed set.
Therefore, by the lemma below, $\#_0(\Gamma)$ cannot exceed the number of extremal points in $\spc{L}=\EE^m/\Gamma$.
Combining this observation with \ref{thm:extr-point}, we get the following.

\begin{thm}{Proposition}\label{prop:2m}
Let $\Gamma\acts \EE^m$ be a crystallographic action.
Then $\#_0(\Gamma)\le 2^m$.
\end{thm}

\begin{thm}{Lemma}
Let $\Gamma\acts \EE^m$ be a crystallographic action and $F$ be a maximal finite subgroup of $\Gamma$ that fixes an isolated point $p$.
Then the image of $p$ in the quotient space $\spc{L}=\EE^m/\Gamma$ is an extremal point.
\end{thm}

\parit{Proof.}
Let $q$ be the image of $p$.
Suppose $q$ is not extremal;
that is, $\mangle \hinge q{y_1}{y_2}>\tfrac\pi2$ for some hinge $\hinge q{y_1}{y_2}$ in $\spc{L}$.

Choose the inverse images $x_1,x_2\in \EE^m$ of $y_1,y_2\in \spc{L}$ such that $\dist{p}{x_i}{\EE^m}=\dist{q}{y_i}{\spc{L}}$.
Note that $\mangle \hinge p{x_1}{x_2}\ge \mangle \hinge q{y_1}{y_2}>\tfrac\pi2$.
Moreover, since $p$ is fixed by $F$, we have
\[\mangle \hinge p{x_1}{g\cdot x_2}>\tfrac\pi2
\eqlbl{eq:>pi/2}\]
for any $g\in F$.

Denote by $z$ the barycenter of the orbit $G\cdot x_2$.
Note that $z$ is a fixed point of $F$.
By \ref{eq:>pi/2}, $z\ne p$;
so $F$ must fix the line $pz$.
But $p$ is an isolated fixed point of $F$ --- a contradiction.
\qeds

\begin{thm}{Exercise}\label{ex:number(m-1)}
Let $\Gamma\acts \EE^m$ be a crystallographic action.
Apply the theorems used in \ref{thm:hsiang-kleiner} to show that
\begin{subthm}{ex:number(m-1):2}
$\#_{m-1}(\Gamma)\le 2$, and
\end{subthm}

\begin{subthm}{ex:number(m-1):1}
if $\#_{m-1}(\Gamma)=1$, then $\#_0(\Gamma)\le 2^{m-1}$.
\end{subthm}

Construct  crystallographic actions with equalities in \ref{SHORT.ex:number(m-1):2} and \ref{SHORT.ex:number(m-1):1}.
\end{thm}

\section{Remarks}

It is expected that \textit{no geodesic $\CBB(1)$ space with a nontrivial isometric $\SSS^1$-action can be homeomorphic to $\SSS^2\times\SSS^2$};
so \ref{thm:hsiang-kleiner} holds for general $\CBB(1)$ space.
The proof of \ref{thm:hsiang-kleiner} would work if we had the following generalization of \ref{ex:S^3/S^1};
see \cite{harvey-searle}.

\begin{thm}{Conjecture}
Let $\Sigma$ be a geodesic $\CBB(1)$ space homeomorphic to $\SSS^3$.
Suppose $\SSS^1$ acts on $\Sigma$ isometrically.
Then any triangle in $\Sigma/\SSS^1$ has perimeter at most $\pi$.
\end{thm}


Compact geodesic $m$-dimensional $\CBB(0)$ spaces with the maximal number of extremal points include $m$-dimensional rectangles and the quotients of flat tori by reflections across a point.
(This action has $2^m$ isolated fixed points; each corresponds to an extremal point in the quotient space $\spc{L}=\TT^m/\ZZ_2$.)
Nina Lebedeva has proved \cite{lebedeva} that \textit{every $m$-dimensional geodesic $\CBB(0)$ space with $2^m$ extremal points is a quotient of Euclidean space by a crystallographic action}.

Counting maximal finite subgroups in a crystallographic group $\Gamma$ is equivalent to counting the so-called primitive extremal subsets in the quotient space $\spc{L}=\EE^m/\Gamma$.
So, \ref{prop:2m} would follow from the next conjecture.

\begin{thm}{Conjecture}
Any $m$-dimensional compact geodesic $\CBB(0)$ space has at most $2^m$ primitive extremal subset.
\end{thm}

A closed subset $E$ in a $\CBB(\kappa)$ space is called 
\emph{extremal} if $\mangle\hinge pxy\le \tfrac\pi2$ for any $x\notin E$ and $p\in E$ such that $\dist{x}{p}{}$ takes minimal value.
An extremal subset is called primitive if it contains no proper \emph{extremal} subsets.
For example, the whole space and empty set are aslo extremal in any space.
Also every vertex, edge, or face (as well as their union) of the cube is an extremal subset of the cube.
Vertices of the cube are the only its primitive extremal subsets.

