\chapter{CBB: first steps}


In this lecture, we start to study metric spaces that satisfy $\CBB$ comparison [see \ref{def:CBB}].
Most of the covered material
will not be used further, it served as a motivation for $\CBB$ comparison.

\section{Quotients and submetries}

\begin{thm}{Theorem}\label{thm:CBB/G}
Assume that group $G$ acts isometrically on a $\Alex0$ space $\spc{L}$ and has closed orbits.
Then the quotient space $\spc{L}/G$ is $\Alex0$.

\end{thm}

\parit{Proof.}
Denote by $\sigma\:\spc{L}\to \spc{L}/G$ the quotient map.

Fix a quadruple of points $p,x_1,x_2,x_3\in \spc{L}/G$.
Choose an arbitrary $\hat p\in \spc{L}$ such that $\sigma(\hat{p})=p$.
Note that we can choose the points $\hat{x}_1,\hat{x}_2,\hat{x}_3\in \spc{L}$ such that $\sigma(\hat x_i)=x_i$ and
\[\dist{p}{x_i}{\spc{L}/G}
\lege
\dist{\hat{p}}{\hat{x}_i}{\spc{L}}
\pm\delta\]
for all $i$ and any fixed $\delta>0$.

Given $\eps>0$, the value $\delta$ can be chosen in such a way that the inequality
\[\angk p{x_i}{x_j}
<
\angk {\hat{p}}{\hat{x}_i}{\hat{x}_j}+\eps
\eqlbl{eq:angles-M-L}\]
holds for all $i$ and $j$.

By $\Alex0$ comparison in $\spc{L}$,
we have
\[\angk {\hat{p}}{\hat{x}_1}{\hat{x}_2}
+\angk {\hat{p}}{\hat{x}_2}{\hat{x}_3}
+\angk {\hat{p}}{\hat{x}_3}{\hat{x}_1}
\le 
2\cdot\pi.\]
Applying  \ref{eq:angles-M-L}, 
we get 
\[\angk p{x_1}{x_2}
+\angk p{x_2}{x_3}
+\angk p{x_3}{x_1}< 2\cdot\pi+3\cdot\eps.\]
Since $\eps>0$ is arbitrary we have 
\[\angk p{x_1}{x_2}
+\angk p{x_2}{x_3}
+\angk p{x_3}{x_1}\le 2\cdot\pi;\]
that is,
the $\Alex0$ comparison holds for this quadruple in $\spc{L}/G$.
\qeds

A map $\sigma\:\spc{X}\to\spc{Y}$ between the metric spaces $\spc{X}$ and $\spc{Y}$
is called a \index{submetry}\emph{submetry} if 
\[\sigma(\oBall(p,r)_\spc{X})=\oBall(\sigma(p),r)_{\spc{Y}}\]
for any $p\in \spc{X}$ and $r\ge 0$.

Suppose $G$ and $\spc{L}$ are as in \ref{thm:CBB/G}.
Observe that the quotient map $\sigma\:\spc{L}\to \spc{L}/G$ is a submetry.
Moreover, the proof above works for any submetry.
Therefore we get the following.

\begin{thm}{Generalization}\label{thm:submetry-CBB}
Let $\sigma\:\spc{L}\to\spc{M}$ be a submetry.
Suppose $\spc{L}$ is a $\Alex{0}$ space, then so is $\spc{M}$.
\end{thm}

\begin{thm}{Advanced exercise}
Let $G$ be a compact Lie group with a bi-invariant Riemannian metric.
Show that $G$ is isometric to a quotient of the Hilbert space by isometric group action.

Conclude that $G\in\CBB(0)$.
\end{thm}



\section{Angles}\label{sec:angles}


The angle measure of a hinge $\hinge p x y$ is defined as the following limit
\[\mangle\hinge p x y=\lim_{\bar x,\bar y\to p} \angk p{\bar x}{\bar y},\]
where $\bar x\in\left]p x\right]$ and $\bar y\in\left]p y\right]$.

Note that if $\mangle\hinge p x y$ is defined, then
\[0\le \mangle\hinge p x y\le \pi.\]

\begin{thm}{Exercise}
Suppose that in the above definition, one uses spherical or hyperbolic model angles instead of Euclidean.
Show that it does not change the value $\mangle\hinge p x y$.
\end{thm}


\begin{thm}{Exercise}\label{ex:undefined-angle}
Give an example of a hinge $\hinge p x y$ in a metric space with an undefined angle $\mangle\hinge p x y$.
\end{thm}

\begin{thm}{Triangle inequality for angles}
\label{claim:angle-3angle-inq}
Let  $[px_1]$, $[px_2]$, and $[px_3]$ be three geodesics in a metric space.
If all of the angles $\alpha_{i j}=\mangle\hinge p {x_i}{x_j}$ are defined then they satisfy the triangle inequality:
\[\alpha_{13}\le \alpha_{12}+\alpha_{23}.\]

\end{thm}


\parit{Proof.}
Since $\alpha_{13}\le\pi$, we can assume that $\alpha_{12}+\alpha_{23}< \pi$.
Denote by $\gamma_i$ the unit-speed parametrization of $[px_i]$ from $p$ to $x_i$.
Given any $\eps>0$, for all sufficiently small $t,\tau,s\in\RR_{\ge0}$ we have
\begin{align*}
\dist{\gamma_1(t)}{\gamma_3(\tau)}{}
&\le 
\dist{\gamma_1(t)}{\gamma_2(s)}{}+\dist{\gamma_2(s)}{\gamma_3(\tau)}{}<\\
&<
\sqrt{t^2+s^2-2\cdot t\cdot  s\cdot \cos(\alpha_{12}+\eps)} +
\\
&\quad+\sqrt{s^2+\tau^2-2\cdot s\cdot \tau\cdot \cos(\alpha_{23}+\eps)}\le
\end{align*}

\begin{wrapfigure}{o}{30 mm}
\vskip-6mm
\centering
\includegraphics{mppics/pic-615}
\vskip3mm
\end{wrapfigure}

Below we define 
$s(t,\tau)$ so that for 
$s=s(t,\tau)$, this chain of inequalities can be continued as follows:
\[\le
\sqrt{t^2+\tau^2-2\cdot t\cdot \tau\cdot \cos(\alpha_{12}+\alpha_{23}+2\cdot \eps)}.
\]

Thus for any $\eps>0$, 
\[\alpha_{13}\le \alpha_{12}+\alpha_{23}+2\cdot \eps.\]
Hence the result follows.

To define $s(t,\tau)$, consider three half-lines $\tilde \gamma_1$, $\tilde \gamma_2$, $\tilde \gamma_3$ on a Euclidean plane starting at one point, such that
$\mangle(\tilde \gamma_1,\tilde \gamma_2)\z=\alpha_{12}+\eps$,
$\mangle(\tilde \gamma_2,\tilde \gamma_3)\z=\alpha_{23}+\eps$,
and $\mangle(\tilde \gamma_1,\tilde \gamma_3)\z=\alpha_{12}\z+\alpha_{23}\z+2\cdot \eps$.
We parametrize each half-line by the distance from the starting point.
Given two positive numbers $t,\tau\in\RR_{\ge0}$, let $s=s(t,\tau)$ be 
the number such that 
$\tilde \gamma_2(s)\in[\tilde \gamma_1(t)\ \tilde \gamma_3(\tau)]$. 
Clearly, $s\le\max\{t,\tau\}$, so $t,\tau,s$ may be taken sufficiently small.
\qeds 

\begin{thm}{Exercise}\label{ex:adjacent-angles}
Prove that the sum of adjacent angles is at least $\pi$.

More precisely: suppose two hinges $\hinge pxz$ and $\hinge pyz$ are \index{adjacent hinges}\emph{adjacent};
that is, they share side $[pz]$, and the union of two sides $[px]$ and $[py]$ form a geodesic $[xy]$.
Show that
\[\mangle\hinge pxz+\mangle\hinge pyz\ge \pi\]
whenever  each angle on the left-hand side is defined.
\end{thm}

The above inequality can be strict.
For example, in a metric tree angles between any two different edges coming out of the same vertex are all equal to $\pi$.

\begin{thm}{Exercise}\label{ex:first-var}
Assume that a hinge $\hinge q p x$ with defined angle measure.
Let $\gamma$ be the unit speed parametrization of $[qx]$ from $q$ to $x$.
Show that
\[\dist{p}{\gamma(t)}{}
\le
\dist{q}{p}{}-t\cdot \cos(\mangle\hinge q p x)+o(t).\]

\end{thm}

\section{Alexandrov's lemma}

Recall that $[xy]$ denotes a geodesic from $x$ to $y$;
set  
\[
\left]x y\right]=[xy]\setminus\{x\},
\quad
\left[x y\right[=[xy]\setminus\{y\},
\quad
\left]x y\right[=[xy]\setminus\{x,y\}.\]

\begin{thm}{Lemma}
\index{Alexandrov's lemma}
\label{lem:alex}  
Let $p,x,y,z$ be distinct points in a metric space such that $z\in \left]x y\right[$.
Then 
the following expressions for the Euclidean model angles have the same sign:

\begin{subthm}{lem-alex-difference}
$\angk x p y
-\angk x p z$,
\end{subthm} 

\begin{subthm}{lem-alex-angle}
$\angk z p x
+\angk z p y -\pi$.
\end{subthm}

\begin{wrapfigure}{r}{25mm}
\vskip-6mm
\centering
\includegraphics{mppics/pic-730}
\end{wrapfigure}

The same holds for the hyperbolic and spherical model angles, 
but in the latter case, one has to assume in addition that
\[\dist{p}{z}{}+\dist{p}{y}{}+\dist{x}{y}{}< 2\cdot\pi.\]

\end{thm}


\parit{Proof.} 
Consider the model triangle $\trig{\tilde x}{\tilde p}{\tilde z}=\modtrig(x p z)$.
Take 
a point $\tilde y$ on the extension of 
$[\tilde x \tilde z]$ beyond $\tilde z$ so that $\dist{\tilde x}{\tilde y}{}=\dist{x}{y}{}$ (and therefore $\dist{\tilde x}{\tilde z}{}=\dist{x}{z}{}$). 

\begin{wrapfigure}{r}{33mm}
\vskip-0mm
\centering
\includegraphics{mppics/pic-740}
\end{wrapfigure}

Since increasing the opposite side in a plane triangle increases the corresponding angle, 
the following expressions have the same sign:
\begin{enumerate}[(i)]
\item $\mangle\hinge{\tilde x}{\tilde p}{\tilde y}-\angk{x}{p}{y}$,
\item $\dist{\tilde p}{\tilde y}{}-\dist{p}{y}{}$,
\item $\mangle\hinge{\tilde z}{\tilde p}{\tilde y}-\angk{z}{p}{y}$.
\end{enumerate}
Since 
\[\mangle\hinge{\tilde x}{\tilde p}{\tilde y}=\mangle\hinge{\tilde x}{\tilde p}{\tilde z}=\angk{x}{p}{z}\]
and
\[ \mangle\hinge{\tilde z}{\tilde p}{\tilde y}
=\pi-\mangle\hinge{\tilde z}{\tilde x}{\tilde p}
=\pi-\angk{z}{x}{p},\]
the first statement follows.
\qeds

\begin{thm}{Exercise}\label{ex:alex-lemma-cat}
Assume $p,x,y,z$ are as in Alexandrov's lemma.
Show that
\[\angk p x y
\ge
\angk p x z + \angk p z y,\]
with equality if and only if the expressions in \ref{SHORT.lem-alex-difference} and \ref{SHORT.lem-alex-angle} vanish.
\end{thm}

Note that if $p\in\left]x y\right[$, then $\angk pxy=\pi$.
Applying Alexandrov's lemma and $\CBB(0)$ comparison, we get the following claim and its corollary.

\begin{thm}{Claim}\label{clm:angle-mono}
If $p,x,y,z$ are points in a $\CBB(0)$ such that $p\in\left]x y\right[$, then 
\[\angk xyz\le \angk xpz.\]
\end{thm}

\begin{wrapfigure}{r}{25mm}
\vskip-0mm
\centering
\includegraphics{mppics/pic-750}
\end{wrapfigure}

\begin{thm}{Exercise}\label{ex:noncreasing}
Let $\hinge p x y$ be a hinge in a $\CBB(0)$ space.
Consider the function
\[f\:(\dist{p}{\bar x}{},\dist{p}{\bar y}{})\mapsto \angk p{\bar x}{\bar y},\]
where $\bar x\in\left]p x\right]$ and $\bar y\in\left]p y\right]$.
Show that $f$ is nonincreasing in each argument.
\end{thm}

Note that \ref{ex:noncreasing} implies the following generalization of \ref{thm:poly-cbb}.

\begin{thm}{Claim}\label{clm:angle-defined}
For any hinge $\hinge p x y$ in a $\CBB(0)$ space,
the angle measure $\mangle\hinge p x y$ is defined, and
\[\mangle\hinge p x y\ge \angk p x y.\]

\end{thm}

\begin{thm}{Exercise}\label{ex:0-angle}
Let $\hinge p x y$ be a hinge in a $\CBB(0)$ space.
Suppose $\mangle\hinge p x y=0$ show that $[px]\subset [py]$ or $[py]\subset [px]$.
\end{thm}

\begin{thm}{Exercise}\label{ex:pi-angle}
Let $[xy]$ be a geodesic in a $\CBB(0)$ space.
Suppose $z\in \left]xy\right[$ show that there is a unique geodesic $[xz]$ and $[xz]\subset [xy]$.
\end{thm}


\begin{thm}{Exercise}\label{ex:adjacent-CBB}
Let $\hinge pxz$ and $\hinge pyz$ be adjacent hinges in a $\CBB(0)$ space.
Show that
\[\mangle\hinge pxz+\mangle\hinge pyz= \pi.\]
\end{thm}


\begin{thm}{Exercise}
\label{ex:pxyvw}
Let 
$p,x,y$ in a $\CBB(0)$ space
and $v,w\in \left]xy\right[$.
Show that  
\[
\angk xyp=\angk xvp
\quad\Longleftrightarrow\quad
\angk xyp=\angk xwp.
\]

\end{thm}

Recall that a \index{triangle}\emph{triangle} $\trig xyz$ in a space $\spc{X}$ 
is a triple of minimizing geodesics $[xy]$, $[yz]$, and $[zx]$.
Consider the  model triangle $\trig{\tilde x}{\tilde y}{\tilde z}=\modtrig{}({x}{y}{z})_{\EE^2}$ in the Euclidean plane.
The \index{natural map}\emph{natural map} $\trig{\tilde x}{\tilde y}{\tilde z}\to \trig{x}{y}{z}$ 
sends a point $\tilde p\in[\tilde x\tilde y]\cup[\tilde y\tilde z]\cup[\tilde z\tilde x]$ to the corresponding point $p\in[ x y]\cup[y z]\cup[ z x]$;
that is, if $\tilde p$ lies on $[\tilde y\tilde z]$,
then $p\in [y z]$ and $\dist{\tilde y}{\tilde p}{}=\dist{y}{p}{}$ (and therefore $\dist{\tilde z}{\tilde p}{}=\dist{z}{p}{}$).
 
\begin{thm}{Definition}\label{def:k-thin-}
A triangle $\trig{x}{y}{z}$ in the metric space $\spc{X}$ 
is called \index{thin triangle}\emph{thin} (or \index{fat triangle}\emph{fat}) if the natural map $\modtrig{}({x}{y}{z})_{\EE^2}\to \trig{x}{y}{z}$ is distance nonincreasing (or respectively distance nondecreasing).

\end{thm}

\begin{thm}{Exercise}\label{ex:fat}
Show that any triangle in a $\CBB(0)$ space is fat.
\end{thm}

\section{Comments}

All the discussed statements admit natural generalizations to $\CBB(\kappa)$ spaces.
Most of the time the proof is the same with uglier formulas.
However, for the $\CBB(1)$ case in \ref{thm:CBB/G} one needs to assume in addition that space has intrinsic metric and the proof requires the globalization theorem which will be discussed later.


